\pdfoutput=1


\documentclass[12pt]{article}

\usepackage{amssymb}
\usepackage{amsfonts}
\usepackage{amsmath}
\usepackage{fancyhdr} 

\DeclareMathOperator{\B}{B}
\DeclareMathOperator{\Conf}{Conf}
\DeclareMathOperator{\Gr}{Gr}
\DeclareMathOperator{\Id}{Id}
\DeclareMathOperator{\Li}{Li}
\DeclareMathOperator{\Lie}{Lie}

\def\ket#1{\langle #1 \rangle}


\makeindex
\oddsidemargin -0.04cm \evensidemargin -0.04cm
\topmargin -0.25cm \textwidth 16.59cm \textheight 20.5cm \headheight 15pt

\begin{document}

\thispagestyle{fancyplain}
 
\fancyhf{}
 
\cfoot{\fancyplain{}{\thepage}}

\lhead{\textbf{Working through $A_2$ and $f_{A_2}$ in detail} \hfill \today}

We wish to define a function $f_{A_2}(x,y)$. It is a function of two variables, and is smooth and real for $x,y>0$. 

\begin{equation}
	\sigma_{A_2} = \mathcal{X}_i \to \mathcal{X}_{i+1}, \qquad \tau_{A_2} = \mathcal{X}_i \to \mathcal{X}_{6-i}
\end{equation}

\begin{multline}
	\sum_{\text{skew-dihedral}} \Li_{(2, 2)}(-1/x_i, -1/x_{i+2}) - \Li_{(1, 3)}(-1/x_i, -1/x_{i+2})
	-6\Li_3(-x_i)\log(x_{i+2})\\+\Li_2(-x_i)\log(x_{i+2})\left(3\log(x_i)+\log(x_{i+2})-\log(x_{i+1})\right)+\frac12\log(x_{i})^2\log(x_{i+1})\log(x_{i-2})
\end{multline}

\begin{equation}
	\sum_{\text{skew-dihedral}}^{A_2} f= \sum_{k=1}^5 \left(\left(\sigma_{A_2}\right)^k - \left(\sigma_{A_2}\right)^k\circ\tau_{A_2}\right) f
\end{equation}

\begin{equation}
	\sum_{\text{skew-dihedral}}^{A_2} f(x,y)=\sum_{i=1}^5f(\mathcal{X}_i,1/\mathcal{X}_{i+1})-f(\mathcal{X}_{6-i},1/\mathcal{X}_{5-i})
\end{equation}
where
\begin{equation}
	\mathcal{X}_1 = x, \quad \mathcal{X}_2 = 1/y, \quad \mathcal{X}_i = (1+\mathcal{X}_{i-1})/\mathcal{X}_{i-2}
\end{equation}



\end{document}
