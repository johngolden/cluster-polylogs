\pdfoutput=1


\documentclass[12pt]{article}

\usepackage{amssymb}
\usepackage{amsfonts}
\usepackage{amsmath}
\usepackage{color}
\usepackage[all]{xy}
\usepackage{fancyhdr} 

\DeclareMathOperator{\B}{B}
\DeclareMathOperator{\Conf}{Conf}
\DeclareMathOperator{\Gr}{Gr}
\DeclareMathOperator{\Id}{Id}
\DeclareMathOperator{\Li}{Li}
\DeclareMathOperator{\Lie}{Lie}

\def\ket#1{\langle #1 \rangle}


\makeindex
\oddsidemargin -0.04cm \evensidemargin -0.04cm
\topmargin -0.25cm \textwidth 16.59cm \textheight 20.5cm \headheight 15pt

\begin{document}

\thispagestyle{fancyplain}
 
\fancyhf{}
 
\cfoot{\fancyplain{}{\thepage}}

\lhead{\textbf{Collinear limits of $A_5$ and $D_5$ functions} \hfill \today}

The nonclassical portion of $f_{A_5}$ vanishes under the $7\to6$ collinear limit for any of the $A_5$ subalgebras of $\Gr(4,7)$. This is quite nice. 

For $D_5$ things are more complicated. None of the 14 $f_{D_5}$s vanish in the collinear limit (from here on out I am implicitly referring not to the whole function but to the nonclassical portion). Let me start by taking my `base' $D_5$ as
\begin{equation}
D_5^{(0,0)} = 	\begin{gathered}
    \begin{xy} 0;<1pt,0pt>:<0pt,-1pt>::
      (0,15) *+{\frac{\langle 1234\rangle  \langle 1267\rangle
   }{\langle 1237\rangle  \langle 1246\rangle }} ="1",
      (90,15) *+{-\frac{\langle 1247\rangle  \langle 3456\rangle
   }{\langle 4(12)(35)(67)\rangle }} ="2",
      (200,15) *+{\frac{\langle 1246\rangle  \langle 1345\rangle  \langle
   4567\rangle }{\langle 1245\rangle  \langle
   1467\rangle  \langle 3456\rangle }} ="3",
      (260,0) *+{\color{white} x_i} ="4",
      (260,30) *+{\color{white} x_i} ="5",
      (302,0) *+{-\frac{\langle 4(12)(35)(67)\rangle }{\langle
   1234\rangle  \langle 4567\rangle }} ="6",
      (302,30) *+{-\frac{\langle 1567\rangle  \langle
   4(12)(35)(67)\rangle }{\langle 1267\rangle  \langle
   1345\rangle  \langle 4567\rangle }} ="7",
      "1", {\ar"2"},
      "2", {\ar"3"},
      "3", {\ar"4"},
      "3", {\ar"5"},
    \end{xy}
    \end{gathered} 
\end{equation}
All other $D_5$s appear as cyclic+flip images of $D_5^{(0,0)}$, which I'll denote by 
\begin{equation}
	D_5^{(i,j)} = \sigma_{D_5}^i \circ \tau_{D_5}^j (D_5^{(0,0)}).	
\end{equation} 
From here on out I'll refer to $f_{D_5}^{D_5^{(i,j)}}$ simply by $(i,j)$. In this notation, we have 
\begin{itemize}
	\item $(3, 0)- (5, 1)$ vanishes in the collinear limit but is ill-defined term-by-term.
	\item $(0, 1) - (5, 0)$ vanishes in the collinear limit but is ill-defined term-by-term.
\end{itemize}
And the each of the remaining 10 $D_5$s have nonzero and well-defined collinear limit individually. They can cancel off each other in 4 linearly independent ways, for example: 
	\begin{itemize}
		\item $-(0, 0) - (1, 0) + (1, 1) + (4, 0) + (4, 1)$
		\item $-(0, 0) - (1, 0) + (1, 1) + (2, 0) + (4, 0)$
		\item $-(0, 0) - (1, 0) + (2, 1) + (3, 1) + (4, 0) - (6, 1)$
		\item $-(1, 1) + (6, 0)$
	\end{itemize}


\end{document}
