
\pdfoutput=1
\documentclass[11pt, reqno,preprint]{article}
\usepackage{jheppub}
\usepackage{epsfig}
\usepackage{amssymb}
\usepackage{amsmath}
\usepackage{mathrsfs}
%\usepackage{cite}
\usepackage{hyperref}
\usepackage{multirow}
\usepackage{feynmp-auto}

\def\be{\begin{equation}}
\def\ee{\end{equation}}
\def\ba{\begin{eqnarray}}
\def\ea{\end{eqnarray}}
\newcommand{\bea}{\begin{eqnarray}}
\newcommand{\eea}{\end{eqnarray}}
\def\nl{\nonumber\\}
\def\Li{\textrm{Li}}
\def\l{\langle}
\def\r{\rangle}
\def\eps{\epsilon}
\def\uo{\underline{0}}
\def\uone{\underline{1}}
\def\ut{\underline{t}}
\def\z{x}
\def\zb{y}
\newcommand{\del}{\partial}
\def\fig#1{fig.~{\ref{#1}}}
\def\Fig#1{Fig.~{\ref{#1}}}
\def\figs#1#2{figs.~{\ref{#1}} and {\ref{#2}}}
\def\Figs#1#2{Figs.~{\ref{#1}} and {\ref{#2}}}
\def\Sect#1{Section~{\ref{#1}}}
\def\sect#1{section~{\ref{#1}}}
\def\eqn#1{eq.~(\ref{#1})}
\def\Eqn#1{Equation~(\ref{#1})}
\def\eqns#1#2{eqs.~(\ref{#1}) and~(\ref{#2})}
\def\Eqns#1#2{Eqs.~(\ref{#1}) and~(\ref{#2})}
\def\tab#1{table~{\ref{#1}}}
\def\Tab#1{Table~{\ref{#1}}}
\def\tabs#1#2{table~{\ref{#1}} and~{\ref{#1}}}
\def\Tabs#1#2{Table~{\ref{#1}} and~{\ref{#1}}}
\def\Eqn#1{Equation~(\ref{#1})}
\def\eqn#1{eq.~(\ref{#1})}
\def\eqns#1#2{eqs.~(\ref{#1}) and~(\ref{#2})}
\def\Eqns#1#2{Eqs.~(\ref{#1}) and~(\ref{#2})}

\newcommand{\cP}{{\cal P}}
\def\lr{\leftrightarrow}

\def\draftnote#1{{\bf [#1]}}


\def\blue#1{{\color{blue}#1}}

\title{Steinmann Relations and the Two-Loop MHV Amplitude in Eight-Particle Kinematics}

\author{John~Golden,$^{1,2}$}
\author{Andrew~J.~McLeod$^{2,3,4}$}


\affiliation{$^1$ Michigan Center for Theoretical Physics and
Randall Laboratory of Physics, Department of Physics,
University of Michigan
Ann Arbor, MI 48109, USA}

\affiliation{$^2$ Kavli Institute for Theoretical Physics, 
UC Santa Barbara, Santa Barbara, CA 93106, USA}

\affiliation{$^3$ SLAC National Accelerator Laboratory,
Stanford University, Stanford, CA 94309, USA}

\affiliation{$^4$ Niels Bohr International Academy, Blegdamsvej 17, 2100 Copenhagen, Denmark}

\abstract{We present the full functional form of the two-loop eight-point MHV amplitude in the planar limit of maximally supersymmetric Yang-Mills theory, in terms of cluster polylogarithms. We also compute the two BDS-like ans\"atze that can be formulated in eight-particle kinematics, and find that \dots}

%\preprint{
%\begin{flushright} DESY ??--??? \\ SLAC--PUB--?????
%\end{flushright}
%}


\begin{document}
\hypersetup{pageanchor=false}
\maketitle
\hypersetup{pageanchor=true}
\begin{fmffile}{feyndiags}


\section{Introduction}

\section{Promoting $R_8^{(2)}$ from Symbol to Function}

Briefly describe the method outlined in~\cite{Golden:2014xqf} for upgrading the $n$-point two-loop MHV symbol to a function.

\section{The Steinmann Relations for Eight Particles}

When the number of gluons $n$ is not a multiple of 4, the BDS-like ansatz is unique because there exists only a single decomposition
\begin{equation}
A^{\text{BDS}}_{n} = A^{\text{BDS-like}}_{n}(\{s_{i,i+1}\}) + Y_{n}(\{u_i\}), \quad n\neq4K,
\end{equation}
such that the kinematic dependence of $A^{\text{BDS-like}}_{n}$ involves only two-particle Mandelstam invariants and $Y_{n}$ is a function of dual conformal invariant cross ratios~\cite{Yang:2010az}. However, when $n$ is a multiple of 4, no decomposition of this type exists, and we are forced to consider multiple BDS-like ans\"atze in order to expose all Steinmann relations between higher-particle Mandelstam invariants. Even so, in eight particle kinematics (where this issue first arises), there are two natural normalization choices we might consider. These correspond to letting the BDS-like ansatz depend on either three- or four-particle Mandelstam invariants in addition to two-particle invariants. We therefore consider a pair of Bose-symmetric BDS-like ans\"atze, respectively satisfying
\begin{align}
{\cal A}^{\text{BDS}}_{n} &= {}^3 {\cal A}^{\text{BDS-like}}_{8}(\{s_{i,i+1}\}, \{s_{i,i+1,i+2,i+3}\}) + {}^3 Y_{8}(\{u_i\}), \label{bds_like_3} \\
{\cal A}^{\text{BDS}}_{n} &= {}^4 {\cal A}^{\text{BDS-like}}_{8}(\{s_{i,i+1}\}, \{s_{i,i+1,i+2}\}) + {}^4 Y_{8}(\{u_i\}). \label{bds_like_4}
\end{align}
In fact, the functions ${}^3 A^{\text{BDS-like}}_{8} $ and ${}^4 A^{\text{BDS-like}}_{8}$ are not uniquely fixed by this choice, as there exists a family of Bose-symmetric solutions to these decompositions. 
 
Any way this final degree of freedom is fixed, ${}^3 A^{\text{BDS-like}}_{8}$ and ${}^4 A^{\text{BDS-like}}_{8}$ give rise to a pair of BDS-like normalized amplitudes that not only retain Bose symmetry, but realize a subset of the Steinmann relations. In particular, defining
\begin{equation}
{}^X {\cal E}_8 \equiv \frac{{\cal A}_8^{\text{MHV}}}{{}^X {\cal A}^{\text{BDS-like}}_{8}} = \exp\left[ R_8 - {}^X Y_8 \frac{\Gamma_{\text{cusp}}}{4} \right] \label{BDS_like_amplitude}
\end{equation}
for any label $X$, we expect that ${}^3 {\cal E}_8$ should satisfy Steinmann relations between all partially overlapping pairs of three-particle invariants, while ${}^4 {\cal E}_8$ should satisfy Steinmann relations between all partially overlapping pairs of four-particle invariants. That is, ${}^3 {\cal E}_8$ is expected to satisfy the relations
\begin{align}
\text{Disc}_{s_{i+1,i+2,i+3}}\left[\text{Disc}_{s_{i,i+1,i+2}} ({}^3 {\cal E}_8) \right] &= 0, \label{stein33_1} \\
\text{Disc}_{s_{i+2,i+3,i+4}}\left[\text{Disc}_{s_{i,i+1,i+2}} ({}^3 {\cal E}_8) \right] &= 0, \label{stein33_2}
\end{align}
for all $i$, while ${}^4 {\cal E}_8$ is expected to satisfy
\begin{align}
\text{Disc}_{s_{i+1,i+2,i+3,i+4}}\left[\text{Disc}_{s_{i,i+1,i+2,i+3}} ({}^4 {\cal E}_8) \right] &= 0, \label{stein44_1} \\
\text{Disc}_{s_{i+2,i+3,i+4,i+5}}\left[\text{Disc}_{s_{i,i+1,i+2,i+3}} ({}^4 {\cal E}_8) \right] &= 0, \label{stein44_2} \\
\text{Disc}_{s_{i+3,i+4,i+5,i+6}}\left[\text{Disc}_{s_{i,i+1,i+2,i+3}} ({}^4 {\cal E}_8) \right] &= 0. \label{stein44_3}
\end{align}
However, conditions~\eqref{stein33_1} through~\eqref{stein44_3} don't exhaust the set of Steinmann relations obeyed by generic eight-particle amplitudes---there are also Steinmann relations between partially overlapping three- and four-particle invariants. In order to make use of these additional relations, we use the fact that it proves possible to define a BDS-like ansatz that depends on all but one of the four-particle invariants (and on no three-particle invariants). That is, we decompose the BDS ansatz as
\begin{equation}
{\cal A}^{\text{BDS}}_{n} = {}^{3,j} {\cal A}^{\text{BDS-like}}_{8}(\{s_{i,i+1}\}, \{s_{i,i+1,i+2,i+3} \neq s_{j,j+1,j+2,j+3} \}) + {}^{3,j} Y_{8}(\{u_i\}), \label{bds_like_3q}
\end{equation}
and in so doing define a BDS-like normalized amplitude that satisfies the Steinmann relations
\begin{align}
\text{Disc}_{s_{j+2,j+3,j+4}}\left[\text{Disc}_{s_{j,j+1,j+2,j+3}} ({}^{3,j} {\cal E}_8) \right] &= 0, \label{stein34_1} \\
\text{Disc}_{s_{j+3,j+4,j+5}}\left[\text{Disc}_{s_{j,j+1,j+2,j+3}} ({}^{3,j} {\cal E}_8) \right] &= 0, \label{stein34_2} \\
\text{Disc}_{s_{j-1,j,j+1}}\left[\text{Disc}_{s_{j,j+1,j+2,j+3}} ({}^{3,j} {\cal E}_8) \right] &= 0, \label{stein34_3} \\
\text{Disc}_{s_{j-2,j-1,j}}\left[\text{Disc}_{s_{j,j+1,j+2,j+3}} ({}^{3,j} {\cal E}_8) \right] &= 0, \label{stein34_4} \\
\text{Disc}_{s_{i+1,i+2,i+3}}\left[\text{Disc}_{s_{i,i+1,i+2}} ({}^{3,j} {\cal E}_8) \right] &= 0, \label{stein34_5} \\
\text{Disc}_{s_{i+2,i+3,i+4}}\left[\text{Disc}_{s_{i,i+1,i+2}} ({}^{3,j} {\cal E}_8) \right] &= 0, \label{stein34_6}
\end{align}
where we note that the relations~\eqref{stein34_3} and~\eqref{stein34_3} were also satisfied by ${}^3 {\cal E}_8$. It is not possible to make the function ${}^{3,j} {\cal A}^{\text{BDS-like}}_{8}$ appearing in the decomposition~\eqref{bds_like_3q} fully Bose-symmetric, but we can require that it is invariant under the dihedral flip that maps $s_{i \dots l} \rightarrow s_{9-i \dots 9 - l}$. There is again a family of solutions to this choice and~\eqref{bds_like_3q}, rather than a unique function defined by these constraints. 

For any choice that fixes the remaining degrees of freedom in ${}^{3,j} {\cal A}^{\text{BDS-like}}_{8}$, momentum conservation implies that ${}^{3,j+4} {\cal A}^{\text{BDS-like}}_{8} = {}^{3,j} {\cal A}^{\text{BDS-like}}_{8}$. This means that every eight-point Steinmann relation is manifestly respected by at least one of the five amplitudes $\{{}^4 {\cal E}_8, {}^{3,1} {\cal E}_8 , {}^{3,2} {\cal E}_8 , {}^{3,3} {\cal E}_8 , {}^{3,4} {\cal E}_8 \}$. However, in practice it may be easier to include ${}^{3} {\cal E}_8$ in the set of functions one considers, since it manifests all Steinmann relations between partially overlapping three-particle invariants in a Bose-symmetric way.

To take advantage of the Steinmann relations, it's convenient to work in terms of symbol letters that isolate different Mandelstam invariants. There are twelve independent dual conformally invariant cross ratios that can appear in these symbols
\begin{align}
u_1 &= \frac{s_{12} s_{4567}}{s_{123} s_{812}}, \quad \text{and cyclic (8-orbit)} \\
u_9 &= \frac{s_{123} s_{567}}{s_{1234} s_{4567}}, \quad \text{and cyclic (4-orbit).}
\end{align}
It is not possible to isolate all eight three-particle Mandelstam invariants and all four-particle Mandelstam invariants simultaneously with a judicious choice of twelve symbol letters. (Of course, there will be more symbol letters, but we're here only concerned with the twelve will appear in the first entry.) However, we can choose different sets of letters that make either all the Steinmann relations in ${}^4 {\cal E}_8$, or all the Steinmann relations in ${}^{3,j} {\cal E}_8$ and ${}^3 {\cal E}_8$, transparent. 

To make all Steinmann relations between four-particle invariants manifest, we can choose
\begin{align}
{}^4 d_1 &= u_2 \ u_6 = \frac{s_{23} \ s_{67} \ (s_{1234})^2}{s_{123} \ s_{234} \ s_{567} \ s_{678}}, \quad \text{and cyclic (4-orbit)} \\
{}^4 d_5 &= u_2/u_6 = \frac{s_{23} \ s_{567} \ s_{678}}{s_{67} \ s_{123} \ s_{234}}, \quad \text{and cyclic (4-orbit)} \\
{}^4 d_9 &= u_1 \ u_2 \ u_5 \ u_6 \ u_9^2 = \frac{s_{12} \ s_{23} \ s_{56} \ s_{67}}{s_{234} \ s_{456} \ s_{678} \ s_{812}}, \quad \text{and cyclic (4-orbit)}
\end{align}
in which case ${}^4 d_1, {}^4 d_2, {}^4 d_3$, and ${}^4 d_4$ each contain a different four-particle Mandelstam invariant, while the other letters are only composed of two- and three-particle invariants. The extended Steinmann relations then tell us that ${}^4 d_1, {}^4 d_2, {}^4 d_3$, and ${}^4 d_4$ can never appear next to each other (but each can still appear next to themselves).

To make all Steinmann relations between three-particle invariants and between three and four-particle invariants manifest, we can choose
\begin{align}
{}^3 d_1 &= \frac{u_1 \ u_2 \ u_4 \ u_7}{u_3 \ u_5 \ u_6 \ u_8 \ u_9^2} = \frac{s_{12} \ s_{23} \ s_{45} \ s_{78} \ (s_{1234})^2 \ (s_{4567})^2}{s_{34} \ s_{56} \ s_{67} \ s_{81} \ (s_{123})^2}, \quad \text{and cyclic (8-orbit)} \\
{}^3 d^4_9 &= u_1 \ u_5 \ u_9 \ u_{12} = \frac{s_{12} \ s_{56}}{s_{1234} \ s_{3456}}, \quad \text{and cyclic (4-orbit)}
\end{align}
in which case ${}^3 d_1$ through ${}^3 d_8$ each contain a different three-particle Mandelstam invariant, as well as four-particle Mandelstams that they are not disbarred from appearing next to it in the symbol. The remaining four letters only contain two- and four-particle invariants. In these letters, conditions~\eqref{stein34_5} and~\eqref{stein34_6} tell us that ${}^3 d_7, {}^3 d_8, {}^3 d_2$, and ${}^3 d_2$ can never appear next to ${}^3 d_1$ in the symbol of ${}^{3,j} {\cal E}_8$ (plus the cyclic images of these restrictions). If we now specialize to ${}^{3,1} {\cal E}_8$ so as to be concrete, conditions~\eqref{stein34_1} through~\eqref{stein34_4} give us the additional restrictions that none of ${}^3 d_1, {}^3 d_5, {}^3 d_9$ and ${}^3 d_{10}$ can ever appear next to ${}^3 d_3, {}^3 d_4, {}^3 d_7,$ or ${}^3 d_8$. These are the restrictions given by the Steinmann relations involving $s_{1234}$ and one of $s_{781}, s_{812}, s_{345}$, or $s_{456}$. The other Steinmann relations between three- and four-particle invariants will not be respected by ${}^{3,1} {\cal E}_8$, since ${}^{3,j} {\cal A}^{\text{BDS-like}}_{8}$ depends on $s_{2345}, s_{3456},$ and $s_{4567}$.

 
\bibliographystyle{ieeetr}

\bibliography{eight_particle_two_loop}

\end{fmffile}
\end{document}
