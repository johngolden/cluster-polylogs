
\pdfoutput=1
\documentclass[11pt, reqno,preprint]{article}
\usepackage{jheppub}
\usepackage{epsfig}
\usepackage{amssymb}
\usepackage{amsmath}
\usepackage{mathrsfs}
%\usepackage{cite}
\usepackage{hyperref}
\usepackage{multirow}
\usepackage{feynmp-auto}

\def\be{\begin{equation}}
\def\ee{\end{equation}}
\def\ba{\begin{eqnarray}}
\def\ea{\end{eqnarray}}
\newcommand{\bea}{\begin{eqnarray}}
\newcommand{\eea}{\end{eqnarray}}
\def\nl{\nonumber\\}
\def\Li{\textrm{Li}}
\def\l{\langle}
\def\r{\rangle}
\def\eps{\epsilon}
\def\uo{\underline{0}}
\def\uone{\underline{1}}
\def\ut{\underline{t}}
\def\z{x}
\def\zb{y}
\newcommand{\del}{\partial}
\def\fig#1{fig.~{\ref{#1}}}
\def\Fig#1{Fig.~{\ref{#1}}}
\def\figs#1#2{figs.~{\ref{#1}} and {\ref{#2}}}
\def\Figs#1#2{Figs.~{\ref{#1}} and {\ref{#2}}}
\def\Sect#1{Section~{\ref{#1}}}
\def\sect#1{section~{\ref{#1}}}
\def\eqn#1{eq.~(\ref{#1})}
\def\Eqn#1{Equation~(\ref{#1})}
\def\eqns#1#2{eqs.~(\ref{#1}) and~(\ref{#2})}
\def\Eqns#1#2{Eqs.~(\ref{#1}) and~(\ref{#2})}
\def\tab#1{table~{\ref{#1}}}
\def\Tab#1{Table~{\ref{#1}}}
\def\tabs#1#2{table~{\ref{#1}} and~{\ref{#1}}}
\def\Tabs#1#2{Table~{\ref{#1}} and~{\ref{#1}}}
\def\Eqn#1{Equation~(\ref{#1})}
\def\eqn#1{eq.~(\ref{#1})}
\def\eqns#1#2{eqs.~(\ref{#1}) and~(\ref{#2})}
\def\Eqns#1#2{Eqs.~(\ref{#1}) and~(\ref{#2})}

\newcommand{\cP}{{\cal P}}
\def\lr{\leftrightarrow}

\def\draftnote#1{{\bf [#1]}}


\def\blue#1{{\color{blue}#1}}

\title{Steinmann Relations and the Two-Loop MHV Amplitude in Eight-Particle Kinematics}

\author{John~Golden,$^{1,2}$}
\author{Andrew~J.~McLeod$^{2,3,4}$}


\affiliation{$^1$ Michigan Center for Theoretical Physics and
Randall Laboratory of Physics, Department of Physics,
University of Michigan
Ann Arbor, MI 48109, USA}

\affiliation{$^2$ Kavli Institute for Theoretical Physics, 
UC Santa Barbara, Santa Barbara, CA 93106, USA}

\affiliation{$^3$ SLAC National Accelerator Laboratory,
Stanford University, Stanford, CA 94309, USA}

\affiliation{$^4$ Niels Bohr International Academy, Blegdamsvej 17, 2100 Copenhagen, Denmark}

\abstract{We present the full functional form of the two-loop eight-point MHV amplitude in the planar limit of maximally supersymmetric Yang-Mills theory, in terms of cluster polylogarithms. We also compute the two BDS-like ans\"atze that can be formulated in eight-particle kinematics, and find that \dots}

%\preprint{
%\begin{flushright} DESY ??--??? \\ SLAC--PUB--?????
%\end{flushright}
%}


\begin{document}
\hypersetup{pageanchor=false}
\maketitle
\hypersetup{pageanchor=true}
\begin{fmffile}{feyndiags}


\section{Introduction}

\section{Promoting $R_8^{(2)}$ from Symbol to Function}

Briefly describe the method outlined in~\cite{Golden:2014xqf} for upgrading the $n$-point two-loop MHV symbol to a function.

\section{The Steinmann Relations for Eight Particles}

When the number of gluons $n$ is not a multiple of 4, the BDS-like ansatz is unique because there exists only a single decomposition
\begin{equation}
A^{\text{BDS}}_{n} = A^{\text{BDS-like}}_{n}(\{s_{ij}\}) + Y_{n}(\{u_i\}), \quad n\neq4K,
\end{equation}
such that the kinematic dependence of $A^{\text{BDS-like}}_{n}$ involves only two-particle Mandelstam invariants and $Y_{n}$ is a function of dual conformal invariant cross ratios~\cite{Yang:2010az}. However, when $n$ is a multiple of 4, no decomposition of this type exists, and we are forced to consider multiple BDS-like ans\"atze in order to expose all Steinmann relations between higher-particle Mandelstam invariants. However, in eight particle kinematics (where this issue first arises), there are two natural normalization choices we might consider. These correspond to letting the BDS-like ansatz depend on either three- or four-particle Mandelstam invariants in addition to two-particle invariants. We therefore consider a pair of Bose-symmetric BDS-like ans\"atze, respectively satisfying
\begin{align}
{\cal A}^{\text{BDS}}_{n} &= {}^3 {\cal A}^{\text{BDS-like}}_{8}(\{s_{ij}\}, \{s_{ijkl}\}) + {}^3 Y_{8}(\{u_i\}), \label{bds_like_3} \\
{\cal A}^{\text{BDS}}_{n} &= {}^4 {\cal A}^{\text{BDS-like}}_{8}(\{s_{ij}\}, \{s_{ijk}\}) + {}^4 Y_{8}(\{u_i\}). \label{bds_like_4}
\end{align}
In fact, these decompositions only single out a one-parameter family of Bose-symmetric solutions for ${}^3 A^{\text{BDS-like}}_{8} $ and ${}^4 A^{\text{BDS-like}}_{8}$, so these functions are not uniquely fixed by this choice. \draftnote{Do we gauge fix for convenience and then provide the full solution in an appendix?}
 
Any choice for the functions ${}^3 A^{\text{BDS-like}}_{8}$ and ${}^4 A^{\text{BDS-like}}_{8}$ allow us to define a pair of BDS-like normalized amplitudes that retain Bose symmetry and realize a subset of the Steinmann relations. In particular, defining
\begin{equation}
{}^X {\cal E}_8 \equiv \frac{{\cal A}_8^{\text{MHV}}}{{}^X {\cal A}^{\text{BDS-like}}_{8}} = \exp\left[ R_8 - {}^X Y_8 \frac{\Gamma_{\text{cusp}}}{4} \right] \label{BDS_like_amplitude}
\end{equation}
for any label $X$, we expect that ${}^3 {\cal E}_8$ should satisfy Steinmann relations between all partially overlapping pairs of three-particle invariants, while ${}^4 {\cal E}_8$ should satisfy Steinmann relations between all partially overlapping pairs of four-particle invariants. That is, ${}^3 {\cal E}_8$ is expected to satisfy the relations
\begin{align}
\text{Disc}_{s_{i+1,i+2,i+3}}\left[\text{Disc}_{s_{i,i+1,i+2}} ({}^3 {\cal E}_8) \right] &= 0, \label{stein33_1} \\
\text{Disc}_{s_{i+2,i+3,i+4}}\left[\text{Disc}_{s_{i,i+1,i+2}} ({}^3 {\cal E}_8) \right] &= 0, \label{stein33_2}
\end{align}
for all $i$, while ${}^4 {\cal E}_8$ is expected to satisfy
\begin{align}
\text{Disc}_{s_{i+1,i+2,i+3,i+4}}\left[\text{Disc}_{s_{i,i+1,i+2,i+3}} ({}^4 {\cal E}_8) \right] &= 0, \label{stein44_1} \\
\text{Disc}_{s_{i+2,i+3,i+4,i+5}}\left[\text{Disc}_{s_{i,i+1,i+2,i+3}} ({}^4 {\cal E}_8) \right] &= 0, \label{stein44_2} \\
\text{Disc}_{s_{i+3,i+4,i+5,i+6}}\left[\text{Disc}_{s_{i,i+1,i+2,i+3}} ({}^4 {\cal E}_8) \right] &= 0. \label{stein44_3}
\end{align}
However, conditions~\eqref{stein33_1} through~\eqref{stein44_3} don't exhaust the set of Steinmann relations obeyed by generic eight-particle amplitudes---there are also Steinmann relations between partially overlapping three- and four-particle invariants. In order to make use of these additional relations, we use the fact that it proves possible to define a BDS-like ansatz that depends on all but one of the four-particle invariants (and on no three-particle invariants). That is, we decompose the BDS ansatz as
\begin{equation}
{\cal A}^{\text{BDS}}_{n} = {}^{3,q} {\cal A}^{\text{BDS-like}}_{8}(\{s_{ij}\}, \{s_{ijkl} \neq s_{qrst} \}) + {}^{3,q} Y_{8}(\{u_i\}), \label{bds_like_3q}
\end{equation}
and in so doing define a BDS-like normalized amplitude that satisfies the Steinmann relations
\begin{align}
\text{Disc}_{s_{i+1,i+2,i+3}}\left[\text{Disc}_{s_{i,i+1,i+2,i+3}} ({}^{3,i} {\cal E}_8) \right] &= 0, \label{stein34_1} \\
\text{Disc}_{s_{i+2,i+3,i+4}}\left[\text{Disc}_{s_{i,i+1,i+2,i+3}} ({}^{3,i} {\cal E}_8) \right] &= 0, \label{stein34_2} \\
\text{Disc}_{s_{i+1,i+2,i+3}}\left[\text{Disc}_{s_{i,i+1,i+2}} ({}^{3,q} {\cal E}_8) \right] &= 0, \label{stein34_3} \\
\text{Disc}_{s_{i+2,i+3,i+4}}\left[\text{Disc}_{s_{i,i+1,i+2}} ({}^{3,q} {\cal E}_8) \right] &= 0, \label{stein34_4}
\end{align}
where we note that~\eqref{stein34_3} and~\eqref{stein34_3} were also satisfied by ${}^3 {\cal E}_8$. It is not possible to make the function ${}^{3,q} {\cal A}^{\text{BDS-like}}_{8}$ appearing in the decomposition~\eqref{bds_like_3q} fully Bose-symmetric, but we can require that it is invariant under the dihedral flip that maps $s_{i \dots l} \rightarrow s_{9-i \dots 9 - l}$. This gives rise to a three-parameter set of solutions to~\eqref{bds_like_3q}. 

For any choice of ${}^{3,q} {\cal A}^{\text{BDS-like}}_{8}$, momentum conservation implies that ${}^{3,q+4} {\cal A}^{\text{BDS-like}}_{8} = {}^{3,q} {\cal A}^{\text{BDS-like}}_{8}$. This means that every eight-point Steinmann relation is manifestly respected by at least one of the five amplitudes $\{{}^4 {\cal E}_8, {}^{3,1} {\cal E}_8 , {}^{3,2} {\cal E}_8 , {}^{3,3} {\cal E}_8 , {}^{3,4} {\cal E}_8 \}$. However, in practice it may be easier to include ${}^{3} {\cal E}_8$ in the set of functions one considers, since it manifests all Steinmann relations between partially overlapping three-particle invariants in a Bose-symmetric way.

...how the Steinmann relations manifest at symbol level...

Note that by momentum conservation there are only four independent four-particle Mandelstam invariants, so all distinct four-particle invariants are disallowed from appearing in adjacent slots in the symbol.



\bibliographystyle{ieeetr}

\bibliography{eight_particle_two_loop}

\end{fmffile}
\end{document}
