\pdfoutput=1
\documentclass[11pt, reqno,preprint]{article}
\usepackage{jheppub}
\usepackage{epsfig}
\usepackage{amssymb}
\usepackage{amsmath}
\usepackage{tikz}
\usepackage{mathrsfs}
%\usepackage{cite}
\usepackage{hyperref}
\usepackage{multirow}
\usepackage{feynmp-auto}

\def\be{\begin{equation}}
\def\ee{\end{equation}}
\def\ba{\begin{eqnarray}}
\def\ea{\end{eqnarray}}
\newcommand{\bea}{\begin{eqnarray}}
\newcommand{\eea}{\end{eqnarray}}
\def\nl{\nonumber\\}
\def\Li{\textrm{Li}}
\DeclareMathOperator{\Gr}{Gr}
\def\l{\langle}
\def\r{\rangle}
\def\eps{\epsilon}
\def\uo{\underline{0}}
\def\uone{\underline{1}}
\def\ut{\underline{t}}
\def\z{x}
\def\zb{y}
\newcommand{\del}{\partial}
\def\fig#1{fig.~{\ref{#1}}}
\def\Fig#1{Fig.~{\ref{#1}}}
\def\figs#1#2{figs.~{\ref{#1}} and {\ref{#2}}}
\def\Figs#1#2{Figs.~{\ref{#1}} and {\ref{#2}}}
\def\Sect#1{Section~{\ref{#1}}}
\def\sect#1{section~{\ref{#1}}}
\def\eqn#1{eq.~(\ref{#1})}
\def\Eqn#1{Equation~(\ref{#1})}
\def\eqns#1#2{eqs.~(\ref{#1}) and~(\ref{#2})}
\def\Eqns#1#2{Eqs.~(\ref{#1}) and~(\ref{#2})}
\def\tab#1{table~{\ref{#1}}}
\def\Tab#1{Table~{\ref{#1}}}
\def\tabs#1#2{table~{\ref{#1}} and~{\ref{#1}}}
\def\Tabs#1#2{Table~{\ref{#1}} and~{\ref{#1}}}
\def\Eqn#1{Equation~(\ref{#1})}
\def\eqn#1{eq.~(\ref{#1})}
\def\eqns#1#2{eqs.~(\ref{#1}) and~(\ref{#2})}
\def\Eqns#1#2{Eqs.~(\ref{#1}) and~(\ref{#2})}
\def\drawOctagon{
\coordinate (P1) at (45:1);
\coordinate (P2) at (90:1);
\coordinate (P3) at (135:1);
\coordinate (P4) at (180:1);
\coordinate (P5) at (225:1);
\coordinate (P6) at (270:1);
\coordinate (P7) at (315:1);
\coordinate (P8) at (359:1);
\draw (P1) -- (P2) -- (P3) -- (P4) -- (P5) -- (P6) -- (P7) -- (P8) -- cycle;
}


\newcommand{\cP}{{\cal P}}
\def\lr{\leftrightarrow}

\def\draftnote#1{{\bf [#1]}}


\def\blue#1{{\color{blue}#1}}

\title{Steinmann Relations and the Two-Loop MHV Amplitude in Eight-Particle Kinematics}

\author{John~Golden,$^{1,2}$}
\author{Andrew~J.~McLeod$^{2,3,4}$}


\affiliation{$^1$ Leinweber  Center for Theoretical Physics and
Randall Laboratory of Physics, Department of Physics,
University of Michigan
Ann Arbor, MI 48109, USA}

\affiliation{$^2$ Kavli Institute for Theoretical Physics, 
UC Santa Barbara, Santa Barbara, CA 93106, USA}

\affiliation{$^3$ SLAC National Accelerator Laboratory,
Stanford University, Stanford, CA 94309, USA}

\affiliation{$^4$ Niels Bohr International Academy, Blegdamsvej 17, 2100 Copenhagen, Denmark}

\abstract{We present the full functional form of the two-loop eight-point MHV amplitude in the planar limit of maximally supersymmetric Yang-Mills theory, in terms of cluster polylogarithms. We also compute the two BDS-like ans\"atze that can be formulated in eight-particle kinematics, and find that \dots}

%\preprint{
%\begin{flushright} DESY ??--??? \\ SLAC--PUB--?????
%\end{flushright}
%}


\begin{document}
\hypersetup{pageanchor=false}
\maketitle
\hypersetup{pageanchor=true}
\begin{fmffile}{feyndiags}


\section{Introduction}

\section{Promoting \texorpdfstring{$R_8^{(2)}$}{R28} from Symbol to Function}

\subsection{Cluster Polylogarithms Review}

Briefly describe the method outlined in~\cite{Golden:2014xqf} for upgrading the $n$-point two-loop MHV symbol to a function.

\subsection{The \texorpdfstring{$A_5$}{A5} Function}
\begin{itemize}
	\item describe the $A_5$ cluster algebra
	\item what makes ``the'' $A_5$ function unique?
\end{itemize}

The $A_5$ cluster algebra is generated from the seed cluster
\begin{equation}
	x_1 \to x_2 \to x_3 \to x_4 \to x_5.
\end{equation}
The full $A_5$ algebra contains 132 clusters with 140 distinct $\mathcal{X}$-coordinates. 

Define:
\begin{equation}
	x_{i_1\ldots i_k} = \sum_{a=1}^k \prod_{b=1}^a x_{i_b} = x_{i_1}+x_{i_1}x_{i_2} + \ldots + x_{i_1}\cdots x_{i_k}.
\end{equation}

The $A_5$ cluster algebra has an eight-fold cyclic symmetry, which we denote by $\sigma$:
\begin{equation}
\begin{split}
	\sigma:\quad& 
		x_1\mapsto\frac{x_2}{1+x_{12}},~~	
		x_2\mapsto\frac{x_3\left(1+x_1\right)}{1+x_{123}},~~
		x_3\mapsto\frac{x_4 \left(1+x_{12}\right)}{1+x_{1234}},\\&
		x_4\mapsto\frac{x_5 \left(1+x_{123}\right)}{1+x_{12345}},~~
		x_5\mapsto\frac{1+x_{1234}}{x_1 x_2 x_3 x_4 x_5}.
\end{split}
\end{equation}
$A_5$ also has a two-fold flip symmetry, which we denote by $\tau$:
\begin{equation}
	\tau:\quad x_i\mapsto\frac{1}{x_{6-i}}.
\end{equation}
There are 56 distinct $A_2$ subalgebras in $A_5$ (56 = $\genfrac(){0pt}{1}{8}{5}$ = number of distinct pentagons inside an octagon), they can be parameterized by:
\begin{equation}
\begin{split}
	&\left\{x_1\to x_2,~~
	x_2\to x_3\left(1+x_4\right),~~
	x_2\left(1+x_3\right)\to \frac{x_3 x_4}{1+x_3}\right\} + \sigma,\\
	&\left\{x_2\to x_3,~~x_1 \left(1+x_2\right)\to \frac{x_2x_3}{1+x_2}\right\} + \sigma + \tau 
   \end{split}
\end{equation}
where by ``$+~\sigma$'' and ``$+~\sigma~+~\tau$'' I mean ``+ cyclic copies'' and ``+ cyclic and flip copies,'' respectively. This correspond to the geometries
\begin{center}
\begin{tikzpicture}
\drawOctagon
\draw[color=red] (P1) -- (P2) -- (P3) -- (P4) -- (P5) -- cycle;
\begin{scope}[xshift=3cm]
\drawOctagon
\draw[color=red] (P1) -- (P2) -- (P3) -- (P5) -- (P7) -- cycle;
\end{scope}
\begin{scope}[xshift=6cm]
\drawOctagon
\draw[color=red] (P1) -- (P2) -- (P4) -- (P5) -- (P7) -- cycle;
\end{scope}
\begin{scope}[yshift=-2.5cm,xshift=1.5cm]
\drawOctagon
\draw[color=red] (P1) -- (P2) -- (P3) -- (P4) -- (P6) -- cycle;
\end{scope}
\begin{scope}[yshift=-2.5cm, xshift=4.5cm]
\drawOctagon
\draw[color=red] (P1) -- (P2) -- (P3) -- (P5) -- (P6) -- cycle;
\end{scope}
\node[] at (9,0) (a) {(8 of each)};
\node[] at (9,-2.5) (a) {(16 of each)};
\end{tikzpicture}
\end{center}
\subsubsection*{The \texorpdfstring{$A_5$}{A5} function}
The $A_5$ function is a sum over two of the classes of $A_2$ subalgebras, $x_2\to x_3\left(1+x_4\right)$ and $x_1 \left(1+x_2\right)\to \frac{x_2x_3}{1+x_2}$, appropriately antisymmetrized so that the overall $f_{A_5}$ picks up a minus sign under both $\sigma$ and $\tau$. Explicitly, this is written
\begin{equation}
	f_{A_5} = \sum_{i=0}^7\sum_{j=0}^1(-1)^{i+j}\sigma^i\tau^j\left(\frac12 f_{A_2}\left(x_2\to x_3\left(1+x_4\right)\right) + f_{A_2}\left(x_1 \left(1+x_2\right)\to \frac{x_2x_3}{1+x_2}\right)\right).
\end{equation}
The factor of $\frac12$ in front of $f_{A_2}\left(x_2\to x_3\left(1+x_4\right)\right)$ is simply a symmetry factor, as it lives in an 8-cycle of $\{\sigma,\tau\}$.

Which pentagonalizations do these two $A_2$s correspond to?

A large outstanding question is: why these two $A_2$s? I have no justification/argument for them other than they work. 

The $A_5$ does not have $B_2\wedge B_2$ expressible in terms of $A_1 \times A_1$ subalgebras (i.e. it is not expressible as a sum of $f_{A_3}$s). This is very surprising!

\subsection{Identifying \texorpdfstring{$A_5$}{A5} subalgebras in \texorpdfstring{$\Gr(4,8)$}{Gr(4,8)}}
Comment on infinite nature of $\Gr(4,8)$, define ``good'' (in the two-loop MHV sense) subalgebras, describe algorithm for finding ``good'', and describe the 56 good $A_5$s in $\Gr(4,8)$.

There are 56 good $A_5$s in $\Gr(4,8)$. They are generated by
\begin{align}
	&\frac{\langle 1238\rangle  \langle 1256\rangle }{\langle
   1235\rangle  \langle 1268\rangle }\to \frac{\langle
   1236\rangle  \langle 2345\rangle }{\langle 1234\rangle
    \langle 2356\rangle }\to \frac{\langle 1235\rangle 
   \langle 3456\rangle }{\langle 1356\rangle  \langle
   2345\rangle }\to \frac{\langle 1567\rangle  \langle
   2356\rangle }{\langle 1256\rangle  \langle 3567\rangle
   }\to \frac{\langle 1356\rangle  \langle 4567\rangle
   }{\langle 1567\rangle  \langle 3456\rangle }\\
   &\frac{\langle 1238\rangle  \langle 2345\rangle
   }{\langle 1234\rangle  \langle 2358\rangle
   }\to-\frac{\langle 1235\rangle  \langle 4568\rangle
   }{\langle 5(18)(23)(46)\rangle }\to\frac{\langle
   1568\rangle  \langle 2358\rangle  \langle 3456\rangle
   }{\langle 1358\rangle  \langle 2356\rangle  \langle
   4568\rangle }\to-\frac{\langle 5(18)(23)(46)\rangle
   }{\langle 1258\rangle  \langle 3456\rangle
   }\to\frac{\langle 1278\rangle  \langle 1358\rangle
   }{\langle 1238\rangle  \langle 1578\rangle }\\
   &\frac{\langle 1234\rangle  \langle 3456\rangle
   }{\langle 1346\rangle  \langle 2345\rangle
   }\to\frac{\langle 1348\rangle  \langle 2346\rangle
   }{\langle 1234\rangle  \langle 3468\rangle
   }\to-\frac{\langle 1346\rangle  \langle 5678\rangle
   }{\langle 6(18)(34)(57)\rangle }\to-\frac{\langle
   1678\rangle  \langle 3468\rangle  \langle 34(128)\cap
   (567)\rangle }{\langle 1268\rangle  \langle
   1348\rangle  \langle 3467\rangle  \langle 5678\rangle
   }\to\frac{\langle 1278\rangle  \langle
   6(18)(34)(57)\rangle }{\langle 1678\rangle  \langle
   34(128)\cap (567)\rangle }\\
   &\frac{\langle 1234\rangle  \langle 1278\rangle
   }{\langle 1238\rangle  \langle 1247\rangle
   }\to-\frac{\langle 1248\rangle  \langle 3457\rangle
   }{\langle 4(12)(35)(78)\rangle }\to-\frac{\langle
   1247\rangle  \langle 12(345)\cap (678)\rangle
   }{\langle 1278\rangle  \langle 4(12)(35)(67)\rangle
   }\to-\frac{\langle 4567\rangle  \langle
   4(12)(35)(78)\rangle }{\langle 1245\rangle  \langle
   3457\rangle  \langle 4678\rangle }\to-\frac{\langle
   4(12)(35)(67)\rangle }{\langle 1234\rangle  \langle
   4567\rangle }
\end{align}
The first $A_5$ lives in an 8-cycle of the $\Gr(4,8)$ dihedral+parity, while the other three live in 16-cycles. Also note that in the first $A_5$, 7 and 8 never appear together in a $\langle\rangle$, and so the $8\to7$ collinear limit is smooth for this $A_5$. The second $A_5$ also features a smooth collinear limit, as 
\begin{equation}
	\frac{\langle 1278\rangle  \langle 1358\rangle
   }{\langle 1238\rangle  \langle 1578\rangle } \xrightarrow{8\to7} \frac{\langle 1267\rangle  \langle 1357\rangle
   }{\langle 1237\rangle  \langle 1567\rangle }.
\end{equation}
Neither of the latter 2 $A_5$s behave smoothly in the collinear limit (and neither do any of their dihedral+parity images).

Note: there are no good $A_6$s in $\Gr(4,8)$.  

\subsection{Representing \texorpdfstring{$R_8^{(2)}$}{R28}}
The $A_5$ contribution to $R^{(2)}_8$ involves simply adding together the two $A_5$s in $\Gr(4,8)$ which behave smoothly in the collinear limit. 
\begin{equation}\label{eq:r28A5}
\begin{split}
	&R^{(2)}_8 = \frac14 f_{A_5}\left(\frac{\langle 1238\rangle  \langle 1256\rangle }{\langle
   1235\rangle  \langle 1268\rangle }\to \frac{\langle
   1236\rangle  \langle 2345\rangle }{\langle 1234\rangle
    \langle 2356\rangle }\to \frac{\langle 1235\rangle 
   \langle 3456\rangle }{\langle 1356\rangle  \langle
   2345\rangle }\to \frac{\langle 1567\rangle  \langle
   2356\rangle }{\langle 1256\rangle  \langle 3567\rangle
   }\to \frac{\langle 1356\rangle  \langle 4567\rangle
   }{\langle 1567\rangle  \langle 3456\rangle }\right)+\\
   &\frac12 f_{A_5}\left(\frac{\langle 1238\rangle  \langle 2345\rangle
   }{\langle 1234\rangle  \langle 2358\rangle
   }\to-\frac{\langle 1235\rangle  \langle 4568\rangle
   }{\langle 5(18)(23)(46)\rangle }\to\frac{\langle
   1568\rangle  \langle 2358\rangle  \langle 3456\rangle
   }{\langle 1358\rangle  \langle 2356\rangle  \langle
   4568\rangle }\to-\frac{\langle 5(18)(23)(46)\rangle
   }{\langle 1258\rangle  \langle 3456\rangle
   }\to\frac{\langle 1278\rangle  \langle 1358\rangle
   }{\langle 1238\rangle  \langle 1578\rangle }\right)\\
   &+\text{ dihedral} + \text{conjugate}
\end{split}
\end{equation}
Again the difference between the overall factors of the two terms is simply a result of symmetry. 

Let me briefly describe the collinear limit for this representation. As discussed previously, the $A_5$s explicitly written in (\ref{eq:r28A5}) behave smoothly under the collinear limit, however not all of their dihedral+parity images do as well. In the case of the first $A_5$, which has 8 images under dihedral+parity, 4 of the $f_{A_5}$s vanish, while the remaining 3 are well-defined. For the second $A_5$, which has 16 images under dihedral+parity, 2 of the $f_{A_5}$s have ``bad'' collinear limits but they cancel off each other in the sum. Out of the remaining 14, 4 have good collinear limits and 10 vanish identically. Therefore, when we add up the contributions from both $A_5$s + their images, we end up with 7 terms -- these correspond to the 7 $A_5$s in $\Gr(4,7)$. 


\begin{itemize}	
	\item $\Li_4$ contribution
	\item $\Li_2\Li_2$ contribution
	\item $\Li_3\Li_1$ contribution
	\item $\Li_2\Li_1^2$ contribution
	\item $\Li_1^4$ contribution
	\item $\Li_2\pi^2$ contribution
	\item $\Li_1^2\pi^2$ contribution
	\item $\pi^4$ constribution
\end{itemize}	

\subsection{Analytic Properties of \texorpdfstring{$R_8^{(2)}$}{R28}}
\begin{itemize}
	\item some plots
	\item agrees with numerics
\end{itemize}

\section{The Steinmann Relations for Eight Particles}

When the number of gluons $n$ is not a multiple of 4, the BDS-like ansatz is unique because there exists only a single decomposition of the BDS ansatz
\begin{equation}
{\cal A}_n^{\text{BDS}} = {\cal A}_n^{\text{BDS-like}}(\{s_{i,i+1}\}) \exp \left[ \frac{\Gamma_{\text{cusp}}}{4} Y_{n}(\{u_i\})  \right], \quad n\neq4K,
\end{equation}
such that the kinematic dependence of $A^{\text{BDS-like}}_{n}$ involves only two-particle Mandelstam invariants while $Y_{n}$ is a weight-two function depending only on dual conformal invariant cross ratios~\cite{Yang:2010az}. %In particular, at one loop this relation becomes
%\begin{equation}
%A^{\text{BDS},(1)}_{n} = A^{\text{BDS-like},(1)}_{n}(\{s_{i,i+1}\}) + Y_{n}(\{u_i\}), \quad n\neq4K,
%\end{equation}
When $n$ is a multiple of 4, no decomposition of this type exists, and we are forced to consider multiple BDS-like ans\"atze if we want to transparently expose all Steinmann relations between higher-particle Mandelstam invariants. 

In eight particle kinematics, there are two natural BDS-like normalization choices we might consider. Namely, we can let our BDS-like ansatz depend on either three- or four-particle Mandelstam invariants in addition to two-particle invariants. In this spirit, let us define a pair of Bose-symmetric BDS-like ans\"atze, respectively satisfying
\begin{align}
{\cal A}_8^{\text{BDS}} &= {}^3 {\cal A}_8^{\text{BDS-like}}(\{s_{i,i+1}\}, \{s_{i,i+1,i+2,i+3}\}) \exp \left[ \frac{\Gamma_{\text{cusp}}}{4}\ {}^3 Y_{8}(\{u_i\})  \right], \label{bds_like_3} \\
%{\cal A}^{\text{BDS},(1)}_{n} &= {}^3 {\cal A}^{\text{BDS-like},(1)}_{8}(\{s_{i,i+1}\}, \{s_{i,i+1,i+2,i+3}\}) + {}^3 Y_{8}(\{u_i\}), \\
{\cal A}_8^{\text{BDS}} &= {}^4 {\cal A}_8^{\text{BDS-like}}(\{s_{i,i+1}\}, \{s_{i,i+1,i+2}\}) \exp \left[ \frac{\Gamma_{\text{cusp}}}{4}\ {}^4 Y_{8}(\{u_i\})  \right]. \label{bds_like_4}
%{\cal A}^{\text{BDS},(1)}_{n} &= {}^4 {\cal A}^{\text{BDS-like},(1)}_{8}(\{s_{i,i+1}\}, \{s_{i,i+1,i+2}\}) + {}^4 Y_{8}(\{u_i\}). 
\end{align}
In fact, the functions ${}^3 A^{\text{BDS-like}}_{8} $ and ${}^4 A^{\text{BDS-like}}_{8}$ are not uniquely fixed by these decomposition choices; each admits a family of Bose-symmetric solutions. 
 
Any choice for ${}^3 A^{\text{BDS-like}}_{8}$ or ${}^4 A^{\text{BDS-like}}_{8}$ that satisfy eqns.~\eqref{bds_like_3} or \eqref{bds_like_4} gives rise to a BDS-like normalized amplitude that manifestly exhibits a subset of the Steinmann relations. In particular, defining
\begin{equation}
{}^X {\cal E}_8 \equiv \frac{{\cal A}_8^{\text{MHV}}}{{}^X {\cal A}^{\text{BDS-like}}_{8}} = \exp\left[ R_8 - \frac{\Gamma_{\text{cusp}}}{4} \  {}^X Y_8 \right] \label{BDS_like_amplitude}
\end{equation}
for any label $X$, we expect that ${}^3 {\cal E}_8$ should satisfy Steinmann relations between all partially overlapping pairs of three-particle invariants, while ${}^4 {\cal E}_8$ should satisfy Steinmann relations between all partially overlapping pairs of four-particle invariants. That is, ${}^3 {\cal E}_8$ is expected to satisfy the relations
\begin{align}
\text{Disc}_{s_{i+1,i+2,i+3}}\left[\text{Disc}_{s_{i,i+1,i+2}} ({}^3 {\cal E}_8) \right] &= 0, \label{stein33_1} \\
\text{Disc}_{s_{i+2,i+3,i+4}}\left[\text{Disc}_{s_{i,i+1,i+2}} ({}^3 {\cal E}_8) \right] &= 0, \label{stein33_2}
\end{align}
for all $i$, while ${}^4 {\cal E}_8$ is expected to satisfy
\begin{align}
\text{Disc}_{s_{i+1,i+2,i+3,i+4}}\left[\text{Disc}_{s_{i,i+1,i+2,i+3}} ({}^4 {\cal E}_8) \right] &= 0, \label{stein44_1} \\
\text{Disc}_{s_{i+2,i+3,i+4,i+5}}\left[\text{Disc}_{s_{i,i+1,i+2,i+3}} ({}^4 {\cal E}_8) \right] &= 0, \label{stein44_2} \\
\text{Disc}_{s_{i+3,i+4,i+5,i+6}}\left[\text{Disc}_{s_{i,i+1,i+2,i+3}} ({}^4 {\cal E}_8) \right] &= 0. \label{stein44_3}
\end{align}
However, conditions~\eqref{stein33_1} through~\eqref{stein44_3} don't exhaust the set of Steinmann relations obeyed by generic eight-particle amplitudes---there are also Steinmann relations between partially overlapping three- and four-particle invariants. If we want to make these additional relations manifest, we can use the fact that it proves possible to define a BDS-like ansatz that depends on all but one of the four-particle invariants (and on no three-particle invariants). That is, we decompose the BDS ansatz as
\begin{equation}
{\cal A}_8^{\text{BDS}} = {}^{3,j} {\cal A}_8^{\text{BDS-like}}(\{s_{i,i+1}\}, \{s_{i,i+1,i+2,i+3} \neq s_{j,j+1,j+2,j+3} \}) \exp \left[ \frac{\Gamma_{\text{cusp}}}{4}\ {}^{3,j} Y_{8}(\{u_i\})  \right], \label{bds_like_3q}
%{\cal A}^{\text{BDS}}_{n} = {}^{3,j} {\cal A}^{\text{BDS-like}}_{8}(\{s_{i,i+1}\}, \{s_{i,i+1,i+2,i+3} \neq s_{j,j+1,j+2,j+3} \}) + {}^{3,j} Y_{8}(\{u_i\}), 
\end{equation}
and in so doing define a BDS-like normalized amplitude that satisfies 
\begin{align}
\text{Disc}_{s_{j+2,j+3,j+4}}\left[\text{Disc}_{s_{j,j+1,j+2,j+3}} ({}^{3,j} {\cal E}_8) \right] &= 0, \label{stein34_1} \\
\text{Disc}_{s_{j+3,j+4,j+5}}\left[\text{Disc}_{s_{j,j+1,j+2,j+3}} ({}^{3,j} {\cal E}_8) \right] &= 0, \label{stein34_2} \\
\text{Disc}_{s_{j-1,j,j+1}}\left[\text{Disc}_{s_{j,j+1,j+2,j+3}} ({}^{3,j} {\cal E}_8) \right] &= 0, \label{stein34_3} \\
\text{Disc}_{s_{j-2,j-1,j}}\left[\text{Disc}_{s_{j,j+1,j+2,j+3}} ({}^{3,j} {\cal E}_8) \right] &= 0, \label{stein34_4} \\
\text{Disc}_{s_{i+1,i+2,i+3}}\left[\text{Disc}_{s_{i,i+1,i+2}} ({}^{3,j} {\cal E}_8) \right] &= 0, \label{stein34_5} \\
\text{Disc}_{s_{i+2,i+3,i+4}}\left[\text{Disc}_{s_{i,i+1,i+2}} ({}^{3,j} {\cal E}_8) \right] &= 0, \label{stein34_6}
\end{align}
where the relations~\eqref{stein34_5} and~\eqref{stein34_6} are the Steinmann relations satisfied by ${}^3 {\cal E}_8$. Clearly it is not possible to find a solution to eq.~\eqref{bds_like_3q} that is Bose-symmetric, but we can require that it is invariant under the dihedral flip that maps $s_{i \dots l} \rightarrow s_{9-i \dots 9 - l}$. There is again a family of solutions that satisfy these constraints. 

Momentum conservation implies that ${}^{3,j+4} {\cal A}^{\text{BDS-like}}_{8} = {}^{3,j} {\cal A}^{\text{BDS-like}}_{8}$, and thus every eight-point Steinmann relation is made manifest in at least one of the amplitudes 
$$\{{}^4 {\cal E}_8, {}^{3,1} {\cal E}_8 , {}^{3,2} {\cal E}_8 , {}^{3,3} {\cal E}_8 , {}^{3,4} {\cal E}_8 \}.$$ 
However, in practice it may prove convenient to retain ${}^{3} {\cal E}_8$ in one's toolbox, since it can be chosen to respect Bose symmetry. It may additionally prove possible to chose these amplitudes to satisfy other desirable properties, since ${}^{3} Y_{8}$, ${}^{4} Y_{8}$, and ${}^{3,j} Y_{8}$ are not uniquely picked out by eqns.~\eqref{bds_like_3}, \eqref{bds_like_4}, and \eqref{bds_like_3q}. In fact there is a 10-dimensional (3-dimensional) space of (Bose-symmetric) solutions for ${}^{3} Y_{8}$, a 36-dimensional (4-dimensional) space of (Bose-symmetric) solutions for ${}^{4} Y_{8}$, and a 5-dimensional (3-dimensional) space of (dihedral flip-invariant) solutions for ${}^{3,j} Y_{8}$.

To take full advantage of the Steinmann relations, it is convenient to work in terms of symbol letters that isolate different Mandelstam invariants. There are twelve independent dual conformally invariant cross ratios that can appear in these symbols
\begin{align}
u_1 &= \frac{s_{12} s_{4567}}{s_{123} s_{812}}, \quad \text{and cyclic (8-orbit)} \\
u_9 &= \frac{s_{123} s_{567}}{s_{1234} s_{4567}}, \quad \text{and cyclic (4-orbit).}
\end{align}
It is not possible to isolate all three- and four-particle Mandelstam invariants simultaneously into twelve different symbol letters. (More than twelve symbol letters will appear in these amplitudes, but we here restrict our attention to the twelve that will appear in the first entry.) However, different choices of letters can be made such that either all the four-particle invariants, or all the three-particle invariants, are isolated.

One choice that isolates the four-particle invariants is
\begin{align}
{}^4 d_1 &= u_2 \ u_6 = \frac{s_{23} \ s_{67} \ (s_{1234})^2}{s_{123} \ s_{234} \ s_{567} \ s_{678}}, \quad \text{and cyclic (4-orbit)} \\
{}^4 d_5 &= u_2/u_6 = \frac{s_{23} \ s_{567} \ s_{678}}{s_{67} \ s_{123} \ s_{234}}, \quad \text{and cyclic (4-orbit)} \\
{}^4 d_9 &= u_1 \ u_2 \ u_5 \ u_6 \ u_9^2 = \frac{s_{12} \ s_{23} \ s_{56} \ s_{67}}{s_{234} \ s_{456} \ s_{678} \ s_{812}}, \quad \text{and cyclic (4-orbit)}.
\end{align}
In this alphabet ${}^4 d_1, {}^4 d_2, {}^4 d_3$, and ${}^4 d_4$ each contain a different four-particle Mandelstam invariant, while the other letters only involve two- and three-particle invariants. The extended Steinmann relations then tell us that ${}^4 d_1, {}^4 d_2, {}^4 d_3$, and ${}^4 d_4$ can never appear next to each other in the symbol of ${}^4 A^{\text{BDS-like}}_{8}$ (but each can still appear next to themselves).

Similarly, we can isolate the three-particle invariants by choosing
\begin{align}
{}^3 d_1 &= \frac{u_1 \ u_2 \ u_4 \ u_7}{u_3 \ u_5 \ u_6 \ u_8 \ u_9^2} = \frac{s_{12} \ s_{23} \ s_{45} \ s_{78} \ (s_{1234})^2 \ (s_{4567})^2}{s_{34} \ s_{56} \ s_{67} \ s_{81} \ (s_{123})^2}, \quad \text{and cyclic (8-orbit)} \\
{}^3 d^4_9 &= u_1 \ u_5 \ u_9 \ u_{12} = \frac{s_{12} \ s_{56}}{s_{1234} \ s_{3456}}, \quad \text{and cyclic (4-orbit)},
\end{align}
in which case ${}^3 d_1$ through ${}^3 d_8$ each contain a different three-particle Mandelstam invariant, as well as four-particle Mandelstams that they don't partially overlap with. The remaining four letters only contain two- and four-particle invariants. In these letters, conditions~\eqref{stein34_5} and~\eqref{stein34_6} tell us that ${}^3 d_7, {}^3 d_8, {}^3 d_2$, and ${}^3 d_3$ can never appear next to ${}^3 d_1$ in the symbols of ${}^{3} {\cal E}_8$ or ${}^{3,j} {\cal E}_8$ (plus the cyclic images of this statement). Moreover, conditions~\eqref{stein34_1} through~\eqref{stein34_4} give us the additional restrictions that none of ${}^3 d_1, {}^3 d_5, {}^3 d_9$ and ${}^3 d_{10}$ can ever appear next to ${}^3 d_3, {}^3 d_4, {}^3 d_7,$ or ${}^3 d_8$ in the symbol of ${}^{3,1} {\cal E}_8$ (analogous relations hold for the other ${}^{3,j} {\cal E}_8$). These are the restrictions given by the Steinmann relations involving $s_{1234}$ and one of $s_{781}, s_{812}, s_{345}$, or $s_{456}$. The other Steinmann relations between three- and four-particle invariants will not be respected by ${}^{3,1} {\cal E}_8$, since ${}^{3,j} {\cal A}^{\text{BDS-like}}_{8}$ depends on $s_{2345}, s_{3456},$ and $s_{4567}$.

\section{Conclusion}

\appendix
\section{\texorpdfstring{$A_5$}{A5} Representation of \texorpdfstring{$R^{(2)}_7$}{R27}}
 
\bibliographystyle{ieeetr}

\bibliography{eight_particle_two_loop}

\end{fmffile}
\end{document}
