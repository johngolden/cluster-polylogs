\pdfoutput=1
\documentclass[11pt, reqno,preprint]{article}
\usepackage{jheppub}
\usepackage{epsfig}
\usepackage{amssymb}
\usepackage{amsmath}
\usepackage{tikz}
\usepackage{mathrsfs}
%\usepackage{cite}
\usepackage{hyperref}
\usepackage{multirow}
\usepackage{feynmp-auto}
\usepackage{scalerel}
\usepackage{mathtools}



\def\ket#1{\langle #1 \rangle}

\DeclareMathOperator{\B}{B}
\DeclareMathOperator{\Conf}{Conf}
\DeclareMathOperator{\Gr}{Gr}
\DeclareMathOperator{\Li}{Li}
\DeclareMathOperator{\sgn}{sgn}

\def\be{\begin{equation}}
\def\ee{\end{equation}}
\def\ba{\begin{eqnarray}}
\def\ea{\end{eqnarray}}
\newcommand{\bea}{\begin{eqnarray}}
\newcommand{\eea}{\end{eqnarray}}
\def\nl{\nonumber\\}
\def\l{\langle}
\def\r{\rangle}
\def\eps{\epsilon}
\def\uo{\underline{0}}
\def\uone{\underline{1}}
\def\ut{\underline{t}}
\def\z{x}
\def\zb{y}
\newcommand{\del}{\partial}
\def\fig#1{fig.~{\ref{#1}}}
\def\Fig#1{Fig.~{\ref{#1}}}
\def\figs#1#2{figs.~{\ref{#1}} and {\ref{#2}}}
\def\Figs#1#2{Figs.~{\ref{#1}} and {\ref{#2}}}
\def\Sect#1{Section~{\ref{#1}}}
\def\sect#1{section~{\ref{#1}}}
\def\eqn#1{eq.~(\ref{#1})}
\def\Eqn#1{Equation~(\ref{#1})}
\def\eqns#1#2{eqs.~(\ref{#1}) and~(\ref{#2})}
\def\Eqns#1#2{Eqs.~(\ref{#1}) and~(\ref{#2})}
\def\tab#1{table~{\ref{#1}}}
\def\Tab#1{Table~{\ref{#1}}}
\def\tabs#1#2{table~{\ref{#1}} and~{\ref{#1}}}
\def\Tabs#1#2{Table~{\ref{#1}} and~{\ref{#1}}}
\def\Eqn#1{Equation~(\ref{#1})}
\def\eqn#1{eq.~(\ref{#1})}
\def\eqns#1#2{eqs.~(\ref{#1}) and~(\ref{#2})}
\def\Eqns#1#2{Eqs.~(\ref{#1}) and~(\ref{#2})}
\def\drawOctagon{
\coordinate (P1) at (45:1);
\coordinate (P2) at (90:1);
\coordinate (P3) at (135:1);
\coordinate (P4) at (180:1);
\coordinate (P5) at (225:1);
\coordinate (P6) at (270:1);
\coordinate (P7) at (315:1);
\coordinate (P8) at (359:1);
\draw (P1) -- (P2) -- (P3) -- (P4) -- (P5) -- (P6) -- (P7) -- (P8) -- cycle;
}


\def\mand#1{\scaleto{s}{4.6pt}_{\scaleto{#1}{5.2pt}}}
\def\EthreeJ{{}^{{\{a,b\}}_3} {\cal E}_8}
\def\EfourJ{{}^{{\{a,b\}}_4} {\cal E}_8}

\newcommand{\cP}{{\cal P}}
\def\lr{\leftrightarrow}

\def\draftnote#1{{\bf [#1]}}


\def\blue#1{{\color{blue}#1}}

\title{Cluster Polylogarithms, Adjacency, and Subalgebra-Constructibility at Eight Points} 

\author{John~Golden$^{1,2}$}
\author{and Andrew~J.~McLeod$^{2,3,4}$}


\affiliation{$^1$ Leinweber  Center for Theoretical Physics and
Randall Laboratory of Physics, Department of Physics,
University of Michigan
Ann Arbor, MI 48109, USA}

\affiliation{$^2$ Kavli Institute for Theoretical Physics, 
UC Santa Barbara, Santa Barbara, CA 93106, USA}

\affiliation{$^3$ SLAC National Accelerator Laboratory,
Stanford University, Stanford, CA 94309, USA}

\affiliation{$^4$ Niels Bohr International Academy, Blegdamsvej 17, 2100 Copenhagen, Denmark}

\abstract{We construct a cluster-polylogarithimc representation of the eight-point two-loop MHV amplitude in the planar limit of maximally supersymmetric Yang-Mills theory. This representation makes manifest a novel cluster-algebraic decomposition of the nonclassical part of this amplitudes into its $A_5$ subalgebras, and limits smoothly to a similar decomposition of the seven-point MHV amplitude in collinear limits. We also investigate the equivalence of the extended Steinmann relations and cluster adjacency in eight-point kinematics by exploring the space of BDS-like normalized amplitudes.}

%\preprint{
%\begin{flushright} DESY ??--??? \\ SLAC--PUB--?????
%\end{flushright}
%}


\begin{document}
\hypersetup{pageanchor=false}
\maketitle
\hypersetup{pageanchor=true}
\begin{fmffile}{feyndiags}


\section{Introduction}

\begin{enumerate}
\item[-] emphasize the fact that promoting symbols to functions is hard---only a few other instances in the literature (don't forget  this is done Regge limits as well---cite)
\item[-] must talk about the importance of automorphisms---let's become the standard physics reference on this!
\item[-] same for the Sklyanin bracket
\item[-] also, discuss the relation between cluster {$\cal A$}-adjacency and cluster {$\cal X$}-adjacency --- can we prove these have to be equivalent by using the conversion $x\sim a^b$ between the two (since this translation is valid on any cluster)?
\item[-] mention existence of $D_5$ function and refer ahead
\item[-] we should also check our function against MRK predictions if possible (but don't want to hold up paper for this... clearly we can publish without)
\item[-] should point out somewhere that the cobracket is the same for the remainder function and bds-like normalized amplitudes---and that the same bootstrap procedure could be carried out for either quantity. However, we carry it out on the remainder function because there's no clear (unique) bds-like normalized amplitude to bootstrap
\end{enumerate}

\section{Cluster Algebras and Cluster Polylogarithms}

\subsection{The A2 function}
We define the $A_2$ function as
\begin{equation}
   f_{A_2}=\sum_{\text{skew-dihedral}} \Li_{2,2}(-x_i,-x_{i+2})-\Li_{1,3}(-x_i,-x_{i+2})-\Li_2(-x_i)\log(x_{i+1})\log(x_{i+2}).
\end{equation}
The ``skew-dihedral'' sum of an arbitrary $f(i)$ is just
\begin{equation}
	\sum_{i=1}^5 f(i)-f(6-i)
\end{equation}
where $6-i$ is understood to be $\mod 5$
It has the symbol
\begin{align}
   -\sum_{\text{skew-dihedral}}
   &x_i \otimes x_i \otimes x_{i+1} \otimes x_{i+1}
   +x_i \otimes x_{i+1} \otimes x_i \otimes x_{i-1}
   +x_i \otimes x_{i+1} \otimes x_{i+1} \otimes x_i\\
   &\quad-2\left(x_i \otimes x_{i+1} \otimes (x_ix_{i+2}) \otimes x_{i+1}
   -x_i \otimes x_{i+1} \otimes x_{i+1} \otimes x_{i+2}\right)\nl
\end{align}
And satisfies the properties:
\begin{itemize}
   \item clustery cobracket
   \item cluster adjacent in $A_2$
   \item smooth and real-valued in the positive domain
\end{itemize}
Some cluster algebra rules:
\begin{equation}
  \label{eq:b-mutation}
  b'_{i j} =
  \begin{cases}
    -b_{i j}, &\quad \text{if $k \in \lbrace i, j\rbrace$,}\\
    b_{i j}, &\quad \text{if $b_{i k} b_{k j} \leq 0$,}\\
    b_{i j} + b_{i k} b_{k j}, &\quad \text{if $b_{i k}, b_{k j} > 0$,}\\
    b_{i j} - b_{i k} b_{k j}, &\quad \text{if $b_{i k}, b_{k j} < 0$.}
  \end{cases}
\end{equation}

\begin{equation}
  \label{eq:x-coords-mutation}
  x_{i}' =
  \begin{cases}
    x_{k}^{-1}, &\quad i=k,\\
    x_{i} (1+x_{k}^{\sgn b_{i k}})^{b_{i k}}, &\quad i \neq k
  \end{cases}.
\end{equation}

\begin{equation}
  \label{eq:poisson-x-coords}
  \lbrace x_{i}, x_{j}\rbrace = b_{i j} x_{i} x_{j}.
\end{equation}

\begin{equation}
  \lbrace x_{i}', x_{j}'\rbrace = b_{i j}' x_{i}' x_{j}'
\end{equation}

In momentum twistor language we have the $n$ momentum twistors $Z_i$, which together form the $4 \times n$ matrix

\begin{equation}  
K = \left(\begin{array}{ccc}
z_{11} & \ldots & z_{n1} \\
z_{12} & \ldots & z_{n2} \\
z_{13} & \ldots & z_{n3} \\
z_{14} & \ldots & z_{n4}\end{array}\right).
\end{equation}
As long as the first 4 columns are non-singular, we can row reduce $K$ in to the form
\begin{equation}
K'=\left(
\begin{array}{ccccccc}
 1 & 0 & 0 & 0 & y_{11} & \ldots  & y_{(n-4)1} \\
 0 & 1 & 0 & 0 & y_{12} & \ldots  & y_{(n-4)2} \\
 0 & 0 & 1 & 0 & y_{13} & \ldots  & y_{(n-4)3} \\
 0 & 0 & 0 & 1 & y_{14} & \ldots  & y_{(n-4)4} \\
\end{array}
\right).
\end{equation}
The columns of $K'$ define a new set of momentum twistors $Z'_i$, where for example $Z'_1 = \{1,0,0,0\}$ and $Z'_5 = \{y_{11},y_{12},y_{13},y_{14}\}$. It is easy to check that 
\begin{align}
   &y_{ij} = (-1)^j \ket{\{1,2,3,4\}\setminus\{j\},i}/\ket{1234},\\
   &\ket{abcd}' = \det(Z'_a Z'_b Z'_c Z'_d) = \ket{abcd}/\ket{1234}.
\end{align}
You can then define the Sklyanin bracket as an operation on these $y_{ij}$ by
\begin{equation}
   \{y_{ij},y_{ab}\} = (\sgn(a-i) - \sgn(b-j)) y_{ib} y_{aj}.
\end{equation}
Which then extends to a bracket on functions of the $y_{ij}$ via
\begin{equation}\label{eq:def}
   \{f(y), g(y)\} =  \sum_{i,a=1}^n\sum_{j,b=1}^4\frac{\partial f}{\partial y_{ij}}  \frac{\partial g}{\partial y_{ab}} 
\{y_{ij}, y_{ab}\}.
\end{equation}
Now if we want to evaluate the Poisson bracket between two $\mathcal{X}$-coordinates, we can instead treat them as functions of the $y_{ij}$ and use eq.~(\ref{eq:def}). To be precise, for each four-bracket $\ket{abcd}$ in the $\mathcal{X}$-coordinates, replace them with $\ket{abcd}'$ expanded out in terms of $y_{ij}$ (e.g. $\ket{1256}' =y_{13} y_{24}-y_{14} y_{23}$). Then you can calculate eq.~(\ref{eq:def}) directly in terms of the $y_{ij}$

\section{Promoting \texorpdfstring{$R_8^{(2)}$}{R28} from Symbol to Function}

\subsection{The \texorpdfstring{$A_5$}{A5} Function}
\begin{itemize}
	\item describe the $A_5$ cluster algebra
	\item what makes ``the'' $A_5$ function unique?
\end{itemize}

The $A_5$ cluster algebra is generated from the seed cluster
\begin{equation}
	x_1 \to x_2 \to x_3 \to x_4 \to x_5.
\end{equation}
The full $A_5$ algebra contains 132 clusters with 140 distinct $\mathcal{X}$-coordinates. 

Define:
\begin{equation}
	x_{i_1\ldots i_k} = \sum_{a=1}^k \prod_{b=1}^a x_{i_b} = x_{i_1}+x_{i_1}x_{i_2} + \ldots + x_{i_1}\cdots x_{i_k}.
\end{equation}

The $A_5$ cluster algebra has an eight-fold cyclic symmetry, which is generated by $\sigma$:
\begin{equation}
\begin{split}
	\sigma:\quad& 
		x_1\mapsto\frac{x_2}{1+x_{12}},~~	
		x_2\mapsto\frac{x_3\left(1+x_1\right)}{1+x_{123}},~~
		x_3\mapsto\frac{x_4 \left(1+x_{12}\right)}{1+x_{1234}},\\&
		x_4\mapsto\frac{x_5 \left(1+x_{123}\right)}{1+x_{12345}},~~
		x_5\mapsto\frac{1+x_{1234}}{x_1 x_2 x_3 x_4 x_5}.
\end{split}
\end{equation}
$A_5$ also has a two-fold flip symmetry, which is generated by $\tau$:
\begin{equation}
	\tau:\quad x_i\mapsto\frac{1}{x_{6-i}}.
\end{equation}
There are 56 distinct $A_2$ subalgebras in $A_5$ (56 = $\genfrac(){0pt}{1}{8}{5}$ = number of distinct pentagons inside an octagon), they can be parameterized by:
\begin{equation}
\begin{split}
	&\left\{x_1\to x_2,~~
	x_2\to x_3\left(1+x_4\right),~~
	x_2\left(1+x_3\right)\to \frac{x_3 x_4}{1+x_3}\right\} + \sigma,\\
	&\left\{x_2\to x_3,~~x_1 \left(1+x_2\right)\to \frac{x_2x_3}{1+x_2}\right\} + \sigma + \tau 
   \end{split}
\end{equation}
where by ``$+~\sigma$'' and ``$+~\sigma~+~\tau$'' I mean ``+ cyclic copies'' and ``+ cyclic and flip copies,'' respectively. This correspond to the geometries
\begin{center}
\begin{tikzpicture}
\drawOctagon
\draw[color=red] (P1) -- (P2) -- (P3) -- (P4) -- (P5) -- cycle;
\begin{scope}[xshift=3cm]
\drawOctagon
\draw[color=red] (P1) -- (P2) -- (P3) -- (P5) -- (P7) -- cycle;
\end{scope}
\begin{scope}[xshift=6cm]
\drawOctagon
\draw[color=red] (P1) -- (P2) -- (P4) -- (P5) -- (P7) -- cycle;
\end{scope}
\begin{scope}[yshift=-2.5cm,xshift=1.5cm]
\drawOctagon
\draw[color=red] (P1) -- (P2) -- (P3) -- (P4) -- (P6) -- cycle;
\end{scope}
\begin{scope}[yshift=-2.5cm, xshift=4.5cm]
\drawOctagon
\draw[color=red] (P1) -- (P2) -- (P3) -- (P5) -- (P6) -- cycle;
\end{scope}
\node[] at (9,0) (a) {(8 of each)};
\node[] at (9,-2.5) (a) {(16 of each)};
\end{tikzpicture}
\end{center}
\subsubsection*{The \texorpdfstring{$A_5$}{A5} function}
The $A_5$ function is a sum over two of the classes of $A_2$ subalgebras, $x_2\to x_3\left(1+x_4\right)$ and $x_1 \left(1+x_2\right)\to \frac{x_2x_3}{1+x_2}$, appropriately antisymmetrized so that the overall $f_{A_5}$ picks up a minus sign under both $\sigma$ and $\tau$. Explicitly, this is written
\begin{equation}
	f_{A_5} = \sum_{i=0}^7\sum_{j=0}^1(-1)^{i+j}\sigma^i\tau^j\left(\frac12 f_{A_2}\left(x_2\to x_3\left(1+x_4\right)\right) + f_{A_2}\left(x_1 \left(1+x_2\right)\to \frac{x_2x_3}{1+x_2}\right)\right).
\end{equation}
The factor of $\frac12$ in front of $f_{A_2}\left(x_2\to x_3\left(1+x_4\right)\right)$ is simply a symmetry factor, as it lives in an 8-cycle of $\{\sigma,\tau\}$.

Which pentagonalizations do these two $A_2$s correspond to?

A large outstanding question is: why these two $A_2$s? I have no justification/argument for them other than they work. 

The $A_5$ does not have $B_2\wedge B_2$ expressible in terms of $A_1 \times A_1$ subalgebras (i.e. it is not expressible as a sum of $f_{A_3}$s). This is very surprising!

\subsection{Identifying \texorpdfstring{$A_5$}{A5} subalgebras in \texorpdfstring{$\Gr(4,8)$}{Gr(4,8)}}
Comment on infinite nature of $\Gr(4,8)$, define ``good'' (in the two-loop MHV sense) subalgebras, describe algorithm for finding ``good'', and describe the 56 good $A_5$s in $\Gr(4,8)$.

There are 56 good $A_5$s in $\Gr(4,8)$. They are generated by
\begin{align}
	&\frac{\langle 1238\rangle  \langle 1256\rangle }{\langle
   1235\rangle  \langle 1268\rangle }\to \frac{\langle
   1236\rangle  \langle 2345\rangle }{\langle 1234\rangle
    \langle 2356\rangle }\to \frac{\langle 1235\rangle 
   \langle 3456\rangle }{\langle 1356\rangle  \langle
   2345\rangle }\to \frac{\langle 1567\rangle  \langle
   2356\rangle }{\langle 1256\rangle  \langle 3567\rangle
   }\to \frac{\langle 1356\rangle  \langle 4567\rangle
   }{\langle 1567\rangle  \langle 3456\rangle }\\
   &\frac{\langle 1238\rangle  \langle 2345\rangle
   }{\langle 1234\rangle  \langle 2358\rangle
   }\to-\frac{\langle 1235\rangle  \langle 4568\rangle
   }{\langle 5(18)(23)(46)\rangle }\to\frac{\langle
   1568\rangle  \langle 2358\rangle  \langle 3456\rangle
   }{\langle 1358\rangle  \langle 2356\rangle  \langle
   4568\rangle }\to-\frac{\langle 5(18)(23)(46)\rangle
   }{\langle 1258\rangle  \langle 3456\rangle
   }\to\frac{\langle 1278\rangle  \langle 1358\rangle
   }{\langle 1238\rangle  \langle 1578\rangle }\\
   &\frac{\langle 1234\rangle  \langle 3456\rangle
   }{\langle 1346\rangle  \langle 2345\rangle
   }\to\frac{\langle 1348\rangle  \langle 2346\rangle
   }{\langle 1234\rangle  \langle 3468\rangle
   }\to-\frac{\langle 1346\rangle  \langle 5678\rangle
   }{\langle 6(18)(34)(57)\rangle }\to-\frac{\langle
   1678\rangle  \langle 3468\rangle  \langle 34(128)\cap
   (567)\rangle }{\langle 1268\rangle  \langle
   1348\rangle  \langle 3467\rangle  \langle 5678\rangle
   }\to\frac{\langle 1278\rangle  \langle
   6(18)(34)(57)\rangle }{\langle 1678\rangle  \langle
   34(128)\cap (567)\rangle }\\
   &\frac{\langle 1234\rangle  \langle 1278\rangle
   }{\langle 1238\rangle  \langle 1247\rangle
   }\to-\frac{\langle 1248\rangle  \langle 3457\rangle
   }{\langle 4(12)(35)(78)\rangle }\to-\frac{\langle
   1247\rangle  \langle 12(345)\cap (678)\rangle
   }{\langle 1278\rangle  \langle 4(12)(35)(67)\rangle
   }\to-\frac{\langle 4567\rangle  \langle
   4(12)(35)(78)\rangle }{\langle 1245\rangle  \langle
   3457\rangle  \langle 4678\rangle }\to-\frac{\langle
   4(12)(35)(67)\rangle }{\langle 1234\rangle  \langle
   4567\rangle }
\end{align}
The first $A_5$ lives in an 8-cycle of the $\Gr(4,8)$ dihedral+parity, while the other three live in 16-cycles. Also note that in the first $A_5$, 7 and 8 never appear together in a $\langle\rangle$, and so the $8\to7$ collinear limit is smooth for this $A_5$. The second $A_5$ also features a smooth collinear limit, as 
\begin{equation}
	\frac{\langle 1278\rangle  \langle 1358\rangle
   }{\langle 1238\rangle  \langle 1578\rangle } \xrightarrow{8\to7} \frac{\langle 1267\rangle  \langle 1357\rangle
   }{\langle 1237\rangle  \langle 1567\rangle }.
\end{equation}
Neither of the latter 2 $A_5$s behave smoothly in the collinear limit (and neither do any of their dihedral+parity images).

Note: there are no good $A_6$s in $\Gr(4,8)$.  

\subsection{Representing \texorpdfstring{$R_8^{(2)}$}{R28}}
The $A_5$ contribution to $R^{(2)}_8$ involves simply adding together the two $A_5$s in $\Gr(4,8)$ which behave smoothly in the collinear limit. 
\begin{equation}\label{eq:r28A5}
\begin{split}
	&R^{(2)}_8 = \frac14 f_{A_5}\left(\frac{\langle 1238\rangle  \langle 1256\rangle }{\langle
   1235\rangle  \langle 1268\rangle }\to \frac{\langle
   1236\rangle  \langle 2345\rangle }{\langle 1234\rangle
    \langle 2356\rangle }\to \frac{\langle 1235\rangle 
   \langle 3456\rangle }{\langle 1356\rangle  \langle
   2345\rangle }\to \frac{\langle 1567\rangle  \langle
   2356\rangle }{\langle 1256\rangle  \langle 3567\rangle
   }\to \frac{\langle 1356\rangle  \langle 4567\rangle
   }{\langle 1567\rangle  \langle 3456\rangle }\right)+\\
   &\frac12 f_{A_5}\left(\frac{\langle 1238\rangle  \langle 2345\rangle
   }{\langle 1234\rangle  \langle 2358\rangle
   }\to-\frac{\langle 1235\rangle  \langle 4568\rangle
   }{\langle 5(18)(23)(46)\rangle }\to\frac{\langle
   1568\rangle  \langle 2358\rangle  \langle 3456\rangle
   }{\langle 1358\rangle  \langle 2356\rangle  \langle
   4568\rangle }\to-\frac{\langle 5(18)(23)(46)\rangle
   }{\langle 1258\rangle  \langle 3456\rangle
   }\to\frac{\langle 1278\rangle  \langle 1358\rangle
   }{\langle 1238\rangle  \langle 1578\rangle }\right)\\
   &+\text{ dihedral} + \text{conjugate}
\end{split}
\end{equation}
Again the difference between the overall factors of the two terms is simply a result of symmetry. 

Let me briefly describe the collinear limit for this representation. As discussed previously, the $A_5$s explicitly written in (\ref{eq:r28A5}) behave smoothly under the collinear limit, however not all of their dihedral+parity images do as well. In the case of the first $A_5$, which has 8 images under dihedral+parity, 4 of the $f_{A_5}$s vanish, while the remaining 3 are well-defined. For the second $A_5$, which has 16 images under dihedral+parity, 2 of the $f_{A_5}$s have ``bad'' collinear limits but they cancel off each other in the sum. Out of the remaining 14, 4 have good collinear limits and 10 vanish identically. Therefore, when we add up the contributions from both $A_5$s + their images, we end up with 7 terms -- these correspond to the 7 $A_5$s in $\Gr(4,7)$. 


\begin{itemize}	
	\item $\Li_4$ contribution
	\item $\Li_2\Li_2$ contribution
	\item $\Li_3\Li_1$ contribution
	\item $\Li_2\Li_1^2$ contribution
	\item $\Li_1^4$ contribution
	\item $\Li_2\pi^2$ contribution
	\item $\Li_1^2\pi^2$ contribution
	\item $\pi^4$ constribution
\end{itemize}	

\subsection{Analytic Properties of \texorpdfstring{$R_8^{(2)}$}{R28}}
\begin{itemize}
	\item some plots
	\item agrees with numerics
\end{itemize}

\section{Steinmann Relations and Cluster Adjacency}

The Steinmann relations dictate that double discontinuities of amplitudes must vanish when taken in partially overlapping momentum channels~\cite{Steinmann,Cahill:1973qp}. It has recently been realized that these restrictions on three- (and higher-)particle channels are transparently encoded in the symbol of BDS-like normalized amplitudes when the number of scattering particles is not a multiple of four~\cite{Caron-Huot:2016owq, Dixon:2016nkn}. This follows from the fact that the BDS-like ansatz in these cases is defined to depend on just two-particle Mandelstam invariants, and thus acts as a spectator when discontinuities are taken in these channels. This subset of the Steinmann relations therefore applies directly to BDS-like-normalized amplitudes for these numbers of particles, where it implies that restricted pairs of Mandelstam invariants cannot appear sequentially in the first two entries of the symbol. In fact, these restrictions have been found to apply at all depths in the symbol, providing strong all-loop constraints on the spaces of functions that are expected to contribute to these amplitudes~\cite{omega_paper,cosmic_galois_paper}. 

More surprisingly, the extended Steinmann constraints have been found to be equivalent to demanding that every pair of sequential symbol entries appears together in some cluster in Gr(4,$n$)~\cite{Drummond:2017ssj}. In particular, it has been checked that this `cluster adjacency' principle is adhered to in all known BDS-like normalized amplitudes in six-, seven-, and nine-particle kinematics, where a unique BDS-like ansatz depending only on two-particle invariants can be defined. However, it remains less well-studied in eight-particle kinematics due to the nonexistence of any such BDS-like normalization; all eight-particle solutions to the anomalous dual conformal Ward identity governing these amplitudes in the infrared involve higher-particle Mandelstam invariants~\cite{Drummond:2007au}. For this reason, it proves necessary to explore the space of BDS-like ans\"atze that can be formed for eight particles before the (vestiges of the) Steinmann relations and cluster adjacency can be studied.

\subsection{BDS-Like Ans\"atze for Eight Particles}

\draftnote{Paragraph introducing the BDS ansatz}

When the number of particles $n$ is not a multiple of four, a unique BDS-like ansatz can be defined that depends on just two-particle Mandelstam invariants. That is, there exists just a single decomposition of the BDS ansatz
\begin{equation}
{\cal A}_n^{\text{BDS}}(\{\mand{i,\dots,i+j}\}) = {\cal A}_n^{\text{BDS-like}}(\{\mand{i,i+1}\}) \exp \left[ - \frac{\Gamma_{\text{cusp}}}{4} Y_{n}(\{u_i\})  \right], \quad n\neq4K,
\end{equation}
such that the kinematic dependence of $A^{\text{BDS-like}}_{n}$ involves only two-particle Mandelstam invariants while $Y_{n}$ depends only on dual-conformal-invariant cross ratios~\cite{Yang:2010az}. %In particular, at one loop this relation becomes
%\begin{equation}
%A^{\text{BDS},(1)}_{n} = A^{\text{BDS-like},(1)}_{n}(\{\mand{i,i+1}\}) + Y_{n}(\{u_i\}), \quad n\neq4K,
%\end{equation}
When $n$ is a multiple of four, no decomposition of this type exists, and we are forced to consider multiple BDS-like ans\"atze if we want to transparently expose the full space of Steinmann relations between higher-particle Mandelstam invariants. 

In eight-particle kinematics, there are still two natural BDS-like normalization choices we might consider. Namely, we can let our BDS-like ansatz depend on either three- or four-particle Mandelstam invariants in addition to two-particle invariants~\cite{Dixon:2016nkn}. In this spirit, let us define a pair of BDS-like ans\"atze, respectively satisfying
\begin{align}
{\cal A}_8^{\text{BDS}}(\{\mand{i,\dots,i+j}\}) &= {}^4 {\cal A}_8^{\text{BDS-like}}(\{\mand{i,i+1}\}, \{\mand{i,i+1,i+2,i+3}\}) \exp \left[ -\frac{\Gamma_{\text{cusp}}}{4}\ {}^4 Y_{8}(\{u_i\})  \right], \label{bds_like_4} \\
%{\cal A}^{\text{BDS},(1)}_{n} &= {}^3 {\cal A}^{\text{BDS-like},(1)}_{8}(\{\mand{i,i+1}\}, \{\mand{i,i+1,i+2,i+3}\}) + {}^3 Y_{8}(\{u_i\}), \\
{\cal A}_8^{\text{BDS}}(\{\mand{i,\dots,i+j}\}) &= {}^3 {\cal A}_8^{\text{BDS-like}}(\{\mand{i,i+1}\}, \{\mand{i,i+1,i+2}\}) \exp \left[ - \frac{\Gamma_{\text{cusp}}}{4}\ {}^3 Y_{8}(\{u_i\})  \right]. \label{bds_like_3}
%{\cal A}^{\text{BDS},(1)}_{n} &= {}^4 {\cal A}^{\text{BDS-like},(1)}_{8}(\{\mand{i,i+1}\}, \{\mand{i,i+1,i+2}\}) + {}^4 Y_{8}(\{u_i\}). 
\end{align}
The functions ${}^4 A^{\text{BDS-like}}_{8}$ and ${}^3 A^{\text{BDS-like}}_{8}$ are not uniquely fixed by these decomposition choices; each admits a family of Bose-symmetric (and a larger family of non-Bose-symmetric) solutions. However, any choice for ${}^4 A^{\text{BDS-like}}_{8}$ or ${}^3 A^{\text{BDS-like}}_{8}$ consistent with eqns.~\eqref{bds_like_4} or \eqref{bds_like_3} gives rise to a BDS-like normalized amplitude that manifestly exhibits a subset of the Steinmann relations. In particular, defining
\begin{equation}
{}^X {\cal E}_8 \equiv \frac{{\cal A}_8^{\text{MHV}}}{{}^X {\cal A}^{\text{BDS-like}}_{8}} = \exp\left[ R_8 - \frac{\Gamma_{\text{cusp}}}{4} \  {}^X Y_8 \right] \label{BDS_like_amplitude}
\end{equation}
for any label $X$, we expect that ${}^4 {\cal E}_8$ should satisfy Steinmann relations between all partially overlapping pairs of three-particle invariants, while ${}^3 {\cal E}_8$ should satisfy Steinmann relations between all partially overlapping pairs of four-particle invariants. That is, ${}^4 {\cal E}_8$ is expected to satisfy the relations
\begin{equation}
\begin{split}
\text{Disc}_{\mand{j,j+1,j+2}}\left[\text{Disc}_{\mand{i,i+1,i+2}} \big({}^4 {\cal E}_8 \big) \right] &= 0, \quad  j \in \{ i \pm 2, i \pm 1 \}, \label{stein33}
\end{split}
\end{equation}
while ${}^3 {\cal E}_8$ is expected to satisfy
\begin{equation}
\begin{split}
\text{Disc}_{\mand{j,j+1,j+2,j+3}}\left[\text{Disc}_{\mand{i,i+1,i+2,i+3}} \big({}^3 {\cal E}_8 \big) \right] &= 0, \quad j \in \{ i \pm 3, i \pm 2, i \pm 1 \}.  \label{stein44}
\end{split}
\end{equation}
Due to momentum conservation in eight-point kinematics, the six relations in~\eqref{stein44} corresponding to a given $i$ only result in three independent constraints; however, these relations will be independent for larger $n$.

Since the functions ${}^4 Y_{8}$ and ${}^3 Y_{8}$ are not unique, it may be possible to chose them in a way that endows the corresponding amplitudes ${}^4 {\cal E}_8$ and ${}^3 {\cal E}_8$ with additional simplifications. \draftnote{\dots} In general, they can be expressed as classical polylogarithms taking \begin{align}
\mathfrak{X}_{i,8} &= \frac{\langle i,i+1,i+2,i+4 \rangle \langle i+1,i+3,i+4,i+5\rangle}{\langle i,i+1,i+4,i+5 \rangle \langle i+1,i+2,i+3,i+4 \rangle}, \\
\mathfrak{X}_{i,4} &= \frac{\langle i,i+1,i+3,i+7 \rangle \langle i,i+2,i+3,i+4 \rangle}{\langle i,i+1,i+2,i+3 \rangle \langle i,i+3,i+4,i+7 \rangle},
\end{align}
as negative arguments, where $\mathfrak{X}_{i,8}$ and $\mathfrak{X}_{i,4}$ are ${\cal X}$-coordinates in Gr(4,8) that respectively carve out an eight-orbit and a four-orbit of the dihedral group. The dilog part of these functions is uniquely determined by~\eqref{bds_like_4} and~\eqref{bds_like_3}, but diagonalizing the $(\text{Li}_1)^2$ contribution in these variables gives the representations
\begin{align}
{}^4 Y_8 &= \sum_{i=1}^8 \bigg[ \text{Li}_2 \left( - \mathfrak{X}_{i,8} \right) + \frac12 \text{Li}_2 \left(- \mathfrak{X}_{i,4}  \right) + \frac14 \text{Li}_1\left(- \mathfrak{X}_{i,4} \right)^2 \bigg], \\
{}^3 Y_8 &= \sum_{i=1}^8 \bigg[ \text{Li}_2 \left( - \mathfrak{X}_{i,8} \right) + \frac12 \text{Li}_2 \left(- \mathfrak{X}_{i,4}  \right) + \frac12 \text{Li}_1\left(- \mathfrak{X}_{i,8} \right)^2 \bigg].
\end{align}
, so  

Conditions~\eqref{stein33} and~\eqref{stein44} don't exhaust the set of Steinmann relations obeyed by generic eight-particle amplitudes---there are also Steinmann relations between partially overlapping three- and four-particle invariants. If we want to make these additional relations manifest, we can instead define a BDS-like ansatz that depends on only a subset of the three- or four-particle invariants. In particular, it proves possible to decompose the BDS ansatz as either
\begin{align}
{\cal A}_8^{\text{BDS}}(\{\mand{i,\dots,i+k}\}) &= {}^{{\{a,b\}}_4} {\cal A}_8^{\text{BDS-like}}(\{\mand{i,i+1}\}, \{\mand{i,i+1,i+2,i+3} | i \in \{a,b\} \})  \label{bds_like_4q} \\ 
&\hspace{5.6cm} \times \exp \left[ - \frac{\Gamma_{\text{cusp}}}{4}\ {}^{{\{a,b\}}_4} Y_{8}(\{u_i\})  \right], \nonumber  \\
{\cal A}_8^{\text{BDS}}(\{\mand{i,\dots,i+k}\}) &= {}^{{\{a,b\}}_3} {\cal A}_8^{\text{BDS-like}}(\{\mand{i,i+1}\}, \{\mand{i,i+1,i+2} | i \in \{a,b\} \})  \label{bds_like_3q} \\ 
&\hspace{5.6cm} \times \exp \left[ - \frac{\Gamma_{\text{cusp}}}{4}\ {}^{{\{a,b\}}_3} Y_{8}(\{u_i\})  \right], \nonumber 
\end{align}
for any $\{a,b\}$ such that $b-a$ is odd.\footnote{The difference $b-a$ should be computed mod 8 in the case of ${}^{{\{a,b\}}_3} {\cal A}_8^{\text{BDS-like}}$ since $\mand{i+8,\dots,i+k+8} = \mand{i,\dots,i+k}$ in general, but should be computed mod 4 in the case of ${}^{{\{a,b\}}_4} {\cal A}_8^{\text{BDS-like}}$ since momentum conservation implies the stronger identity $\mand{i+4,i+5,i+6,i+7} = \mand{i,i+1,i+2,i+3}$ between four-particle invariants.} Any solution to~\eqref{bds_like_4q} defines a BDS-like normalized amplitude $\EfourJ$ that respects the Steinmann relations
\begin{equation}
\begin{rcases}
\text{Disc}_{\mand{j,j+1,j+2}}\left[\text{Disc}_{\mand{i,i+1,i+2,i+3}} \big(\EfourJ \big) \right] \! \! \! \! &= 0, \hspace{.3cm} \\
\text{Disc}_{\mand{i,i+1,i+2,i+3}}\left[\text{Disc}_{\mand{j,j+1,j+2}} \big(\EfourJ \big) \right] \! \! \! \! &= 0, \label{stein34}
\end{rcases} \quad 
\begin{gathered} i \notin \{a,b\}, \\ j \in \{i-2, i-1, i+2, i+3\}, \end{gathered}
\end{equation}
in addition to all the Steinmann relations satisfied by ${}^4 {\cal E}_8$ as given in eq.~\eqref{stein33}. Moreover, it will respect many of the Steinmann relations satisfied by ${}^3 {\cal E}_8$---namely, those that don't involve a discontinuity in either $\mand{a,a+1,a+2,a+3}$ or $\mand{b,b+1,b+2,b+3}$. Similarly, any solution to~\eqref{bds_like_3q} defines an amplitude $\EthreeJ$ that respects
\begin{equation}
\begin{rcases}
\text{Disc}_{\mand{i,i+1,i+2}}\left[\text{Disc}_{\mand{j,j+1,j+2,j+3}} \big(\EthreeJ \big) \right] \! \! \! \! &= 0, \hspace{.3cm} \\
\text{Disc}_{\mand{j,j+1,j+2,j+3}}\left[\text{Disc}_{\mand{i,i+1,i+2}} \big(\EthreeJ \big) \right] \! \! \! \! &= 0, \label{stein43}
\end{rcases} \quad 
\begin{gathered} i \notin \{a,b\}, \\ j \in \{i-3, i-2, i+1, i+2\}, \end{gathered}
\end{equation}
%\begin{equation}
%\begin{rcases}
%\text{Disc}_{\mand{i+2,i+3,i+4}}\left[\text{Disc}_{\mand{i,i+1,i+2,i+3}} \big(\EfourJ \big) \right] \! \! \! \! &= 0, \hspace{.3cm} \\
%\text{Disc}_{\mand{i+3,i+4,i+5}}\left[\text{Disc}_{\mand{i,i+1,i+2,i+3}} \big(\EfourJ \big) \right] \! \! \! \! &= 0,  \\
%\hspace{0.324cm} \text{Disc}_{\mand{i-1,i,i+1}}\left[\text{Disc}_{\mand{i,i+1,i+2,i+3}} \big(\EfourJ \big) \right] \! \! \! \! &= 0,  \\
%\hspace{0.324cm} \text{Disc}_{\mand{i-2,i-1,i}}\left[\text{Disc}_{\mand{i,i+1,i+2,i+3}} \big(\EfourJ \big) \right] \! \! \! \! &= 0,  \\
%\text{Disc}_{\mand{i,i+1,i+2,i+3}}\left[\text{Disc}_{\mand{i+2,i+3,i+4}} \big(\EfourJ \big) \right] \! \! \! \! &= 0,  \\
%\text{Disc}_{\mand{i,i+1,i+2,i+3}}\left[\text{Disc}_{\mand{i+3,i+4,i+5}} \big(\EfourJ \big) \right] \! \! \! \! &= 0, \\
%\hspace{0.324cm} \text{Disc}_{\mand{i,i+1,i+2,i+3}}\left[\text{Disc}_{\mand{i-1,i,i+1}} \big(\EfourJ \big) \right] \! \! \! \! &= 0, \\
%\hspace{0.324cm} \text{Disc}_{\mand{i,i+1,i+2,i+3}}\left[\text{Disc}_{\mand{i-2,i-1,i}} \big(\EfourJ \big) \right] \! \! \! \! &= 0, \label{stein34}
%\end{rcases} i \notin \{a,b\}
%\end{equation}
as well as all the Steinmann relations satisfied by ${}^3 {\cal E}_8$ and described in eq.~\eqref{stein44}, and all the relations specified in eq.~\eqref{stein33} that don't involve a discontinuity in either $\mand{a,a+1,a+2}$ or $\mand{b,b+1,b+2}$. Clearly it is not possible for BDS-like amplitudes of either type to be Bose-symmetric; however, it is possible to construct a solution to~\eqref{bds_like_3q} such that $\EthreeJ$ is invariant under the dihedral flip $s_{i,\dots,i+k} \rightarrow s_{9-i,\dots,9-i-k}$ when this flip is oriented to map $\mand{a,a+1,a+2}$ and $\mand{b,b+1,b+2}$ between each other.


\newpage
These considerations implies that all eight-point Steinmann relations are made manifest in at least one of the amplitudes 
$$\{{}^4 {\cal E}_8, {}^{{\{1,2\}}_3} {\cal E}_8 ,  {}^{{\{5,6\}}_3} {\cal E}_8 \}.$$ 
However, in practice it may prove convenient to retain ${}^{3} {\cal E}_8$ in one's toolbox, since it can be chosen to respect Bose symmetry. The extended Steinmann relations~\eqref{stein33},~\eqref{stein44}, and~\eqref{stein34} can be checked by computing the appropriate BDS-like normalized amplitudes from the remainder function, as per eq.~\eqref{BDS_like_amplitude}; while all of these relations are satisfied as expected, all further relations are violated at two loops (for any choice of...)

\draftnote{To Do: can any given Steinmann relation be saved (in Bose-symmetric or ...)? Any other features of the full space worth working out?}

Explicit forms for ${}^3 Y_{8}(\{u_i\})$, ${}^4 Y_{8}(\{u_i\})$, and ${}^{3,j} Y_{8}(\{u_i\})$ are provided in appendix~\ref{appendix:bds_like}.

\draftnote{To Do: define $\Gamma_{\text{cusp}}$ in this section if we don't earlier}

\draftnote{To Do: comment about the fact that we don't know how to extend the Steinmann relations beyond symbol level (or figure out how to do so...)}


\subsection{Cluster Adjacency in ${\cal A}$- and ${\cal X}$-coordinates}

The extended Steinmann relations~\eqref{stein33_1} through~\eqref{stein34_8} can be checked by computing the appropriate BDS-like normalized amplitudes from the remainder function, as per eq.~\eqref{BDS_like_amplitude}. While these relations are satisfied, every Steinmann relation that is not preserved by the choice of BDS-like ansatz is violated by these amplitudes.

\draftnote{To Do: discuss which nonadjacent pairs appear in the amplitude}
\draftnote{To Do: email Christian to ask about the Sklyanin bracket on ${\cal A}$-coordinates, and about $\delta = \rho \circ \delta \circ \rho$}
\draftnote{To Do: discuss the Sklyanin bracket on ${\cal A}$-coordinates}

\subsection{Restoring all Steinmann Relations}

\draftnote{To Do: discuss the possibility of repairing Steinmann and cluster adjacency at the cost of dual conformal invariance, and also in special kinematic limits where the additional three- or four-particle dependence drops out}


\section{Conclusion}

\appendix
\section{\texorpdfstring{$A_5$}{A5} Representation of \texorpdfstring{$R^{(2)}_7$}{R27}}

\section{BDS-Like Conversions for Eight Particles} \label{appendix:bds_like}

The decompositions~\eqref{bds_like_3}, \eqref{bds_like_4}, and \eqref{bds_like_3q} do not uniquely determine ${}^{3} Y_{8}$, ${}^{4} Y_{8}$, or ${}^{3,j} Y_{8}$. In fact, there exists a 10-dimensional (3-dimensional) space of (Bose-symmetric) solutions for ${}^{3} Y_{8}$, a 36-dimensional (5-dimensional) space of (Bose-symmetric) solutions for ${}^{4} Y_{8}$, and a 3-dimensional space of solutions for ${}^{3,j} Y_{8}$. 
 \begin{align}
{}^{3,1}Y_8 &= {}^{3}Y_8 -
\Big( \text{Li}_1(-\mathfrak{X}_{1, 4}) + \text{Li}_1(-\mathfrak{X}_{2, 4}) + \text{Li}_1(-\mathfrak{X}_{1, 8}) + \text{Li}_1(-\mathfrak{X}_{5, 8}) \Big)  \\
&\hspace{3.4cm} \times \Big( \text{Li}_1(-\mathfrak{X}_{1, 4}) + \text{Li}_1(-\mathfrak{X}_{4, 4}) + \text{Li}_1(-\mathfrak{X}_{4, 8}) + \text{Li}_1(-\mathfrak{X}_{8, 8}) \Big) \nonumber
\end{align}

 \begin{align*}
 &\ \hspace{1.4cm}- \log(s_{\scaleto{1234}{4.4pt}} s_{\scaleto{3456}{4.4pt}}) \log(s_{\scaleto{2345}{4.4pt}} s_{\scaleto{4567}{4.4pt}}) \nonumber
 \end{align*}
 
 \begin{align*}
&\ \hspace{2.4cm}- \frac12 \log(\mand{i,i+1,i+2}) \log\left(\frac{\mand{i,i+1,i+2} \ \mand{i+1,i+2,i+3}^2}{\mand{i+4,i+5,i+6}}\right) \bigg] \nonumber
 \end{align*}
 
 To take full advantage of the Steinmann relations, it is convenient to work in terms of symbol letters that isolate different Mandelstam invariants. There are twelve independent dual conformally invariant cross ratios that can appear in these symbols
\begin{align}
u_1 &= \frac{\mand{12} \mand{4567}}{\mand{123} \mand{812}}, \quad \text{and cyclic (8-orbit)} \\
u_9 &= \frac{\mand{123} \mand{567}}{\mand{1234} \mand{4567}}, \quad \text{and cyclic (4-orbit).}
\end{align}
It is not possible to isolate all three- and four-particle Mandelstam invariants simultaneously into twelve different symbol letters. (More than twelve symbol letters will appear in these amplitudes, but we here restrict our attention to the twelve that will appear in the first entry.) However, different choices of letters can be made such that either all the four-particle invariants, or all the three-particle invariants, are isolated.

One choice that isolates the four-particle invariants is
\begin{align}
{}^4 d_1 &= u_2 \ u_6 = \frac{\mand{23} \ \mand{67} \ (\mand{1234})^2}{\mand{123} \ \mand{234} \ \mand{567} \ \mand{678}}, \quad \text{and cyclic (4-orbit)} \\
{}^4 d_5 &= u_2/u_6 = \frac{\mand{23} \ \mand{567} \ \mand{678}}{\mand{67} \ \mand{123} \ \mand{234}}, \quad \text{and cyclic (4-orbit)} \\
{}^4 d_9 &= u_1 \ u_2 \ u_5 \ u_6 \ u_9^2 = \frac{\mand{12} \ \mand{23} \ \mand{56} \ \mand{67}}{\mand{234} \ \mand{456} \ \mand{678} \ \mand{812}}, \quad \text{and cyclic (4-orbit)}.
\end{align}
In this alphabet ${}^4 d_1, {}^4 d_2, {}^4 d_3$, and ${}^4 d_4$ each contain a different four-particle Mandelstam invariant, while the other letters only involve two- and three-particle invariants. The extended Steinmann relations then tell us that ${}^4 d_1, {}^4 d_2, {}^4 d_3$, and ${}^4 d_4$ can never appear next to each other in the symbol of ${}^4 A^{\text{BDS-like}}_{8}$ (but each can still appear next to themselves).

Similarly, we can isolate the three-particle invariants by choosing
\begin{align}
{}^3 d_1 &= \frac{u_1 \ u_2 \ u_4 \ u_7}{u_3 \ u_5 \ u_6 \ u_8 \ u_9^2} = \frac{\mand{12} \ \mand{23} \ \mand{45} \ \mand{78} \ (\mand{1234})^2 \ (\mand{4567})^2}{\mand{34} \ \mand{56} \ \mand{67} \ \mand{81} \ (\mand{123})^2}, \quad \text{and cyclic (8-orbit)} \\
{}^3 d^4_9 &= u_1 \ u_5 \ u_9 \ u_{12} = \frac{\mand{12} \ \mand{56}}{\mand{1234} \ \mand{3456}}, \quad \text{and cyclic (4-orbit)},
\end{align}
in which case ${}^3 d_1$ through ${}^3 d_8$ each contain a different three-particle Mandelstam invariant, as well as four-particle Mandelstams that they don't partially overlap with. The remaining four letters only contain two- and four-particle invariants. In these letters, conditions~\eqref{stein34_5} and~\eqref{stein34_6} tell us that ${}^3 d_7, {}^3 d_8, {}^3 d_2$, and ${}^3 d_3$ can never appear next to ${}^3 d_1$ in the symbols of ${}^{3} {\cal E}_8$ or ${}^{3,j} {\cal E}_8$ (plus the cyclic images of this statement). Moreover, conditions~\eqref{stein34_1} through~\eqref{stein34_4} give us the additional restrictions that none of ${}^3 d_1, {}^3 d_5, {}^3 d_9$ and ${}^3 d_{10}$ can ever appear next to ${}^3 d_3, {}^3 d_4, {}^3 d_7,$ or ${}^3 d_8$ in the symbol of ${}^{3,1} {\cal E}_8$ (analogous relations hold for the other ${}^{3,j} {\cal E}_8$). These are the restrictions given by the Steinmann relations involving $\mand{1234}$ and one of $\mand{781}, \mand{812}, \mand{345}$, or $\mand{456}$. The other Steinmann relations between three- and four-particle invariants will not be respected by ${}^{3,1} {\cal E}_8$, since ${}^{3,j} {\cal A}^{\text{BDS-like}}_{8}$ depends on $\mand{2345}, \mand{3456},$ and $\mand{4567}$.


\bibliographystyle{ieeetr}

\bibliography{eight_particle_two_loop}

\end{fmffile}
\end{document}
