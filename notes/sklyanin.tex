\pdfoutput=1


\documentclass[12pt]{article}

\usepackage{amssymb}
\usepackage{amsfonts}
\usepackage{amsmath}
\usepackage{fancyhdr} 

\DeclareMathOperator{\B}{B}
\DeclareMathOperator{\Conf}{Conf}
\DeclareMathOperator{\Gr}{Gr}
\DeclareMathOperator{\Id}{Id}
\DeclareMathOperator{\Li}{Li}
\DeclareMathOperator{\Lie}{Lie}
\DeclareMathOperator{\sgn}{sgn}

\def\ket#1{\langle #1 \rangle}


\makeindex
\oddsidemargin -0.04cm \evensidemargin -0.04cm
\topmargin -0.25cm \textwidth 16.59cm \textheight 20.5cm \headheight 15pt

\begin{document}

\thispagestyle{fancyplain}
 
\fancyhf{}
 
\cfoot{\fancyplain{}{\thepage}}

\lhead{\textbf{Computing the Sklyanin Bracket} \hfill \today}

In momentum twistor language we have the $n$ momentum twistors $Z_i$, which together form the $4 \times n$ matrix
\begin{equation}	
K = \left(\begin{array}{ccc}
z_{11} & \ldots & z_{n1} \\
z_{12} & \ldots & z_{n2} \\
z_{13} & \ldots & z_{n3} \\
z_{14} & \ldots & z_{n4}\end{array}\right).
\end{equation}
As long as the first 4 columns are non-singular, we can row reduce $K$ in to the form
\begin{equation}
K'=\left(
\begin{array}{ccccccc}
 1 & 0 & 0 & 0 & y_{51} & \ldots  & y_{n1} \\
 0 & 1 & 0 & 0 & y_{52} & \ldots  & y_{n2} \\
 0 & 0 & 1 & 0 & y_{53} & \ldots  & y_{n3} \\
 0 & 0 & 0 & 1 & y_{54} & \ldots  & y_{n4} \\
\end{array}
\right).
\end{equation}
The columns of $K'$ define a new set of momentum twistors $Z'_i$, where for example $Z'_1 = \{1,0,0,0\}$ and $Z'_5 = \{y_{51},y_{52},y_{53},y_{54}\}$. It is easy to check that 
\begin{align}
	&y_{ij} = (-1)^j \ket{\{1,2,3,4\}\setminus\{j\},i}/\ket{1234},\label{eq:numerics}\\
	&\ket{abcd}' = \det(Z'_a Z'_b Z'_c Z'_d) = \ket{abcd}/\ket{1234}.
\end{align}
The Sklyanin bracket on these $y_{ij}$ is given by
\begin{equation}
	\{y_{ij},y_{ab}\} = (\sgn(a-i) - \sgn(b-j)) y_{ib} y_{aj}.
\end{equation}
This extends to a bracket on functions of the $y_{ij}$ via
\begin{equation}\label{def:sklyanin}
	\{f(y), g(y)\} =  \sum_{i,a=1}^n\sum_{j,b=1}^4\frac{\partial f}{\partial y_{ij}}  \frac{\partial g}{\partial y_{ab}} 
\{y_{ij}, y_{ab}\}.
\end{equation}
What we commonly refer to as ``the Poisson bracket'' between two $\mathcal{X}$-coordinates, $\{\mathcal{X}_1,\mathcal{X}_2\}$ is then calculated via
\begin{equation}
	\{\mathcal{X}_1,\mathcal{X}_2\} = \frac{1}{2 \mathcal{X}_1\mathcal{X}_2} \{\mathcal{X}_1',\mathcal{X}_2'\}
\end{equation}
where $\mathcal{X}_i' = \mathcal{X}_i$ with $\ket{ijkl} \to \ket{ijkl}'$. 
\end{document}
