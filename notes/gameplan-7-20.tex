\documentclass[12pt]{article}

\usepackage{amssymb}
\usepackage{amsfonts}
\usepackage{amsmath}
\usepackage{fancyhdr}
\usepackage{parskip} 

\DeclareMathOperator{\Li}{Li}

\oddsidemargin -0.04cm 
\evensidemargin -0.04cm
\topmargin -0.25cm 
\textwidth 16.59cm 
\textheight 20.5cm

\begin{document}

\thispagestyle{fancyplain}
\fancyhf{}

\lhead{\textbf{Cluster Polylogarithms Gameplan} \hfill \today}

\fbox{\emph{Goal: closed-form representation of n-particle, two-loop MHV amplitude in N=4 SYM.}}\\ \\
\emph{Fall-back goal: 8-particle, two-loop MHV amplitude.}\\ \\

\underline{Individual steps}
\begin{itemize}
	\item Generate subalgebras of finite cluster algebras: $A_2, A_3, A_4, D_4, A_5, D_5, E_6$.
	\item Apply $f_{A_2}$ and $f_{A_3}$ across subalgebras, cataloguing identities etc.
	\item Find larger subalgebras with ``unique'' functions by applying analytic and/or algebraic criteria such as:
		\begin{itemize}
		\item collinear limits
		\item branch cuts (first-entry, Steinmann)
		\item Poisson bracket structure
		\end{itemize}
	\item Related, find ``completions'' (products of lower weight functions) for $f_{A_2}$ and $f_{A_3}$ by considering similar criteria.
	\item Build 7- and 8-pt amplitudes out of new larger subalgebra functions, hopefully in a fully unique way. (Can build 8-pt amplitude out of pre-existing functions, so this is the fall-back paper.)
	\item Use 7- and 8-pt amplitude structure as guide for constructing closed-form representation of $n$-particle, two-loop MHV amplitude. \\\\
\end{itemize}

\underline{Other Opportunities}
\begin{itemize}
	\item Higher-weight cluster polylogarithms (test Goncharov hypothesis, try to re-write 6-pt 3-loop in terms of cluster functions, etc).
	\item Re-write known non-MHV results in cluster algebra language.
	\item New functional relations for $\Li_4$ (and higher?) of cluster algebraic origin.
	\item Understand larger connections between Feynman diagram integrals and cluster algebra polylogarithms.
\end{itemize}

\end{document}
