\pdfoutput=1


\documentclass[12pt]{article}
\usepackage{epsfig}
\usepackage{fancyhdr} 
\usepackage{amssymb}
\usepackage{amsmath}
\usepackage{tikz}
\usepackage{mathrsfs}
\usepackage{hyperref}
\usepackage{multirow}
\usepackage{scalerel}
\usepackage{mathtools}
\usepackage{textcomp}
\usepackage{color}
\usepackage[all]{xy}

\usetikzlibrary{calc}


\DeclareMathOperator{\B}{B}
\DeclareMathOperator{\Conf}{Conf}
\DeclareMathOperator{\Gr}{Gr}
\DeclareMathOperator{\Li}{Li}
\DeclareMathOperator{\sgn}{sgn}


\def\ket#1{\langle #1 \rangle}
\def\nl{\nonumber\\}
\def\nn{\nonumber}
\def\x{\mathcal{X}}
\def\xcoord{$\mathcal{X}$-coordinate }
\def\xcoords{$\mathcal{X}$-coordinates }
\def\a{\mathcal{A}}
\def\acoord{$\mathcal{A}$-coordinate }
\def\acoords{$\mathcal{A}$-coordinates }
\def\draftnote#1{{\bf [#1]}}
\def\flag{{\huge \color{red} \textinterrobang}}
\def\pdfeq#1{\texorpdfstring{$#1$}{a}}

\def\fd5{f_{D_5}}
\def\fa3{f_{A_3}}
\def\rn{R^{(2)}_n}
\def\rsev{R^{(2)}_7}

\def\bb2{B_2\wedge B_2}
\def\b3c{B_3 \otimes \mathbb{C}^*}


\def\drawPentagon{
\coordinate (P1) at (90:1);
\coordinate (P2) at (18:1);
\coordinate (P3) at (306:1);
\coordinate (P4) at (234:1);
\coordinate (P5) at (162:1);
\draw (P1) -- (P2) -- (P3) -- (P4) -- (P5) -- cycle;
}

\def\drawLabeledPentagon{
\coordinate (P1) at (90:1);
\coordinate (P2) at (18:1);
\coordinate (P3) at (306:1);
\coordinate (P4) at (234:1);
\coordinate (P5) at (162:1);
\draw (P1) -- (P2) -- (P3) -- (P4) -- (P5) -- cycle;
\draw (0,1.2) node {1};
\draw (1,.3) node[anchor=west] {2};
\draw (.5,-.9) node[anchor=west] {3};
\draw (-.5,-.9) node[anchor=east] {4};
\draw (-1,.3) node[anchor=east] {5};
}

\def\drawHexagon{
\coordinate (P1) at (90:1);
\coordinate (P2) at (30:1);
\coordinate (P3) at (330:1);
\coordinate (P4) at (270:1);
\coordinate (P5) at (210:1);
\coordinate (P6) at (150:1);
\draw (P1) -- (P2) -- (P3) -- (P4) -- (P5) -- (P6) -- cycle;
}


\def\drawLabeledHexagon{
\coordinate (P1) at (90:1);
\coordinate (P2) at (30:1);
\coordinate (P3) at (330:1);
\coordinate (P4) at (270:1);
\coordinate (P5) at (210:1);
\coordinate (P6) at (150:1);
\draw (P1) -- (P2) -- (P3) -- (P4) -- (P5) -- (P6) -- cycle;
\draw (0,1.2) node {1};
\draw (30:1.1) node[anchor=west] {2};
\draw (330:1.1) node[anchor=west] {3};
\draw (270:1.2) node {4};
\draw (210:1.1) node[anchor=east] {5};
\draw (150:1.1) node[anchor=east] {6};
}

\def\drawOctagon{
\coordinate (P1) at (45:1);
\coordinate (P2) at (90:1);
\coordinate (P3) at (135:1);
\coordinate (P4) at (180:1);
\coordinate (P5) at (225:1);
\coordinate (P6) at (270:1);
\coordinate (P7) at (315:1);
\coordinate (P8) at (359:1);
\draw (P1) -- (P2) -- (P3) -- (P4) -- (P5) -- (P6) -- (P7) -- (P8) -- cycle;
}


\def\mand#1{\scaleto{s}{4.6pt}_{\scaleto{#1}{5.2pt}}}
\def\EthreeJ{{}^{{\{a,b\}}_3} {\cal E}_8}
\def\EfourJ{{}^{{\{a,b\}}_4} {\cal E}_8}
\def\LiOneCalX#1#2{\text{Li}_1(-\mathfrak{X}_{#1,#2})}
\def\LiOneBarCalX#1#2{\text{Li}_1(-\overline{\mathfrak{X}}_{#1,#2})}

\newcommand{\cP}{{\cal P}}
\def\lr{\leftrightarrow}

\makeindex
\oddsidemargin -0.04cm \evensidemargin -0.04cm
\topmargin -0.25cm \textwidth 16.59cm \textheight 20.5cm \headheight 15pt

\begin{document}

\thispagestyle{fancyplain}
 
\fancyhf{}
 
\cfoot{\fancyplain{}{\thepage}}

\lhead{\textbf{Working through subalgebra-constructibility for $A_2 \subset A_3$}}

In this section we determine the space of weight-4 cluster polylogarithms associated with the $A_3$ cluster algebra. 

A comment on notation: when we refer to simply ``$A_3$ functions'', we are implicitly referring to $A_2 \subset A_3$ functions, i.e. using the $A_2$ subalgebras to construct a function. This is the weakest form of subalgebra constructibility since $A_2$ functions, evaluated across all $A_2$ subalgebras of a larger algebra, form a basis for all cluster polylogarithms on the larger algebra. This is why we don't explicitly refer to $A_2$. However in the future we will look at, for example, $A_3 \subset D_5$ functions, and in that case we will explicitly refer to the fact that we are using $A_3$ functions as building blocks (although they are of course already built out of $A_2$'s). 

Begin with the six $A_2$ subalgebras in $A_3$ -- as discussed in a previous section, there are multiple ways to think of these subalgebras. The simplest is that these are the $\binom{6}{5}$ pentagons embeddable in a hexagon. These are also present as the six pentagonal faces in the $A_3$ associahedron, fig. (ref). In practice it is most practical to list the six distinct $A_2$ seeds that appear in the 14 cluster generated by mutating on the standard $A_3$ seed, $x_1 \to x_2 \to x_3$:
\begin{equation}\label{eq:a2-in-a3}
\begin{gathered}
	x_1 \to x_2, \quad 
	x_2 \to x_3, \quad 
	\frac{x_2}{1+x_{12}}\to \frac{\left(1+x_1\right) x_3}{1+x_{123}},\\ \\
	\frac{x_1 x_2}{1+x_1}\to x_3,\quad 
	x_1 \left(1+x_2\right)\to \frac{x_2 x_3}{1+x_2},\quad
	 x_1\to x_2 \left(1+x_3\right).
\end{gathered}	
\end{equation}

We now construct an ansatz for $f_{A_3}$ as a sum of $f_{A_2}$ evaluated on these six subalgebras:
\begin{equation}
	f_{A_3} = c_1 f_{A_2}(x_1 \to x_2) + \ldots + c_6 f_{A_2}(x_1\to x_2 \left(1+x_3\right)). 
\end{equation}	
Due to the inherited properties of $f_{A_2}$, the above ansatz already has a cluster-y cobracket, satisfies cluster adjacency, and is smooth and real-valued in the positive domain. The remaining problem to solve is to find values for the $c_i$ such that $f_{A_3}$ is invariant (up to an overall sign) under the automorphisms of $A_3$. These were discussed in a previous section (ref), we write them down explicitly here:
\begin{align}
	&\sigma_{A_3}:\quad x_1 \to \frac{x_2}{1+x_1 + x_1 x_2}, ~~x_2 \to \frac{x_3(1+x_1)}{1+x_1 + x_1x_2 +x_1x_2x_3},~~ x_3 \to \frac{1+x_1 + x_1 x_2}{x_1x_2x_3},\nl \\
	&\tau_{A_3}:\quad x_1 \to \frac{1}{x_3}, ~~x_2\to\frac{1}{x_2},~~x_3\to \frac{1}{x_1}.\nn
\end{align}
Recall that $\sigma$ is essentially the cyclic automorphism on the hexagon, and $\tau$ is the dihedral flip. 

The first thing to try is to make $f_{A_3}$ completely invariant under the $A_3$ automorphisms:
\begin{equation}\label{eq:fA3++}
	\sigma_{A_3}(f_{A_3}) = f_{A_3},\quad \tau_{A_3}(f_{A_3}) = f_{A_3}.  
\end{equation}
However it turns out there is no collection of $c_i$ that solves such a constraint. If instead we impose 
\begin{equation}
	\sigma_{A_3}(f_{A_3}) = f_{A_3},\quad \tau_{A_3}(f_{A_3}) = -f_{A_3},
\end{equation}
Then we find the solution 
\begin{equation}
	c_i = \text{constant}.
\end{equation}
We label this particular solution by 
\begin{equation}
	f_{A_3}^{+-} = f_{A_2}(x_1 \to x_2) + \ldots + f_{A_2}(x_1\to x_2 \left(1+x_3\right)) = \sum_{i=1}^6 \sigma_{A_3}^i\big(f_{A_2}(x_1\to x_2)\big),
\end{equation}
where the signs indicate behavior under $\sigma$ and $\tau$, respectively. Similarly, based on the lack of a solution for eq.~(\ref{eq:fA3++}) we say that $f_{A_3}^{++} = 0$. There are two remaining sign choices to check, $f_{A_3}^{-+}$ and $f_{A_3}^{--}$, and we find $f_{A_3}^{-+} = 0$ and		
\begin{equation}
	f_{A_3}^{--} =\sum_{i=1}^6(-1)^i\sigma_{A_3}^i \big(f_{A_2}(x_1\to x_2)\big).
\end{equation}
In conclusion, there are choices for $f_{A_3}$, which we denote by their behavior under the dihedral cycle and flip automorphisms: $f_{A_3}^{+-}$ and $f_{A_3}^{--}$. These functions arise purely from the interplay between the overall symmetries of the $A_3$ cluster algebra and the structure of the $A_2$ subalgebras in $A_3$, i.e. there has been no physics input so far. We can tabulate our results in the following table:
\begin{equation}\label{eq:a3-func-counts}
\begin{array}{ l | c | c | c | c }			
  & \sigma^+\tau^+ & \sigma^+\tau^- & \sigma^-\tau^+ & \sigma^-\tau^- \\
  \hline
  A_3 & 0 & 1 & 0 & 1 \\  
\end{array} 
\end{equation}
This displays the total number of weight-4 polylogarithm functions associated with the $A_3$ cluster algebra with the given sign changes under the $\sigma$ and $\tau$ automorphisms. 

\end{document}
