\pdfoutput=1


\documentclass[12pt]{article}
\usepackage{epsfig}
\usepackage{fancyhdr} 
\usepackage{amssymb}
\usepackage{amsmath}
\usepackage{tikz}
\usepackage{mathrsfs}
\usepackage{hyperref}
\usepackage{multirow}
\usepackage{scalerel}
\usepackage{mathtools}
\usepackage{textcomp}
\usepackage{color}
\usepackage[all]{xy}

\usetikzlibrary{calc}


\DeclareMathOperator{\B}{B}
\DeclareMathOperator{\Conf}{Conf}
\DeclareMathOperator{\Gr}{Gr}
\DeclareMathOperator{\Li}{Li}
\DeclareMathOperator{\sgn}{sgn}


\def\ket#1{\langle #1 \rangle}
\def\nl{\nonumber\\}
\def\nn{\nonumber}
\def\x{\mathcal{X}}
\def\xcoord{$\mathcal{X}$-coordinate }
\def\xcoords{$\mathcal{X}$-coordinates }
\def\a{\mathcal{A}}
\def\acoord{$\mathcal{A}$-coordinate }
\def\acoords{$\mathcal{A}$-coordinates }
\def\draftnote#1{{\bf [#1]}}
\def\flag{{\huge \color{red} \textinterrobang}}
\def\pdfeq#1{\texorpdfstring{$#1$}{a}}

\def\fd5{f_{D_5}}
\def\fa3{f_{A_3}}
\def\rn{R^{(2)}_n}
\def\rsev{R^{(2)}_7}

\def\bb2{B_2\wedge B_2}
\def\b3c{B_3 \otimes \mathbb{C}^*}


\def\drawPentagon{
\coordinate (P1) at (90:1);
\coordinate (P2) at (18:1);
\coordinate (P3) at (306:1);
\coordinate (P4) at (234:1);
\coordinate (P5) at (162:1);
\draw (P1) -- (P2) -- (P3) -- (P4) -- (P5) -- cycle;
}

\def\drawLabeledPentagon{
\coordinate (P1) at (90:1);
\coordinate (P2) at (18:1);
\coordinate (P3) at (306:1);
\coordinate (P4) at (234:1);
\coordinate (P5) at (162:1);
\draw (P1) -- (P2) -- (P3) -- (P4) -- (P5) -- cycle;
\draw (0,1.2) node {1};
\draw (1,.3) node[anchor=west] {2};
\draw (.5,-.9) node[anchor=west] {3};
\draw (-.5,-.9) node[anchor=east] {4};
\draw (-1,.3) node[anchor=east] {5};
}

\def\drawHexagon{
\coordinate (P1) at (90:1);
\coordinate (P2) at (30:1);
\coordinate (P3) at (330:1);
\coordinate (P4) at (270:1);
\coordinate (P5) at (210:1);
\coordinate (P6) at (150:1);
\draw (P1) -- (P2) -- (P3) -- (P4) -- (P5) -- (P6) -- cycle;
}


\def\drawLabeledHexagon{
\coordinate (P1) at (90:1);
\coordinate (P2) at (30:1);
\coordinate (P3) at (330:1);
\coordinate (P4) at (270:1);
\coordinate (P5) at (210:1);
\coordinate (P6) at (150:1);
\draw (P1) -- (P2) -- (P3) -- (P4) -- (P5) -- (P6) -- cycle;
\draw (0,1.2) node {1};
\draw (30:1.1) node[anchor=west] {2};
\draw (330:1.1) node[anchor=west] {3};
\draw (270:1.2) node {4};
\draw (210:1.1) node[anchor=east] {5};
\draw (150:1.1) node[anchor=east] {6};
}

\def\drawOctagon{
\coordinate (P1) at (45:1);
\coordinate (P2) at (90:1);
\coordinate (P3) at (135:1);
\coordinate (P4) at (180:1);
\coordinate (P5) at (225:1);
\coordinate (P6) at (270:1);
\coordinate (P7) at (315:1);
\coordinate (P8) at (359:1);
\draw (P1) -- (P2) -- (P3) -- (P4) -- (P5) -- (P6) -- (P7) -- (P8) -- cycle;
}


\def\mand#1{\scaleto{s}{4.6pt}_{\scaleto{#1}{5.2pt}}}
\def\EthreeJ{{}^{{\{a,b\}}_3} {\cal E}_8}
\def\EfourJ{{}^{{\{a,b\}}_4} {\cal E}_8}
\def\LiOneCalX#1#2{\text{Li}_1(-\mathfrak{X}_{#1,#2})}
\def\LiOneBarCalX#1#2{\text{Li}_1(-\overline{\mathfrak{X}}_{#1,#2})}

\newcommand{\cP}{{\cal P}}
\def\lr{\leftrightarrow}

\makeindex
\oddsidemargin -0.04cm \evensidemargin -0.04cm
\topmargin -0.25cm \textwidth 16.59cm \textheight 20.5cm \headheight 15pt

\begin{document}

\thispagestyle{fancyplain}
 
\fancyhf{}
 
\cfoot{\fancyplain{}{\thepage}}

\lhead{\textbf{$A_3$-constructible function counts}}

In this section we extend the subalgebra-constructibility results to the more intricate case of $A_2\subset A_3$-constructible functions.

For $A_3$ there are two possible functions, as described in a previous section: $f_{A_3}^{+-}$ and $f_{A_3}^{--}$. We will work throught the case of $f_{A_3}^{--}$ as it is known to be connected with $R^{(2)}_7$~\cite{}, however it remains an interesting open question to find applications for $f_{A_3}^{+-}$-constructible functions. 

The algorithm described in the previous section applies, changing only $A_2\to A_3$:
\begin{itemize}
	\item begin with an ansatz of all $f_{A_3}^{+-}$ applied across all $A_3$ subalgebras of a given larger algebra,
	\item impose that the overall function is invariant under automorphisms up to overall sign choices,
	\item count the number of solutions to these constraints.
\end{itemize}
To reiterate, the functions that satisfy these constraints are decomposable in to both $A_2$ and $A_3$ building blocks, thus making contact with multiple layes of the intricate cluster algebraic structure. 

There are no $A_3^{--}\subset A_4$ or $A_3^{--}\subset D_4$ functions. For $A_5$ we have 

\begin{equation}\label{eq:a3-func-counts}
\begin{array}{ l | c | c | c | c }      
  & \sigma^+\tau^+ & \sigma^+\tau^- & \sigma^-\tau^+ & \sigma^-\tau^- \\
  \hline
  A_3^{--}\subset A_5 & 1 & 1 & 0 & 3 
\end{array} 
\end{equation}
For $D_5$ we have
\begin{equation}
\begin{tabular}{| c | c |}
\multicolumn{2}{c}{$\underline{\sigma^+ \tau^+}$} \\[-1em]
\multicolumn{1}{c}{} & \multicolumn{1}{c}{} \\
\multicolumn{1}{c}{$\mathbb{Z}_2^+$} & \multicolumn{1}{c}{$\mathbb{Z}_2^-$} \\[-1em]
\multicolumn{1}{c}{} & \multicolumn{1}{c}{} \\
\hline
2 & 0 \\
\hline
\end{tabular} 
\hspace{1.2cm}
\begin{tabular}{| c | c |}
\multicolumn{2}{c}{$\underline{\sigma^+ \tau^-}$} \\[-1em]
\multicolumn{1}{c}{} & \multicolumn{1}{c}{} \\
\multicolumn{1}{c}{$\mathbb{Z}_2^+$} & \multicolumn{1}{c}{$\mathbb{Z}_2^-$} \\[-1em]
\multicolumn{1}{c}{} & \multicolumn{1}{c}{} \\
\hline
3 & 0 \\
\hline
\end{tabular} 
\hspace{1.2cm}
\begin{tabular}{| c | c |}
\multicolumn{2}{c}{$\underline{\sigma^- \tau^+}$} \\[-1em]
\multicolumn{1}{c}{} & \multicolumn{1}{c}{} \\
\multicolumn{1}{c}{$\mathbb{Z}_2^+$} & \multicolumn{1}{c}{$\mathbb{Z}_2^-$} \\[-1em]
\multicolumn{1}{c}{} & \multicolumn{1}{c}{} \\
\hline
0 & 1 \\
\hline
\end{tabular} 
\hspace{1.2cm}
\begin{tabular}{| c | c |}
\multicolumn{2}{c}{$\underline{\sigma^- \tau^-}$} \\[-1em]
\multicolumn{1}{c}{} & \multicolumn{1}{c}{} \\
\multicolumn{1}{c}{$\mathbb{Z}_2^+$} & \multicolumn{1}{c}{$\mathbb{Z}_2^-$} \\[-1em]
\multicolumn{1}{c}{} & \multicolumn{1}{c}{} \\
\hline
0 & 5 \\
\hline
\end{tabular} 
\end{equation}
And finally $E_6$ gives
\begin{equation}
\begin{tabular}{| c | c |}
\multicolumn{2}{c}{$\underline{\sigma^+ \tau^+}$} \\[-1em]
\multicolumn{1}{c}{} & \multicolumn{1}{c}{} \\
\multicolumn{1}{c}{$\mathbb{Z}_2^+$} & \multicolumn{1}{c}{$\mathbb{Z}_2^-$} \\[-1em]
\multicolumn{1}{c}{} & \multicolumn{1}{c}{} \\
\hline
- & - \\
\hline
\end{tabular} 
\hspace{1.2cm}
\begin{tabular}{| c | c |}
\multicolumn{2}{c}{$\underline{\sigma^+ \tau^-}$} \\[-1em]
\multicolumn{1}{c}{} & \multicolumn{1}{c}{} \\
\multicolumn{1}{c}{$\mathbb{Z}_2^+$} & \multicolumn{1}{c}{$\mathbb{Z}_2^-$} \\[-1em]
\multicolumn{1}{c}{} & \multicolumn{1}{c}{} \\
\hline
- & - \\
\hline
\end{tabular} 
\hspace{1.2cm}
\begin{tabular}{| c | c |}
\multicolumn{2}{c}{$\underline{\sigma^- \tau^+}$} \\[-1em]
\multicolumn{1}{c}{} & \multicolumn{1}{c}{} \\
\multicolumn{1}{c}{$\mathbb{Z}_2^+$} & \multicolumn{1}{c}{$\mathbb{Z}_2^-$} \\[-1em]
\multicolumn{1}{c}{} & \multicolumn{1}{c}{} \\
\hline
- & - \\
\hline
\end{tabular} 
\hspace{1.2cm}
\begin{tabular}{| c | c |}
\multicolumn{2}{c}{$\underline{\sigma^- \tau^-}$} \\[-1em]
\multicolumn{1}{c}{} & \multicolumn{1}{c}{} \\
\multicolumn{1}{c}{$\mathbb{Z}_2^+$} & \multicolumn{1}{c}{$\mathbb{Z}_2^-$} \\[-1em]
\multicolumn{1}{c}{} & \multicolumn{1}{c}{} \\
\hline
- & - \\
\hline
\end{tabular} 
\end{equation}
\end{document}
