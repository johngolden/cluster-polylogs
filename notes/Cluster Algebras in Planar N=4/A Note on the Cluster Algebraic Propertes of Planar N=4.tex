
\pdfoutput=1
\documentclass[11pt, reqno,preprint]{article}
\usepackage{jheppub}
\usepackage{epsfig}
\usepackage{amssymb}
\usepackage{amsmath}
\usepackage{mathrsfs}
%\usepackage{cite}
\usepackage{hyperref}
\usepackage{multirow}
\usepackage{feynmp-auto}

\def\be{\begin{equation}}
\def\ee{\end{equation}}
\def\ba{\begin{eqnarray}}
\def\ea{\end{eqnarray}}
\newcommand{\bea}{\begin{eqnarray}}
\newcommand{\eea}{\end{eqnarray}}
\def\nl{\nonumber\\}
\def\Li{\textrm{Li}}
\def\l{\langle}
\def\r{\rangle}
\def\eps{\epsilon}
\def\uo{\underline{0}}
\def\uone{\underline{1}}
\def\ut{\underline{t}}
\def\z{x}
\def\zb{y}
\newcommand{\del}{\partial}
\def\fig#1{fig.~{\ref{#1}}}
\def\Fig#1{Fig.~{\ref{#1}}}
\def\figs#1#2{figs.~{\ref{#1}} and {\ref{#2}}}
\def\Figs#1#2{Figs.~{\ref{#1}} and {\ref{#2}}}
\def\Sect#1{Section~{\ref{#1}}}
\def\sect#1{section~{\ref{#1}}}
\def\eqn#1{eq.~(\ref{#1})}
\def\Eqn#1{Equation~(\ref{#1})}
\def\eqns#1#2{eqs.~(\ref{#1}) and~(\ref{#2})}
\def\Eqns#1#2{Eqs.~(\ref{#1}) and~(\ref{#2})}
\def\tab#1{table~{\ref{#1}}}
\def\Tab#1{Table~{\ref{#1}}}
\def\tabs#1#2{table~{\ref{#1}} and~{\ref{#1}}}
\def\Tabs#1#2{Table~{\ref{#1}} and~{\ref{#1}}}
\def\Eqn#1{Equation~(\ref{#1})}
\def\eqn#1{eq.~(\ref{#1})}
\def\eqns#1#2{eqs.~(\ref{#1}) and~(\ref{#2})}
\def\Eqns#1#2{Eqs.~(\ref{#1}) and~(\ref{#2})}

\newcommand{\cP}{{\cal P}}
\def\lr{\leftrightarrow}

\def\draftnote#1{{\bf [#1]}}


\def\blue#1{{\color{blue}#1}}

\title{A Note on the Cluster-Algebraic Properties of Maximally Supersymmetric Yang-Mills Theory}

\author{John~Golden,$^{1,2}$}
\author{Andrew~J.~McLeod$^{2,3,4}$}


\affiliation{$^1$ Michigan Center for Theoretical Physics and
Randall Laboratory of Physics, Department of Physics,
University of Michigan
Ann Arbor, MI 48109, USA}

\affiliation{$^2$ Kavli Institute for Theoretical Physics, 
UC Santa Barbara, Santa Barbara, CA 93106, USA}

\affiliation{$^3$ SLAC National Accelerator Laboratory,
Stanford University, Stanford, CA 94309, USA}

\affiliation{$^4$ Niels Bohr International Academy, Blegdamsvej 17, 2100 Copenhagen, Denmark}

\abstract{Cluster algebras appear in amplitudes in the planar limit of maximally supersymmetric gauge theory in a number of surprising ways. In this note, we explore how far these cluster-algebraic properties reach and what types of physical information they encode.}

%\preprint{
%\begin{flushright} DESY ??--??? \\ SLAC--PUB--?????
%\end{flushright}
%}




\begin{document}
\hypersetup{pageanchor=false}
\maketitle
\hypersetup{pageanchor=true}
\begin{fmffile}{feyndiags}


\section{Introduction}

Make contact with how cluster algebras appear in the integrand (positive grassmannian)?

\section{Steinmann Relations and the Cobracket}

Talk about how the information encoded in the Steinmann relations and the information encoded in the cobracket of the amplitude appear to be completely orthogonal (at least at six points---need to check more generally). In particular, note that $\rho(R_n) = \rho({\cal E}_n)$, where ${\cal E}_n$ is any BDS-like normalized MHV amplitude, since these objects differ from the remainder function only by products of lower-weight functions. However, this projection operator maximally scrambles the Steinmann relations by 'subtracting off' products of functions, which don't obey Steinmann. Thus, the nonclassical part of the amplitude doesn't know anything about the Steinmann relations and must be encoding something else. But it also means the Steinmann relations provide nice restrictions on the space of symbols that a cobracket-level object should be lifted to.


\section{Cluster Polylogarithms of Larger Subalgebras}

We can here tabulate all the subalgebras we were interested in checking. It doesn't appear that any interesting functions can be constructed out of $A_2$ or $A_3$ building blocks for any larger subalgebras of $E_6$....


\section{The Six-Particle Amplitude}

\subsection{The MHV Amplitude at Two Loops}

While the $A_2$ function studied in~\cite{Golden:2014xqa} provides a natural building block for the non-classical part of the two-loop MHV amplitude for more than six particles, there is no way to choose its classical part that gets the six-particle amplitude (which has no non-classical part) correct. Namely, taking the totally dihedrally symmetric and parity even combination of a general ansatz for the classical completions of $A_2$ returns zero.


\subsection{Two-Loop Ratio Function}

\subsection{The Three-Loop}

\section{The Seven-Particle Amplitude}

\bibliographystyle{ieeetr}

\bibliography{cluster_algebra_note}

\end{fmffile}
\end{document}
