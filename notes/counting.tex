\pdfoutput=1


\documentclass[12pt]{article}

\usepackage{amssymb}
\usepackage{amsfonts}
\usepackage{amsmath}
\usepackage{parskip}
\usepackage{fancyhdr}
\usepackage{multirow}

\DeclareMathOperator{\B}{B}
\DeclareMathOperator{\Conf}{Conf}
\DeclareMathOperator{\Gr}{Gr}
\DeclareMathOperator{\Id}{Id}
\DeclareMathOperator{\Li}{Li}
\DeclareMathOperator{\Lie}{Lie}

\def\ket#1{\langle #1 \rangle}


\makeindex
\oddsidemargin -0.04cm \evensidemargin -0.04cm
\topmargin -0.25cm \textwidth 16.59cm \textheight 20.5cm \headheight 15pt

\setlength{\parindent}{0pt}

\begin{document}

\thispagestyle{fancyplain}

\fancyhf{}

\cfoot{\fancyplain{}{\thepage}}

\lhead{\textbf{Cluster Algebra Counting} \hfill \today}

In this note we catalog the subalgebra structure for the finite cluster algebras \(\subseteq E_6\).

These algebras are: \(A_2, A_3, A_4, D_4, A_5, D_5, E_6\).\\ \\ 

{\Huge\underline{\(A_2\)}} \quad clusters: 5 \qquad $a$-coordinates: 5 \qquad $x$-coordinates: 10 \\ \\ \\


{\Huge\underline{\(A_3\)}} \quad clusters: 14 \qquad \(a\)-coordinates: 9 \qquad \(x\)-coordinates: 30 \\

\begin{tabular}{ | l | l | l |}
\multicolumn{1}{c}{Type} &  \multicolumn{1}{c}{Sub-polytopes}  &  \multicolumn{1}{c}{Distinct Subalgebras} \\
\hline \(A_2\) & 6 & 6 \\ 
\hline \(A_1 \times A_1\) & 3 & 3 \\ 
\hline
\end{tabular} \\ \\ \\ 


{\Huge \underline{\(A_4\)}} \quad clusters: 42 \qquad \(a\)-coordinates: 14 \qquad \(x\)-coordinates: 70\\ 

\begin{tabular}{ | l | l | l |}
\multicolumn{1}{c}{Type} &  \multicolumn{1}{c}{Sub-polytopes}  &  \multicolumn{1}{c}{Distinct Subalgebras} \\
\hline \(A_2\) & 28 & 21 \\ 
\hline \(A_1 \times A_1\) & 28 & 28 \\ \hline 
\hline \(A_3\) & 7 & 7 \\ 
\hline \(A_2 \times A_1\) & 7 & 7 \\ 
\hline \(A_1 \times A_1 \times A_1\) & 0 & 0 \\ 
\hline
\end{tabular} \\ \\ \\ 

{\Huge\underline{\(D_4\)}} \quad clusters: 50 \qquad \(a\)-coordinates: 16 \qquad \(x\)-coordinates: 104\\

\begin{tabular}{ | l | l | l |}
\multicolumn{1}{c}{Type} &  \multicolumn{1}{c}{Sub-polytopes}  &  \multicolumn{1}{c}{Distinct Subalgebras} \\
\hline \(A_2\) & 36 & 36 \\ 
\hline \(A_1 \times A_1\) & 30 & 18 \\ \hline 
\hline \(A_3\) & 12 & 12 \\ 
\hline \(A_2 \times A_1\) & 0 & 0 \\ 
\hline \(A_1 \times A_1 \times A_1\) & 4 & 4 \\ 
\hline
\end{tabular} 

{\Huge\underline{\(A_5\)}} \quad clusters: 132 \qquad \(a\)-coordinates: 20 \qquad \(x\)-coordinates: 140\\

\begin{tabular}{ | l | l | l |}
\multicolumn{1}{c}{Type} &  \multicolumn{1}{c}{Sub-polytopes}  &  \multicolumn{1}{c}{Distinct Subalgebras} \\
\hline \(A_2\) & 120 & 56 \\ 
\hline \(A_1 \times A_1\) & 180 & 144 \\ \hline 
\hline \(A_3\) & 36 & 28 \\ 
\hline \(A_2 \times A_1\) & 72 & 72 \\ 
\hline \(A_1 \times A_1 \times A_1\) & 12 & 12 \\ \hline 
\hline \(D_4\) & 0 & 0 \\ 
\hline \(A_4\) & 8 & 8 \\ 
\hline \(A_3 \times A_1\) & 8 & 8 \\ 
\hline \(A_2 \times A_2\) & 4 & 4 \\ 
\hline \(A_2 \times A_1 \times A_1\) & 0 & 0 \\ 
\hline \(A_1 \times A_1 \times A_1 \times A_1\) & 0 & 0 \\ 
\hline
\end{tabular} \\ \\ \\ \\ \\


{\Huge\underline{\(D_5\)}} \quad clusters: 182 \qquad \(a\)-coordinates: 25 \qquad \(x\)-coordinates: 260\\

\begin{tabular}{ | l | l | l |}
\multicolumn{1}{c}{Type} &  \multicolumn{1}{c}{Sub-polytopes}  &  \multicolumn{1}{c}{Distinct Subalgebras} \\
\hline \(A_2\) & 180 & 125 \\ 
\hline \(A_1 \times A_1\) & 230 & 145 \\ \hline 
\hline \(A_3\) & 70 & 65 \\ 
\hline \(A_2 \times A_1\) & 60 & 50 \\ 
\hline \(A_1 \times A_1 \times A_1\) & 30 & 30 \\ \hline 
\hline \(D_4\) & 5 & 5 \\ 
\hline \(A_4\) & 10 & 10 \\ 
\hline \(A_3 \times A_1\) & 5 & 5 \\ 
\hline \(A_2 \times A_2\) & 0 & 0 \\ 
\hline \(A_2 \times A_1 \times A_1\) & 5 & 5 \\ 
\hline \(A_1 \times A_1 \times A_1 \times A_1\) & 0 & 0 \\ 
\hline
\end{tabular} \\ \\ \\

\pagebreak

{\Huge\underline{\(E_6\)}} \quad clusters: 833 \qquad \(a\)-coordinates: 42 \qquad \(x\)-coordinates: 770\\

\begin{tabular}{ | l | l | l |}
\multicolumn{1}{c}{Type} &  \multicolumn{1}{c}{Sub-polytopes}  &  \multicolumn{1}{c}{Distinct Subalgebras} \\
\hline \(A_2\) & 1071 & 504 \\ 
\hline \(A_1 \times A_1\) & 1785 & 833 \\ \hline 
\hline \(A_3\) & 476 & 364 \\ 
\hline \(A_2 \times A_1\) & 714 & 490 \\ 
\hline \(A_1 \times A_1 \times A_1\) & 357 & 357 \\ \hline 
\hline \(D_4\) & 35 & 35 \\ 
\hline \(A_4\) & 112 & 98 \\ 
\hline \(A_3 \times A_1\) & 112 & 112 \\ 
\hline \(A_2 \times A_2\) & 21 & 14 \\ 
\hline \(A_2 \times A_1 \times A_1\) & 119 & 119 \\ 
\hline \(A_1 \times A_1 \times A_1 \times A_1\) & 0 & 0 \\ \hline 
\hline \(D_5\) & 14 & 14 \\ 
\hline \(A_5\) & 7 & 7 \\ 
\hline \(D_4 \times A_1\) & 0 & 0 \\ 
\hline \(A_4 \times A_1\) & 14 & 14 \\ 
\hline \(A_3 \times A_2\) & 0 & 0 \\ 
\hline \(A_3 \times A_1 \times A_1\) & 0 & 0 \\ 
\hline \(A_2 \times A_2 \times A_1\) & 7 & 7 \\ 
\hline \(A_2 \times A_1 \times A_1 \times A_1\) & 0 & 0 \\ 
\hline \(A_1 \times A_1 \times A_1 \times A_1 \times A_1\) & 0 & 0 \\ 
\hline
\end{tabular} \\ \\ \\


\newpage
\subsection*{Symbol Spaces on Cluster Algebras}

The number of weight-two symbols defined on the finite cluster algebras. 

\begin{tabular}{ | c | c | c | c | c | c |  c |}
\multicolumn{1}{c}{\multirow{2}{*}{Type}} & \multicolumn{1}{c}{\multirow{2}{*}{Integrable}} & \multicolumn{1}{c}{\multirow{2}{*}{Cluster-Adjacent}} & \multicolumn{4}{c}{Automorphic} \\
\multicolumn{1}{c}{}                                  & \multicolumn{1}{c}{}                                          & \multicolumn{1}{c}{}                                                     & \multicolumn{1}{c}{$\sigma^+ \tau^+$} & \multicolumn{1}{c}{$\sigma^+ \tau^-$} & \multicolumn{1}{c}{$\sigma^- \tau^+$} & \multicolumn{1}{c}{$\sigma^- \tau^-$} \\
\hline \(A_2\) & 19 & 14 & 2 & 0 & 0 & 0 \\ 
\hline \(A_3\) & 55 & 40 & 6 & 2 & 1 & 5 \\ 
\hline \(A_4\) & 125 & 90 & 9 & 3 & 0 & 0\\ 
\hline \(D_4\) & 163 & 109 & $\mid$ & $\mid$ & $\mid$ & $\mid$  \\ 
\hline \(A_5\) & 245 & 175 & 1 & 0 & 0 & 0 \\ 
\hline \(D_5\) & 381 & 241 & $\mid$ & $\mid$ & $\mid$ & $\mid$  \\ 
\hline \(E_6\) & - & 573 & - & - & - & -  \\ 
\hline
\end{tabular} \\ \\ \\ 
 
The number of weight-four symbols defined on the finite cluster algebras. 

\begin{tabular}{ | c | c | c | c | c | c |  c |}
\multicolumn{1}{c}{\multirow{2}{*}{Type}} & \multicolumn{1}{c}{\multirow{2}{*}{Integrable}} & \multicolumn{1}{c}{\multirow{2}{*}{Cluster-Adjacent}} & \multicolumn{4}{c}{Automorphic} \\
\multicolumn{1}{c}{} & \multicolumn{1}{c}{} & \multicolumn{1}{c}{} & \multicolumn{1}{c}{$\sigma^+ \tau^+$} & \multicolumn{1}{c}{$\sigma^+ \tau^-$} & \multicolumn{1}{c}{$\sigma^- \tau^+$} & \multicolumn{1}{c}{$\sigma^- \tau^-$} \\
\hline \(A_2\) & 211 & 81 & 11 & 6 & 0 & 0 \\ 
\hline \(A_3\) & 1351 & 432 & 49 & 29 & 24 & 42 \\ 
\hline \(A_4\) & - & 1652 & - & - & - & -\\ 
\hline \(D_4\) & - & - & $\mid$ & $\mid$ & $\mid$ & $\mid$  \\ 
\hline \(A_5\) & - & - & - & - & - & - \\ 
\hline \(D_5\) & - & - & $\mid$ & $\mid$ & $\mid$ & $\mid$  \\ 
\hline \(E_6\) & - & - & - & - & - & -  \\ 
\hline
\end{tabular} \\ \\ \\ 


\pagebreak
\subsection*{How to count co-dimension 1 \& 2 subalgebras}

This algorithm comes from Hugh Thomas.

Our goal is to count the number of cluster algebras of type \(Y\) of rank \(n-1\) in a finite type cluster algebra of type \(X\) of rank \(n\).

Let \(N\) be the number of cluster \(a\)-coordinates in the cluster algebra of type \(X\).\\
Let \(n\) be the rank of cluster algebra of type \(X\).\\
Let \(Z\) be the number of ways to remove a node from a Dynkin diagram of type \(X\) to obtain one of type \(Y\).

The number of cluster algebras of type \(Y\) is then \(NZ/n\).

As an example, let's count the number of \(A_4\)'s in \(A_5\):

\(N\) = \# of cluster \(a\)-coordinates in \(A_5 = 20\)\\
\(n\) = rank of \(A_5 = 5\)\\
\(Z\) = \# of ways to remove a node from \(A_5\) and get an \(A_4 = 2\)

\(20\times2/5 = 8 \Rightarrow\) there are 8 \(A_4\)'s in \(A_5\).


\end{document}
