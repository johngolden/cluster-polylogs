\pdfoutput=1


\documentclass[12pt]{article}

\usepackage{amssymb}
\usepackage{amsfonts}
\usepackage{amsmath}
\usepackage{fancyhdr} 

\DeclareMathOperator{\B}{B}
\DeclareMathOperator{\Conf}{Conf}
\DeclareMathOperator{\Gr}{Gr}
\DeclareMathOperator{\Id}{Id}
\DeclareMathOperator{\Li}{Li}
\DeclareMathOperator{\Lie}{Lie}

\def\ket#1{\langle #1 \rangle}
\def\a{\mathcal{A}}
\def\x{\mathcal{X}}

\makeindex
\oddsidemargin -0.04cm \evensidemargin -0.04cm
\topmargin -0.25cm \textwidth 16.59cm \textheight 20.5cm \headheight 15pt

\begin{document}

\thispagestyle{fancyplain}
 
\fancyhf{}
 
\cfoot{\fancyplain{}{\thepage}}

\lhead{\textbf{$A_2$-adjacent functions} \hfill \today}

\noindent In this note we describe the space of weight-4 polylogarithm functions which depend on $\mathcal{X}$-coordinates of the $A_2$ cluster algebra. Taking our seed cluster as $x_1\to x_2$, we define our $\x$-coordinates:\\
\begin{equation}\label{def:xcoords}
	\x_1 = \frac{1}{x_1}, ~~~\x_2 = x_2,~~~ \x_3 = x_1 (1 + x_2),~~~ \x_4=\frac{1+x_1+x_1 x_2}{x_2},~~~ \x_5 = \frac{1+x_1}{x_1 x_2}.
\end{equation}\\
These satisfy $1+\x_i = \x_{i-1}\x_{i+1}$, and $\{1/\x_i, \x_{i+1}\}$ form the 5 clusters. Of course you can also work in the $\Gr(2,5)$ language, where we have the $\a$-coordinates $\ket{ij}$ and the $\x$-coordinates\\
\begin{equation}
	\x_1 = \frac{\ket{14}\ket{23}}{\ket{12}\ket{34}}, ~~~\x_2 = \frac{\ket{13}\ket{45}}{\ket{15}\ket{34}},~~~ \x_3 = \frac{\ket{12}\ket{35}}{\ket{15}\ket{23}},~~~ \x_4=\frac{\ket{25}\ket{34}}{\ket{23}\ket{45}},~~~ \x_5 = \frac{\ket{15}\ket{24}}{\ket{12}\ket{45}}.
\end{equation}\\
This makes it easy to define cluster $\a$-adjacency: the allowed pairs are $\{\ket{13},\ket{14}\}$, $\{\ket{14},\ket{24}\}$, $\{\ket{24},\ket{25}\}$, $\{\ket{25},\ket{35}\}$, $\{\ket{35},\ket{13}\}$ (along with all of the frozen nodes, $\ket{i~i+1}$). 

\subsubsection*{Integrable Symbol}		
Imposing integrability on a generic symbol of weight 4 with arguments drawn from (\ref{def:xcoords}) leaves us with 211 free parameters. 

\subsubsection*{$\a$-adjacency vs. $\x$-adjacency}
The two criteria we consider are:
\begin{itemize}
	\item cluster $\a$-adjacency (only $\{\ket{i~i+2},\ket{i~i+3}\}$+frozen can appear),
	\item cluster $\x$-adjacency (only $\{\x_i,\x_{i+1}\}$ can appear).
\end{itemize}
For weight-4 integrable symbols on $A_2$, these criteria turn out to be equivalent (this is not surprising since there are an equal number of $\a$- and $\x$-coordinates). In both cases, 130 parameters are fit by cluster adjacency, leaving us with 81.

\subsubsection*{Imposing cluster-y coproduct}
Now that we have a symbol which is cluster adjacent, let us also impose that the coproduct be cluster-y as well. Based on experience we know that imposing full cluster adjacency at the level of the coproduct is not possible for $A_2$ (we have to wait for $A_3$ for that property), but we can at least impose that the $B_2\wedge B_2$ and $B_3 \otimes \mathbb{C}^*$ take only $\{\x\}_k$ as arguments. \\

\noindent Interestingly, cluster adjacency at symbol level already imposes a cluster-y coproduct at the level of $B_2\wedge B_2$. Imposing this for $B_3 \otimes \mathbb{C}^*$ fixes 15 parameters (I think this is killing off terms like $\Li_4(-1-\x)$ and $\Li_4(-1-1/\x)$), leaving us with 66.

\subsubsection*{Imposing symmetries}

We now impose that the function respect the automorphisms of $A_2$. $A_2$ has a cyclic symmetry, $\sigma: \x_i\to \x_{i+1}$, and a flip symmetry $\tau: \x_i \to \x_{6-i}$. For each symmetry, we consider the case where $f_{A_2}$ is either invariant or ``covariant'' -- i.e. $\sigma(f_{A_2}) = f_{A_2}$ or $\sigma(f_{A_2}) = -f_{A_2}$. We will now tabulate how many free parameters remain for each of the 4 possible sign choices:
\begin{center}
\begin{tabular}{ rcccc }
\multicolumn{1}{r}{}
 & \multicolumn{1}{c}{$\sigma^+\tau^+$}
 & \multicolumn{1}{c}{$\sigma^+\tau^-$}
 & \multicolumn{1}{c}{$\sigma^-\tau^+$}
 & \multicolumn{1}{c}{$\sigma^-\tau^-$} \\
\cline{2-5}
$f_{A_2}$: & 9 & 5 & 0 & 0 \\
\end{tabular}
\end{center}
We'll refer to these functions by their behavior under $\sigma$ and $\tau$: $f_{A_2}^{++}$ and $f_{A_2}^{+-}$. Note that only $f_{A_2}^{+-}$ has non-zero $B_2\wedge B_2$ -- this is what we have traditionally called ``the $A_2$ function'' (at coproduct level).

\subsubsection*{Poisson Structure and Mutation Interpretation}
It would be nice to have an understanding of the Poisson structure in adjacent symbol entries for these functions -- in particular, can we fit the remaining parameters in order to bring out a nice Poisson structure? A dream scenario would be to have an interpretation of these symbol terms as a mutation sequence. \\

\noindent Unfortunately at this point I don't seen any obvious interpretation, but this is worth returning to!

\subsubsection*{Summary}

We have determined two weight-4 functions of interested related to the $A_2$ cluster algebra: $f_{A_2}^{++}$ and $f_{A_2}^{+-}$. They have 9,5 free parameters (respectively), and $f_{A_2}^{+-}$ has non-zero $B_2\wedge B_2$. The next step is to evaluate these functions across $A_3$, impose the $++$ automorphism sign choice, and see how close we get to $\mathcal{E}^{(2)}_6$.

\end{document}
