\pdfoutput=1


\documentclass[12pt]{article}

\usepackage{amssymb}
\usepackage{amsfonts}
\usepackage{amsmath}
\usepackage{fancyhdr} 
\usepackage{dsfont}
\usepackage[all]{xy}


\DeclareMathOperator{\B}{B}
\DeclareMathOperator{\Conf}{Conf}
\DeclareMathOperator{\Gr}{Gr}
\DeclareMathOperator{\Id}{Id}
\DeclareMathOperator{\Li}{Li}
\DeclareMathOperator{\Lie}{Lie}

\def\ket#1{\langle #1 \rangle}
\def\acoord{$\mathcal{A}$-coordinate}
\def\acoords{$\mathcal{A}$-coordinates}
\def\a{\mathcal{A}}

\newcommand{\unit}{1\!\!1}

\makeindex
\oddsidemargin -0.04cm \evensidemargin -0.04cm
\topmargin -0.25cm \textwidth 16.59cm \textheight 20.5cm \headheight 15pt

\begin{document}

\thispagestyle{fancyplain}
 
\fancyhf{}
 
\cfoot{\fancyplain{}{\thepage}}

\lhead{\textbf{Poisson Bracket for \acoords} \hfill \today}

The basic idea for generating a well-defined Poisson bracket on \acoords is to start with some seed (in terms of \acoords) with standard adjacency matrix $B$ written as a $n \times k$ matrix (this the case when you have $n$ total nodes with $k$ mutable). Then find some skew-symmetric $n \times n$ matrix $\Omega$ such that 
\begin{equation}\label{eq:omega-def}
	\Omega B = \left(\begin{array}{c}
					\mathds{1}_{k\times k}\\0_{(n-k)\times k}
				\end{array}\right)
\end{equation}
Then the claim is that the entries $\omega_{ij}$ form a Poisson bracket on \acoords, i.e. 
\begin{equation}
	\{\a_i,\a_j\} = \omega_{ij}\a_i\a_j \Rightarrow \{\bar{\a_i},\bar{\a_j}\} = \bar{\omega_{ij}}\bar{\a_i}\bar{\a_j}
\end{equation}
where $\bar{}$ indicates mutation. The interesting feature here is that imposing eq.~(\ref{eq:omega-def}) does not entirely constrain $\Omega$. For example, let's take a seed from $\Gr(2,5)$:
\begin{equation}\label{eq:gr25-seed}
\begin{gathered}
\begin{xy} 0;<1pt,0pt>:<0pt,-1pt>::
	(25,25) *+{\langle 13\rangle} ="0",
	(75,25) *+{\langle 14\rangle} ="1",
	(125,25) *+{\framebox[5ex]{$\langle 15\rangle$}} ="2",
	(125,75) *+{\framebox[5ex]{$\langle 45\rangle$}} ="3",
	(75,75) *+{\framebox[5ex]{$\langle 34\rangle$}} ="4",
	(25,75) *+{\framebox[5ex]{$\langle 23\rangle$}} ="5",
	(0,0) *+{\framebox[5ex]{$\langle 12\rangle$}} ="6",
	(145,75) *+{},
	"0", {\ar"1"},
	"4", {\ar"0"},
	"0", {\ar"5"},
	"6", {\ar"0"},
	"1", {\ar"2"},
	"3", {\ar"1"},
	"1", {\ar"4"},
\end{xy}
\end{gathered}
\end{equation}
Associating $\ket{13}$ and $\ket{14}$ with nodes 1 and 2, resp., and then $\ket{12},\ldots,\ket{15}$ with nodes $3,\ldots,7$ we have
\begin{equation}
	B = \left(
\begin{array}{cc}
 0 & 1 \\
 -1 & 0 \\
 1 & 0 \\
 -1 & 0 \\
 1 & -1 \\
 0 & 0 \\
 0 & -1 \\
\end{array}
\right)
\end{equation}
in which case $\Omega$ takes the form 
{\tiny
\begin{equation*}\hspace{-2cm}
	\Omega = \left(
\begin{array}{ccccccc}
 0 & c_{13}-c_{14}+c_{15}-1 & c_{13} & c_{14} & c_{15} & c_{16} & -c_{15} \\
 -c_{13}+c_{14}-c_{15}+1 & 0 & c_{23} & c_{24} & c_{24}-c_{23} & c_{26} & -c_{13}+c_{14}-c_{15}+c_{23}-c_{24} \\
 -c_{13} & -c_{23} & 0 & c_{34} & c_{34}-c_{23} & c_{36} & -c_{13}+c_{23}-c_{34} \\
 -c_{14} & -c_{24} & -c_{34} & 0 & c_{34}-c_{24} & c_{46} & -c_{14}+c_{24}-c_{34} \\
 -c_{15} & c_{23}-c_{24} & c_{23}-c_{34} & c_{24}-c_{34} & 0 & c_{26}-c_{36}+c_{46} & -c_{15} \\
 -c_{16} & -c_{26} & -c_{36} & -c_{46} & -c_{26}+c_{36}-c_{46} & 0 & -c_{16}+c_{26}-c_{36}+c_{46} \\
 c_{15} & c_{13}-c_{14}+c_{15}-c_{23}+c_{24} & c_{13}-c_{23}+c_{34} & c_{14}-c_{24}+c_{34} & c_{15} & c_{16}-c_{26}+c_{36}-c_{46} & 0 
\end{array}
\right)
\end{equation*}}
The $c_{ij}$ are arbitrary, and I guess we really only care about the $2\times2$ entries in the upper left corner but I present the full matrix for completion's sake. 

Some considerable confusion remains, at least on my part. In particular, what mutation rule does one use to generate mutated $\Omega$'s? One can just follow the standard $B$-matrix mutation rules, although it becomes tricky because the $B$-matrix rules require checking the sign of elements, which is difficult if you want to keep the $c_{ij}$ generic. Furthermore, how does this relate to the Sklyanin bracket calculations done on \acoords? If one chooses to mutate $\Omega$ via the $B$-matrix rules, then the Sklyanin bracket (as we know it) is not compatible with any choice of $c_{ij}$. 

The best case scenario here is to find some Poisson bracket on the $\a$'s which makes manifest some particularly nice adjacency structure in the (Steinmann) remainder function. Perhaps this is equivalent to finding some representation of the symbol in terms of $\mathcal{X}$-coordinates where again the Poisson bracket between adjacent pairs follows some nice pattern. 
\end{document}
