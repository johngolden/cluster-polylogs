\documentclass[12pt]{article}

\usepackage{amssymb}
\usepackage{amsfonts}
\usepackage{amsmath}
\usepackage{fancyhdr}
\usepackage{parskip} 
\usepackage{hyperref}

\DeclareMathOperator{\Li}{Li}
\DeclareMathOperator{\Gr}{Gr}

\newcommand*\checked{\item[\checkmark]}

\oddsidemargin -0.04cm 
\evensidemargin -0.04cm
\topmargin -0.25cm 
\textwidth 16.59cm 
\textheight 20.5cm

\begin{document}

\thispagestyle{fancyplain}
\fancyhf{}

\lhead{\textbf{Cluster Polylogarithms Gameplan} \hfill \today}

8-pt BDS-like amplitudes
\begin{itemize}
	\item analyze the space of BDS-like ansatze (done at symbol-level)
	\item check if cluster adjacency breaks down in one-to-one correspondence with broken Steinmann  relations 
	\item get a cluster-polylogarithmic representation of $R^{(2)}_8$ at function level 
	\item see if we can put the conversion from $R^{(2)}_8$ to the various BDS-like anstaze in a cluster-polylog form---if not, find a cluster-polylog form for these amplitudes as well
	\item reach goal: go to higher points, or find significant improvements to the published algorithm (especially taking advantage of Steinmann/cluster adjacency)\\
\end{itemize}

Cobracket-level subalgebra constructibility
\begin{itemize}
	\item look for more subalgebra-constructible representations of $B_2\wedge B_2$ or $B_3\otimes C^*$ at 6 and 7 points
	\item find a principle for selecting only a 'good' set of of subalgebras out of $\Gr(4,8)$, and find a subalgebra-constructible representation in 8-point kinematics \\
\end{itemize}

Symbol-level subalgebra constructibility
\begin{itemize}
	\item comprehensive analysis of cluster-algebraic symbol representations of amplitudes
	\item in what ways is this a useful/productive way to think about the function space relevant for n-point planar $\mathcal{N}=4$ amplitudes?
	\item are physical functions always in the non-subalgebra-constructible part? Is there a way to benefit from this computationally? 
	\item lots and lots and lots of tables\\
\end{itemize}

Reach goal: a similar type of analysis to the last two papers, but at three loops
\begin{itemize}
	\item In particular, in 1512.07910 Tom and Mark point out that $B_3\wedge B_3$ of $R^{(3)}_6$ isn't expressible in terms of $\mathcal{X}$-coordinates (probably you make this point in one of your papers as well). Any bets on whether or not the BDS-like normalized amplitude can be expressed this way ;)?\\
\end{itemize}


Cluster automorphisms
\begin{itemize}
	\item Write note on cluster automorphisms, $D_5$ function, and how $D_4$ and $A_4$ fail to have relevant cluster functions.

	\item Understand more deeply what sign choices to take for cluster automorphisms.

	\item Fix remaining free parameters in $D_5$ function.\\
\end{itemize}

\end{document}
