\pdfoutput=1


\documentclass[12pt]{article}

\usepackage{amssymb}
\usepackage{amsfonts}
\usepackage{amsmath}
\usepackage{fancyhdr} 

\DeclareMathOperator{\B}{B}
\DeclareMathOperator{\Conf}{Conf}
\DeclareMathOperator{\Gr}{Gr}
\DeclareMathOperator{\Id}{Id}
\DeclareMathOperator{\Li}{Li}
\DeclareMathOperator{\Lie}{Lie}

\def\ket#1{\langle #1 \rangle}

\oddsidemargin -0.04cm 
\evensidemargin -0.04cm
\topmargin -0.25cm 
\textwidth 16.59cm 
\textheight 20.5cm 
\headheight 15pt

\begin{document}

\thispagestyle{fancyplain}
 
\fancyhf{}
 
\cfoot{\fancyplain{}{\thepage}}

\lhead{\textbf{Questions and Comments from Sasha Goncharov}}

With comments from Marcus Spradlin (MS) and Miguel Paulos (MP).

\begin{enumerate}

\item The motivic function $L_{2,2}(x,y)$ coincides with the
motivic  function $\kappa(x,y)$ introduced in formula 4.5 in
``Polylogarithms and motivic Galois group" paper modulo $\Li_4$'s, up to
$\Li_4(-y/x)$ term.

Is it better to have $\Li_4(-y/x)$ summand?

\item What geometry responsible for linear relations between $A_2$ and $A_3$ functions?

\item So we have a ``function"

pentagons on the cluster polytope $\longmapsto B_2 \wedge B_2$

Let us try to find ANY linear combinations of pentagons on the
cluster polytope such that the resulting $B_2 \wedge B_2$ function vanishes.

Any such linear combination is of huge interest.

For example -- what is the $B_2 \wedge B_2$ component of the sum of pentagon functions related to $A_2 \times A_1$ -- this must be trivial, but yet ...

Then: we end up with a ``function" which assigns
to 3d faces of the cluster polyhedron these $B_2 \wedge B_2$ invariants.

\item Observe that cluster polytope for 7 points is a polytope, that is
topologically trivial.

Take the sum of (geometric) pentagons of the cluster poytope
which make the 7-point amplitude.

Take its ``differential", that is the sum of their boundary 1-segments.

It follows that the $B_2\wedge B_2$ content of the amplitude
depends only on this chain.

Q: what is this 1-chain?

This mean: are there pentagons in the formula which
share a common side with different orientations? I guess plenty, so
is there any simpler way to describe the 1-dim boundary then just
saying it is a boundary
of those pentagons? 

Before we described it as some ``Poisson commuting" pairs, that is
rectangles in the cluster polytope.

(This might be silly: taking the boundary of the sum of
pentagons we get the naked $B_2\wedge B_2$ term:

$\{x\}_2 \wedge \{y\}_2$ means the edge $(x,y)$.)

\item What does one get if you take alternating sum of
7 6-point functions, i.e. start from 7 points on $P^1$, forget them one
by one, and take the alternating sum of the obtained 6 point amplitudes?

Although it is clear that pentagon function is a
``primary" or ``atomic" object of the story,
it is not 100\% clear to me what are the ``quarks"
in which even the 6 point amplitude is decomposed.

So far it looks that these ``quarks" are the 6 pentagons,

Notice however that the boundary of the 3d Stasheff polytope has
not only 6 pentagons, but also 3 squares, and these squares
sort of match the 2+2+2 terms in the 9 term formula.

\item \underline{Comment from MS}

You often speculated that some role might be played by functions whose $B_2 \wedge B_2$ component *is*
the Poisson bracket, as in
$$
\sum_{i \in~\text{clusters}} \qquad \sum_{\text{coordinates}~x, y~\text{in cluster}~i} \{x,y\} {x}_2
\wedge {y}_2
$$
where $\{x,y\}$ is the Poisson bracket.  We have not found a role for
things like this to play in the amplitude story, but it might be
interesting to note that the $A_2$ and $A_3$ cluster functions seem to have
$B_3 \otimes C^*$ component which is the Poisson bracket, that means it
can be written as
$$
\sum_{i \in~\text{clusters}} \qquad \sum_{\text{coordinates}~x, y~\text{in cluster}~i} \{x,y\}
\{x\}_3 \otimes y
$$
In particular, there is a sense in which they have $B_3 \otimes C^*$
components $\{a\}_3 \otimes b$ which are purely antisymmetric under
exchange of $a$ and $b$.  (Though I don't know how to make this property
well-defined because to rewrite the $C^*$ component using multiplicative
identities which $B_3$ doesn't have).

Goncharov replies:  I do not know how to formulate a Poisson property of ${a}_3 \otimes b$
since $\{a\}_3 = \{a^{-1}\}_3$. 

In fact, one reason of talking about Poisson properties before was a
a  way to express / formalize  the idea that
the terms which appear in $B_2 \wedge B_2$ should not be ``far away" in
the cluster polytope. Similar hope, of course, for $B_3 \otimes C^*$.

\item The main question I have is this:

Q. What are all relations between the $B_2 \wedge B_2$ parts of the
pentagon functions?

Precisely:

Q1) Are there any ``non-trivial", (and also, just any) relations
between $B_2 \wedge B_2$ parts of the pentagon functions? ``Non-trivial"
means the proof
involves the 5-term identities: the cancellation holds for the sum in
$B_2 \wedge B_2$, but not on in $Z[P^1]$, or just assuming $\{x\}_2 = -
\{x^{-1}\}_2$.

I imagine there are some ``basic" relations from which any other should follow.
The question is well posed for any cluster variety:
Take all pentagons one can find there, and study linear relations between
$B_2 \wedge B_2$ parts of the pentagon functions of those pentagons.
It is a finite problem for finite type, but well posed problem even in general.

WHY IMPORTANT:

1) Any linear relation for
$B_2 \wedge B_2$ parts of the pentagon functions must be upgraded then
to an identity

sum of the actual pentagon functions is a sum of $\Li_4$'s of some variables.

It would be terrific if this variables of $\Li_4$ can be taken cluster
variables --
so in finite cases, like $D_4$, $E_6$ one can just ask computer.

So the question is, do we have any non-trivial identity?

2) In the next step (I skip explanations) one gets functional
equations for 4-logs.
They will be ``Buld on" functional equations for 3-logs, presumably the
40 term ones.

I hope that all questions have answers understood via
geometry of the cluster polytope.

\item Although for the amplitude $A_3$ function may be much better,
the pentagon functions are probably more fundamental.

We were looking for a loong time for a simple 2-variable
weight 4 function with simple coproduct (testing versions of
Omega-functions of Dixon)
but now it is clear that it is the pentagon function.

\item \underline{Comment from MP}

- In $A_3$ there are no relations between pentagon functions. 

- In $A_4$ there are no relations between pentagon functions except ``trivial ones": there are 28 pentagonal faces but only 21 distinct $A_2$ subalgebras because there are things like $A_1 \times A_2$ in the polytope. These 21 $A_2$s lead to 21 independent pentagon functions.

- In $D_4$ there are \emph{a priori} 36 pentagon functions, but only 34 of them are independent. These two relations depend non-trivially on 5 term identities among $B_2$ elements on the $B_2 \wedge B_2$ part; and on the 40 term $\Li_3$ identity on the $B_3 \otimes C^*$ part. Similarly there are 12 independent $A_3$ subalgebras, and corresponding 12 $A_3$ functions - of which only 9 are independent. These three relations surely follow from the relations among pentagons but I haven't worked out the details.

- Regarding identities between $A_3$ functions: the two identities between the 36 pentagon functions involve 24 pentagons each. One can understand the origin of these identities quite beautifully in terms of the $A_3$ functions. As said before there are 12 $A_3$ functions, but only 9 independent. Each of the 3 identities involves 4 $A_3$ functions - and since each $A_3$ has 6 pentagons one gets the 24. These identities are quite pretty geometrically: each $A_3$ has three squares, so visualize each of them as a triangle with squares at the vertices. Then we can get an identity by taking four $A_3$s and gluing them together by their squares in such a way as to get a tetrahedron. 

Algebraically, the $B_2 \wedge B_2$ components of the 4 $A_3$ functions share terms pairwise, and adding them up with the right sign they all cancel. 

I have checked that the 3 $A_3$ identities generate the 2 independent pentagon identities. The fact that there are 2, not 3 identities then means that the three tetrahedra must be also glued up somehow by their pentagons.

The conclusion is that the seemingly non-trivial identities among pentagon functions are better understood as simple geometric consequences of gluing up $A_3$'s, where they become trivial.

\item How to see geometrically the 4 $A_3$ polytopes which produce a relation?

Means: $D_4$ means a configuration of 6 points on $P^2$. 

Each $A_3$ polytope corresponds to certain configuration of 6 points on $P^1$. 
So a relation corresponds to cerrtain way to produce FOUR configurations of 
6 points on $P^1$ from configurations of 6 points on $P^2$.
How to describe it geomtrically, in terms of the projective geometry?
(Like take two lines formed by the two pairs of points, find their intersection etc.) 

This relations in $D_4$ should lead to CLUSTER  functional equations for $\Li_4$. 
One way is probably to take the amplitude for 7 points, it has several presentations. 
They are equal and thus the sum of the corresponding $A_3$ functions should be equal, 
modulo $\Li_4$'s entering the game. This means a CLUSTER relation for $\Li_4$ - 
EXTREMELY VALUABLE. 

\item \underline{Comment from MS}

(1) Sasha has a proposal for
how to write down ``pentagon" (or $A_3$) functions for weight $> 4$.  Let
$f_4(x,y)$ be your favorite weight 4 function, then we consider the
following recursive way to build up ``cluster polylogarithm" functions
at weight $> 4$, which we call $f_k$.  The idea is just to make the
following ansatz for the coproduct
$$
\delta f_k(x,y) = f_{k-1}(x,y) \otimes x/y + ~\text{stuff}
$$
where ``stuff" is to be determined by imposing the condition that
$\delta$ should annihilate the right-hand side.  The natural ansatz,
given the cluster structure, is that
$$
\text{stuff}~ = \sum_{i,j,m} {x_i}_m \wedge {x_j}_{k-m}
$$
The sum is somewhat schematic:  $i,j$ run over the cluster coordinates,
and m runs from 2 to $k-2$.  The sum just indicates all possible terms
which could possibly appear there.  The idea is to make an ansatz (put
a free coefficient in front of each term) and see what works; that is,
solve that $\delta$ should kill the right-hand side.  This might give a
recursive way to define cluster polylogarithms for all higher weight!

(2) Sasha remains interested in finding
non-trivial $\Li_4$ cluster identities.  Right now we don't know any.
One reason we need to understand all possible identities is that in
doing calculations like the one above, $\delta$ can only annihilate the
right-hand side due to nontrivial identities.  So first we should
understand what kinds of identities exist!

(3) An earlier discussion wish Sasha reminded us that we really need
to start looking at collinear limits.  Presumably only very special $A_3$
functions will be well-defined under collinear limits; random ones
won't be.

(4) It might also be possible, more conjecturally, to naturally assign
functions to higher algebras using a recursive method like the one
described in (1) above.  For example, if you wanted to assign a weight
5 function to an $A_4$ algebra, that means you would have to propose an
element of $L_4 \otimes C^*$ and an element of $B_3 \otimes B_2$, which
satisfy the integrability.  You could try some kind of ansatz where
the $L_4$ is given by some sum of pentagon functions, and for the $B_3
\otimes B_2$ maybe some kind of $A_2$ and $A_1$ subalgebras.  Well I'm not
being very clear but the idea is maybe to recursively define higher
weight functions based on geometric data, in terms of lower weight
functions associated to subalgebras.

\item Here is a question on cluster nature of the funct eq. for 3-log.

Observe that: 

1. Cluster $\mathcal{A}$-coordinates are assigned to the top dimensional faces of the cluster polytope. 

2. Cluster $\mathcal{X}$-coordinates are assigned to the oriented edges of the cluster polytope. 

In the $D_4$ case the codimension one cells are $A_3$ polytopes or $A_2 \times A_1$ polytopes. 
The best way to think about them geometrically - as of cluster $\mathcal{A}$-coordinates.

So the 40-term funct. eq. means that we have a sum 
$$
(\text{Pentagon}_i) \otimes A_3~\text{polytope}_i
\quad or \quad A_2 \otimes A_1~\text{polytope}_i
$$

Question: what the coproduct of that 40 term identity looks like this way? 

In our paper, I wrote it in the Appendix explicitly as 
Pentagon $\otimes$ $\mathcal{A}$-coordinate expression - so
what it means geometrically? 

\item \underline{Comment from MP}

$D_4$ contains no $A_2\times A_1$ but it does contain four cubes $A_1 \times A_1 \times A_1$ and 12 $A_3$'s. As you say, to each of these objects there corresponds a unique $\mathcal{A}$-coordinate - but these are reflected in the multiplicatively independent $\mathcal{X}$-coordinates, of which there are indeed 16. So, I investigated the coproduct of the 40 term identity and one gets a sum of terms of the form
$$
\sum \text{pentagon} * \text{multiplicative~independent}~\mathcal{X}\text{-coordinate}
$$
However, not all possible $\mathcal{X}$-coordinates appear! Out of the 16 only 12 appear, which leads me to conjecture that the desired expression is
$$
\sum \text{pentagon} \otimes A_3
$$
It is not clear to me however how the association goes: which pentagon(s) go with which $A_3$? To see this I'll have to write the $\mathcal{X}$-coordinates in terms of the $\mathcal{A}$-coordinates to make the map precise.

\item I mean try to understand contributions of the ${\mathcal{A}}$-coordinates, since they are multiplicatively independent. A set of independent $\mathcal{X}$-coordinates is not quite well defined.

\item Let us take the pentagon functions assigned to all pentagons on
the cluster polytope of type $E_7$, that is $\Conf_7(P^3)$.

Let us skewsymmetrise each of these pentagon functions.

1) How many different skewsymmetric functions on $\Conf_7(P^3)$ we get
this way?

2) If not that many, can one list them?

To clarify:

Take 7 points in $P^3$, so far cyclically ordered.
The space of such cyclically ordered 7-tuples has $E_7$ type cluster structure.

So there are plenty pentagons on its Stasheff polytope. Take one of them.
This means that given 7 points in $P^3$, $(z_1, ..., z_7)$
we created a configuration of 5 points on $P^1$, say described by two
cluster coordinates

$$
X(z_1, ..., z_7), Y(z_1, ..., z_7)
$$

Now make pentagon "function" meaning element in

$$
B_3 \otimes C^* + B_2 \wedge B_2
$$
-- depending on those $(z_1, ..., z_7)$.

Skew symmetrise it means apply 7! permutations to $(z_1, ..., z_7)$, for
each of them
calculate the
$B_3 \otimes C^* + B_2 \wedge B_2$ invariant, and take sum with signs.

In practice most of them immediately die, so we get zero.

So some of these pentagon are distinguished by surviving skew symmetrization.

\end{enumerate}

\end{document}
