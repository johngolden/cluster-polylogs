\pdfoutput=1


\documentclass[12pt]{article}

\usepackage{amssymb}
\usepackage{amsfonts}
\usepackage{amsmath}
\usepackage{fancyhdr} 
\usepackage[all]{xy}

\DeclareMathOperator{\B}{B}
\DeclareMathOperator{\Conf}{Conf}
\DeclareMathOperator{\Gr}{Gr}
\DeclareMathOperator{\Id}{Id}
\DeclareMathOperator{\Li}{Li}
\DeclareMathOperator{\Lie}{Lie}

\def\x{\mathcal{X}}
\def\a{\mathcal{A}}

\def\ket#1{\langle #1 \rangle}


\makeindex
\oddsidemargin -0.04cm \evensidemargin -0.04cm
\topmargin -0.25cm \textwidth 16.59cm \textheight 20.5cm \headheight 15pt

\begin{document}

\thispagestyle{fancyplain}
 
\fancyhf{}
 
\cfoot{\fancyplain{}{\thepage}}

\lhead{\textbf{Cluster Automorphisms} \hfill \today}

In this note we describe cluster automorphisms for the finite cluster algebras $A_2, A_3, A_4, D_4, A_5, D_5, E_6$. We then describe ways of defining non-classical weight-4 cluster polylogarithms which respect these automorphisms.\\

{\Huge\underline{\(A_2\)}}  $\simeq\Gr(2,5)$ \quad clusters: 5 \qquad $a$-coordinates: 5 \qquad $x$-coordinates: 10 \\ \\
$\x$-coordinate seed: $x_1 \to x_2$\\ \\
$\a$-coordinate seed, with frozen coordinates, in both $\Gr(2,5)$ as well as generic coordinates: 
\begin{equation}
\begin{gathered}
\begin{xy} 0;<1pt,0pt>:<0pt,-1pt>::
(25,25) *+{\langle 13\rangle} ="0",
(75,25) *+{\langle 14\rangle} ="1",
(125,25) *+{\framebox[5ex]{$\langle 15\rangle$}} ="2",
(125,75) *+{\framebox[5ex]{$\langle 45\rangle$}} ="3",
(75,75) *+{\framebox[5ex]{$\langle 34\rangle$}} ="4",
(25,75) *+{\framebox[5ex]{$\langle 23\rangle$}} ="5",
(0,0) *+{\framebox[5ex]{$\langle 12\rangle$}} ="6",
(145,75) *+{},
"0", {\ar"1"},
"4", {\ar"0"},
"0", {\ar"5"},
"6", {\ar"0"},
"1", {\ar"2"},
"3", {\ar"1"},
"1", {\ar"4"},
\end{xy}\quad\Leftrightarrow\quad
\begin{xy} 0;<1pt,0pt>:<0pt,-1pt>::
(25,25) *+{a_1} ="0",
(75,25) *+{a_2} ="1",
(125,25) *+{\framebox[5ex]{$f_5$}} ="2",
(125,75) *+{\framebox[5ex]{$f_4$}} ="3",
(75,75) *+{\framebox[5ex]{$f_3$}} ="4",
(25,75) *+{\framebox[5ex]{$f_2$}} ="5",
(0,0) *+{\framebox[5ex]{$f_1$}} ="6",
(145,75) *+{},
"0", {\ar"1"},
"4", {\ar"0"},
"0", {\ar"5"},
"6", {\ar"0"},
"1", {\ar"2"},
"3", {\ar"1"},
"1", {\ar"4"},
\end{xy}
\end{gathered}
\end{equation}
$\Rightarrow x_1 = \frac{f_1 f_3}{f_2 a_2}, x_2 = \frac{a_1 f_4}{f_3 f_5}$ 

\end{document}
