
\pdfoutput=1
\documentclass[11pt, reqno,preprint]{article}
\usepackage{jheppub}
\usepackage{epsfig}
\usepackage{amssymb}
\usepackage{amsmath}
\usepackage{mathrsfs}
%\usepackage{cite}
\usepackage{hyperref}
\usepackage{multirow}
\usepackage{feynmp-auto}

\def\be{\begin{equation}}
\def\ee{\end{equation}}
\def\ba{\begin{eqnarray}}
\def\ea{\end{eqnarray}}
\newcommand{\bea}{\begin{eqnarray}}
\newcommand{\eea}{\end{eqnarray}}
\def\nl{\nonumber\\}
\def\Li{\textrm{Li}}
\def\l{\langle}
\def\r{\rangle}
\def\eps{\epsilon}
\def\uo{\underline{0}}
\def\uone{\underline{1}}
\def\ut{\underline{t}}
\def\z{x}
\def\zb{y}
\newcommand{\del}{\partial}
\def\fig#1{fig.~{\ref{#1}}}
\def\Fig#1{Fig.~{\ref{#1}}}
\def\figs#1#2{figs.~{\ref{#1}} and {\ref{#2}}}
\def\Figs#1#2{Figs.~{\ref{#1}} and {\ref{#2}}}
\def\Sect#1{Section~{\ref{#1}}}
\def\sect#1{section~{\ref{#1}}}
\def\eqn#1{eq.~(\ref{#1})}
\def\Eqn#1{Equation~(\ref{#1})}
\def\eqns#1#2{eqs.~(\ref{#1}) and~(\ref{#2})}
\def\Eqns#1#2{Eqs.~(\ref{#1}) and~(\ref{#2})}
\def\tab#1{table~{\ref{#1}}}
\def\Tab#1{Table~{\ref{#1}}}
\def\tabs#1#2{table~{\ref{#1}} and~{\ref{#1}}}
\def\Tabs#1#2{Table~{\ref{#1}} and~{\ref{#1}}}
\def\Eqn#1{Equation~(\ref{#1})}
\def\eqn#1{eq.~(\ref{#1})}
\def\eqns#1#2{eqs.~(\ref{#1}) and~(\ref{#2})}
\def\Eqns#1#2{Eqs.~(\ref{#1}) and~(\ref{#2})}

\newcommand{\cP}{{\cal P}}
\def\lr{\leftrightarrow}

\def\draftnote#1{{\bf [#1]}}


\def\blue#1{{\color{blue}#1}}

\title{Novel Cluster-Algebraic Decompositions of Integrals and Amplitides in Planar ${\cal N}=4$}

\author{John~Golden,$^{1,2}$}
\author{Andrew~J.~McLeod$^{2,3,4}$}


\affiliation{$^1$ Michigan Center for Theoretical Physics and
Randall Laboratory of Physics, Department of Physics,
University of Michigan
Ann Arbor, MI 48109, USA}

\affiliation{$^2$ Kavli Institute for Theoretical Physics, 
UC Santa Barbara, Santa Barbara, CA 93106, USA}

\affiliation{$^3$ SLAC National Accelerator Laboratory,
Stanford University, Stanford, CA 94309, USA}

\affiliation{$^4$ Niels Bohr International Academy, Blegdamsvej 17, 2100 Copenhagen, Denmark}

\abstract{\dots}

%\preprint{
%\begin{flushright} DESY ??--??? \\ SLAC--PUB--?????
%\end{flushright}
%}




\begin{document}
\hypersetup{pageanchor=false}
\maketitle
\hypersetup{pageanchor=true}
\begin{fmffile}{feyndiags}


\section{Introduction}

\begin{itemize}
\item coproducts, cobrackets, symbols
\item known cluster-algebraic properties of amplitudes (especially two-loop MHV)
\item while symbol-level cluster adjacency may encode precisely the (extended) Steinmann constraints, it's not clear what the cluster structure at the cobracket level encodes (it precisely cannot be Steinmann, which is concerned with products)
\end{itemize}

\section{Amplitudes in Planar ${\cal N} = 4$}

\section{Cluster Polylogarithms}

\begin{itemize}
\item previous work (Golden et al)
\item A2 and A3 functions
\item we want to flesh out the definition of cluster polylogarithms
\end{itemize}

\subsection{Cluster Automorphisms}
\begin{itemize}
\item (or maybe this goes in the next section, if this section is just review)
\end{itemize}


\section{Novel Cluster-Algebraic Poisson Cobracket Structures}

\begin{itemize}
\item {\emph{this section all at cobracket level}}
\item outline general approach, finite cluster algebras, cluster automorphisms, nested structure
\item identities between A2/A3 in these larger algebras 
\end{itemize}

\subsection{The D5 Function}

\subsection{The E6 Function}

\subsection{Beyond Seven Points}

\begin{itemize}
\item failure of D4/A4/A5 to have interesting functions that can be written in terms of A3 functions (but what about NMHV/omega integrals?)
\item what can we encode in terms of these functions? (Just seven points, or higher points as well? If we can get an eight-point result in this section at cobracket level, we can include 8-pt BDS-like analysis for the next section; if not, we should punt it off to a later paper)
\item {\emph{hopefully we can come up with additional interesting cobracket-level functions by looking at 7-pt NMHV and omega integrals}}
\end{itemize}

\section{Uplifting Cobrackets to Coproducts}

\begin{itemize}
\item outline general method (and maybe offer our even fuller refined definition of cluster polylogarithms)
\item symbol-level cluster adjacency as a further guiding principle --- this property is automatically inherited by all subalgebras of cluster-adjacent function (I believe?)
\item Steinmann/first entry/Qbar eqn helps by reducing the space of symbols we care about
\end{itemize}

\subsection{The A2 Function}

\subsection{The A3 Function}

\subsection{The D5 Function}

\subsection{The E6 Function}

\subsection{Cluster Adjacency Beyond the Symbol}
\begin{itemize}
\item present all cluster-adjacency properties we can find beyond symbol level (hopefully all-$n$ result for two loops?)
\end{itemize}

\subsection{More General Cluster Polylogarithms}

\begin{itemize}
\item the full (nested) space of cluster polylogarithms (on finite algebras) is an interesting place to look for full symbol-level cluster-algebraic decompositions of amplitudes and integrals
\item we leave analysis of this full space to later work, pointing out in particular we will explore linear combinations there
\item could positivity (in the positive region) play a role?
\item is this full space endowed with any interesting topological/gemotric structure (like a filtration)?
\end{itemize}

\section{Conclusion}

\begin{itemize}
\item not yet sure what this cluster structure at cobracket level encodes physically (and point out why everything in our current list of candidates is ruled out)
\item future work to systematically explore this space at weight 4... and would also like to carry out such an analysis at weight 6
\end{itemize}

\bibliographystyle{ieeetr}

\bibliography{cluster_algebra_note}

\end{fmffile}
\end{document}
