\pdfoutput=1


\documentclass[12pt]{article}

\usepackage{amssymb}
\usepackage{amsfonts}
\usepackage{amsmath}
\usepackage{fancyhdr} 

\DeclareMathOperator{\B}{B}
\DeclareMathOperator{\Conf}{Conf}
\DeclareMathOperator{\Gr}{Gr}
\DeclareMathOperator{\Id}{Id}
\DeclareMathOperator{\Li}{Li}
\DeclareMathOperator{\Lie}{Lie}

\def\ket#1{\langle #1 \rangle}
\def\a{\mathcal{A}}
\def\x{\mathcal{X}}
\def\acoord{$\mathcal{A}$-coordinate}
\def\acoords{$\mathcal{A}$-coordinates}
\def\xcoord{$\mathcal{X}$-coordinate}
\def\xcoords{$\mathcal{X}$-coordinates}

\def\fd5{f_{D_5}}
\def\fa3{f_{A_3}}
\def\a2{A_2}
\def\a3{A_3}
\def\d5{D_5}
\def\r27{R^{(2)}_7}
\def\e6{E_6}

\def\bb2{B_2\wedge B_2}

\makeindex
\oddsidemargin -0.04cm \evensidemargin -0.04cm
\topmargin -0.25cm \textwidth 16.59cm \textheight 20.5cm \headheight 15pt
\setlength{\parindent}{0pt}

\begin{document}

\thispagestyle{fancyplain}
 
\fancyhf{} 
 
\cfoot{\fancyplain{}{\thepage}}

\lhead{\textbf{Describing the $\d5$ function} \hfill \today}

In this note we analyze ``the'' $\d5$ function, $\fd5$, which we generate by:
\begin{itemize}
	\item start with an ansatz of all distinct $\fa3$'s in $\d5$,
	\item impose antisymmetry under all of the $\d5$ automorphisms $\{\sigma, \tau, \mathbb{Z}_2\}$,
	\item take the fully symmetric sum of this $\fd5$ over $\e6$ and fit to $\r27$.
\end{itemize}
The resulting function has 1 free parameter, which represents an internal degree of freedom in $\fd5$ that cancels in the symmetric sum over $\e6$. Later in the note we will explore tuning these free parameters to make some (tbd) nice property manifest.

\subsubsection*{Reader's Digest: Deriving $\fd5$}

There are 65 distinct $\a3$'s in $\d5$. Of these, only 42 are linearly independent. Imposing full antisymmetry leaves only 5 degrees of freedom. Requiring that the fully symmetric sum of $\fd5$ gives $\r27$ fixes 4 of these parameters, leaving us with only 1. 

\subsubsection*{Describing $\fd5$}

I can express $\fd5$ as a sum over 13 $\a3$'s -- maybe less? What is the fewest number of $\bb2$ terms I can express it in terms of?

\subsubsection*{Representing $\r27$ in terms of $\fd5$}

\begin{equation}
	\r27 = \sum_{i=1}^{14} \pm \fd5^{(i)}
\end{equation}
where $i$ indexes the 14 $\d5$'s in $\e6$, and the $\pm$ is to get the symmetries to work out right -- half of the $\fd5$'s have $+$ and half have $-$. This needs sharper restatement. 

\subsubsection*{Constraining the remaining parameter}

Can this last parameter be tuned to make $\fd5$ have some other nice amplitudes property? For example, can $\fd5$ be well-defined in the collinear limit? What are other properties it can have?


\end{document}
