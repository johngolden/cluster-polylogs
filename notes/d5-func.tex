\pdfoutput=1


\documentclass[12pt]{article}

\usepackage{amssymb}
\usepackage{amsfonts}
\usepackage{amsmath}
\usepackage{fancyhdr} 
\usepackage[all]{xy}

\DeclareMathOperator{\B}{B}
\DeclareMathOperator{\Conf}{Conf}
\DeclareMathOperator{\Gr}{Gr}
\DeclareMathOperator{\Id}{Id}
\DeclareMathOperator{\Li}{Li}
\DeclareMathOperator{\Lie}{Lie}

\def\ket#1{\langle #1 \rangle}
\def\a{\mathcal{A}}
\def\x{\mathcal{X}}
\def\acoord{$\mathcal{A}$-coordinate}
\def\acoords{$\mathcal{A}$-coordinates}
\def\xcoord{$\mathcal{X}$-coordinate}
\def\xcoords{$\mathcal{X}$-coordinates}

\def\fd5{f_{D_5}}
\def\fa3{f_{A_3}}
\def\a2{A_2}
\def\a3{A_3}
\def\d5{D_5}
\def\e6{E_6}
\def\r27{R^{(2)}_7}
%\def\r28{R^{(2)}_8}

\def\bb2{B_2\wedge B_2}

\makeindex
\oddsidemargin -0.04cm \evensidemargin -0.04cm
\topmargin -0.25cm \textwidth 16.59cm \textheight 20.5cm \headheight 15pt
\setlength{\parindent}{0pt}

\begin{document}

\thispagestyle{fancyplain}
 
\fancyhf{} 
 
\cfoot{\fancyplain{}{\thepage}}

\lhead{\textbf{Describing the $\d5$ function} \hfill \today}

In this note we analyze ``the'' $\d5$ function, $\fd5$, which we generate by:
\begin{itemize}
	\item start with an ansatz of all distinct $\fa3$'s in $\d5$,
	\item impose antisymmetry under all of the $\d5$ automorphisms $\{\sigma, \tau, \mathbb{Z}_2\}$,
	\item take the fully symmetric sum of this $\fd5$ over $\e6$ and fit to $\r27$.
\end{itemize}
The resulting function has 1 free parameter, which represents an internal degree of freedom in $\fd5$ that cancels in the symmetric sum over $\e6$. Later in the note we will explore tuning these free parameters to make some (tbd) nice property manifest.

\subsubsection*{Reader's Digest: Deriving $\fd5$}

We'll be working with the following $\d5$ seed cluster:

\begin{equation}
\begin{gathered}
\begin{xy} 0;<1pt,0pt>:<0pt,-1pt>::
(0,25) *+{x_1} ="0",
(50,25) *+{x_2} ="1",
(100,25) *+{x_3} ="2",
(150,0) *+{x_4} ="3",
(150,50) *+{x_5} ="4",
"0", {\ar"1"},
"1", {\ar"2"},
"2", {\ar"3"},
"2", {\ar"4"},
\end{xy}
\end{gathered}
\end{equation}

There are 65 distinct $\a3$'s in $\d5$. Of these, only 42 produce linearly independent $\fa3$'s. Imposing full $\d5$ antisymmetry on this collection of $\fa3$'s leaves only 5 degrees of freedom. Requiring that the full $\e6$-symmetric sum of $\fd5$ gives $\r27$ fixes 4 of these parameters, leaving us with only 1 degree of freedom. Of course when we are looking for a particular representation of $\fd5$ we have 24 degrees of freedom (23 of which are equivalent to adding zero).\\

It would be nice to find a property that fixes some of these parameters that does not rely on explicitly knowing $\r27$. Of course a cluster-y property would be great, but even a physics one would be nice. 

\subsubsection*{Describing $\fd5$}

Because of the 1 degree of freedom, it is difficult to describe in detail the properties of $\fd5$ until we have set this value. Furthermore, we likely want to keep this parameter free so that we have some freedom when we try to express $\r27$ in terms of $\fd5$. \\

The piece that does not cancel in the full $\e6$ sum can be represented in terms of 13 $\a3$'s (maybe less, I haven't done an exhaustive check). The following 8 enter with coefficient $+1/2$:
\begin{gather*}
	x_1\to x_2\to x_3 \left(1+x_5\right),\quad x_2\to x_3\to x_5,\quad \frac{x_1 x_2}{1+x_1}\to x_3\to x_4,\quad x_1\left(1+x_2\right)\to \frac{x_2 x_3\left(1+x_4\right)}{1+x_2}\to x_5,\\ \\
	\frac{1}{x_4}\to x_3 \left(1+x_4\right)\to x_5,\quad \frac{1+x_3}{x_3x_4}\to x_2 \left(1+x_3+x_3 x_4\right)\to \frac{x_3x_5}{1+x_3},\\ \\
	\frac{1+x_2+x_2 x_3}{x_2 x_3 x_4}\to x_1\left(1+x_2+x_2 x_3+x_2 x_3 x_4\right)\to \frac{x_2x_3 x_5}{1+x_2+x_2 x_3},\\ \\
	\frac{x_1 x_2 x_3x_5}{1+x_1+x_1 x_2+x_1 x_2 x_3}\to\frac{\left(1+x_1\right) x_3 x_4}{\left(1+x_3\right)\left(1+x_1+x_1 x_2+x_1 x_2 x_3+x_1 x_2 x_3x_4\right)}\to \frac{1+x_1+x_1 x_2+x_1 x_2x_3}{\left(1+x_1\right) x_3},\\
\end{gather*}
and these 5 enter with coefficient $-1/2$:
\begin{gather*}
	x_1\to x_2\to x_3 \left(1+x_4\right),\quad x_2\to x_3\to x_4,\quad \frac{x_1 x_2}{1+x_1}\to x_3\to x_5,\quad x_1\left(1+x_2\right)\to \frac{x_2 x_3\left(1+x_5\right)}{1+x_2}\to x_4,\\ \\
	\frac{x_1 x_2 x_3x_4}{1+x_1+x_1 x_2+x_1 x_2 x_3}\to\frac{\left(1+x_1\right) x_3 x_5}{\left(1+x_3\right)\left(1+x_1+x_1 x_2+x_1 x_2 x_3+x_1 x_2 x_3x_5\right)}\to \frac{1+x_1+x_1 x_2+x_1 x_2x_3}{\left(1+x_1\right) x_3}.
\end{gather*}

I don't have a nice representation of the piece of the function that cancels in the full $\e6$ sum. The shortest representation I have involves 17 $\a3$'s. 

\subsubsection*{Representing $\r27$ in terms of $\fd5$}

Next step: write down the first $\d5$ and then show how $\sigma, \tau,$ and $\mathbb{Z}_2$ $\e6$ symmetries create the other 13 $\d5$'s. Use this as a way to better express:

\begin{equation}
	\r27 = \sum_{i=1}^{14} \pm \fd5^{(i)}
\end{equation}
where $i$ indexes the 14 $\d5$'s in $\e6$, and the $\pm$ is to get the symmetries to work out right -- half of the $\fd5$'s have $+$ and half have $-$. This needs sharper restatement. 

\subsubsection*{Constraining the remaining parameter}

In the $7\to6$ collinear limit, the 14 $\fd5$'s have the following behavior:

\begin{itemize}
	\item $\{1,4,5,7,10,11,12,14\}$ vanish identically
	\item $\{2,3,8,9\}$ are non-zero, $2=-3$ and $8=-9$
	\item $\{6,13\}$ each vanish if you set the remaining parameter to 1, and otherwise $6=-13$
\end{itemize}
(where $\{1,\ldots,14\}$ label the 14 $\d5$'s in the arbitrary way my code decided to do things. I'll clean this up tomorrow.)\\ 

This provides some motivation (though not 100\% clear-cut) for determining this parameter. 

\subsubsection*{A brief idea for $\Gr(4,8)$}

Are there $\e6$'s in $\Gr(4,8)$ which have $\{v,z\}$ squares? What if I try the 42-term $\{v,z\}$ expression for $\r27 \sim f_{E_6}$ and see if that evaluates to ``good'' things on (at least some of) the $\e6$'s I've found in $\Gr(4,8)$.

\end{document}
