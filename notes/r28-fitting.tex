\pdfoutput=1


\documentclass[12pt]{article}

\usepackage{amssymb}
\usepackage{amsfonts}
\usepackage{amsmath}
\usepackage{fancyhdr} 

\DeclareMathOperator{\B}{B}
\DeclareMathOperator{\Conf}{Conf}
\DeclareMathOperator{\Gr}{Gr}
\DeclareMathOperator{\Id}{Id}
\DeclareMathOperator{\Li}{Li}
\DeclareMathOperator{\Lie}{Lie}

\def\ket#1{\langle #1 \rangle}


\makeindex
\oddsidemargin -0.04cm \evensidemargin -0.04cm
\topmargin -0.25cm \textwidth 16.59cm \textheight 20.5cm \headheight 15pt

\begin{document}

\thispagestyle{fancyplain}
 
\fancyhf{}
 
\cfoot{\fancyplain{}{\thepage}}

\lhead{\textbf{Fitting $R^{(2)}_8$ with Cluster Polylogarithms} \hfill \today}

In this note we explore different attempts at expressing the $B_2\wedge B_2$ component of $R^{(2)}_8$ with cluster polylogarithms.

First let us define ``good'' $\mathcal{X}$-coordinates: an $\mathcal{X}$-coordinate $R$ is good if $R$ and $1+R$ can be expressed as ratios of $\mathcal{A}$-coordinates of the form
\begin{align}\label{def:letters}
\begin{split}
\ket{i\,\,i{+}1\,\,jk},& \quad 
\ket{i(i{-}1\,\,i{+}1)(j\,\,j{+}1)(k\,\,k{+}1)}, \\ 
\ket{i\,\,i{+}1\,\,\bar{j}\cap\bar{k}},& \quad
\ket{i(i{-}2\,\,i{-}1)(i{+}1\,\,i{+}2)(j\,\,j{+}1)}.
\end{split}
\end{align}
An algebra is good if it contains only good $\mathcal{X}$-coordinates. 

\subsubsection*{Fitting with $A_3$'s}

There are 1600 good $A_3$'s in $\Gr(4,8)$, with only 513 actual degrees of freedom in $B_2\wedge B_2$. An ansatz of these functions can be fit to $R^{(2)}_8$. A zero-thought choice of free parameters leaves us with an expression involving 223 $A_3$'s.

\subsubsection*{Fitting with $A_4$'s}

There are 496 good $A_4$'s. 

\subsubsection*{Fitting with $D_4$'s}

There are 24 good $D_4$'s. 

\subsubsection*{Fitting with $A_5$'s}

There are 56 good $A_5$'s. No choice of $f_{A_5}$ can fit $R^{(2)}_8$.

\subsubsection*{Fitting with $D_5$'s}

There are no good $D_5$'s. Furthermore, no $D_5$ contains enough good $A_3$'s to allow us to tune the free parameters in $f_{D_5}$ in order to have only good $A_3$'s appear in $f_{D_5}$. In other words, any representation of $R^{(2)}_8$ in terms of $f_{D_5}$ will necessarily involve ``bad'' ratios inside the $f_{D_5}$'s which must cancel in the full sum. This is upsetting! 

\subsubsection*{Fitting with $E_6$'s}

There are no good $E_6$'s. There are 216 $E_6$'s which have seed clusters involving only good $\mathcal{X}$-coordinates. None of these 216 $E_6$'s have sufficient $\{v,z\}$-squares to constitute a completely good $\sim f_{E_6} \sim R^{(2)}_7$. The most $\{v,z\}$-squares inside one of these $E_6$'s is 33, and there are 64 $E_6$'s which have at least 30 (out of 42) good $\{v,z\}$ squares. 

\end{document}
